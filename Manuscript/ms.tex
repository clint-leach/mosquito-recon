% Template for PLoS
% Version 3.4 January 2017
%
% % % % % % % % % % % % % % % % % % % % % %
%
% -- IMPORTANT NOTE
%
% This template contains comments intended 
% to minimize problems and delays during our production 
% process. Please follow the template instructions
% whenever possible.
%
% % % % % % % % % % % % % % % % % % % % % % % 
%
% Once your paper is accepted for publication, 
% PLEASE REMOVE ALL TRACKED CHANGES in this file 
% and leave only the final text of your manuscript. 
% PLOS recommends the use of latexdiff to track changes during review, as this will help to maintain a clean tex file.
% Visit https://www.ctan.org/pkg/latexdiff?lang=en for info or contact us at latex@plos.org.
%
%
% There are no restrictions on package use within the LaTeX files except that 
% no packages listed in the template may be deleted.
%
% Please do not include colors or graphics in the text.
%
% The manuscript LaTeX source should be contained within a single file (do not use \input, \externaldocument, or similar commands).
%
% % % % % % % % % % % % % % % % % % % % % % %
%
% -- FIGURES AND TABLES
%
% Please include tables/figure captions directly after the paragraph where they are first cited in the text.
%
% DO NOT INCLUDE GRAPHICS IN YOUR MANUSCRIPT
% - Figures should be uploaded separately from your manuscript file. 
% - Figures generated using LaTeX should be extracted and removed from the PDF before submission. 
% - Figures containing multiple panels/subfigures must be combined into one image file before submission.
% For figure citations, please use "Fig" instead of "Figure".
% See http://journals.plos.org/plosone/s/figures for PLOS figure guidelines.
%
% Tables should be cell-based and may not contain:
% - spacing/line breaks within cells to alter layout or alignment
% - do not nest tabular environments (no tabular environments within tabular environments)
% - no graphics or colored text (cell background color/shading OK)
% See http://journals.plos.org/plosone/s/tables for table guidelines.
%
% For tables that exceed the width of the text column, use the adjustwidth environment as illustrated in the example table in text below.
%
% % % % % % % % % % % % % % % % % % % % % % % %
%
% -- EQUATIONS, MATH SYMBOLS, SUBSCRIPTS, AND SUPERSCRIPTS
%
% IMPORTANT
% Below are a few tips to help format your equations and other special characters according to our specifications. For more tips to help reduce the possibility of formatting errors during conversion, please see our LaTeX guidelines at http://journals.plos.org/plosone/s/latex
%
% For inline equations, please be sure to include all portions of an equation in the math environment.  For example, x$^2$ is incorrect; this should be formatted as $x^2$ (or $\mathrm{x}^2$ if the romanized font is desired).
%
% Do not include text that is not math in the math environment. For example, CO2 should be written as CO\textsubscript{2} instead of CO$_2$.
%
% Please add line breaks to long display equations when possible in order to fit size of the column. 
%
% For inline equations, please do not include punctuation (commas, etc) within the math environment unless this is part of the equation.
%
% When adding superscript or subscripts outside of brackets/braces, please group using {}.  For example, change "[U(D,E,\gamma)]^2" to "{[U(D,E,\gamma)]}^2". 
%
% Do not use \cal for caligraphic font.  Instead, use \mathcal{}
%
% % % % % % % % % % % % % % % % % % % % % % % % 
%
% Please contact latex@plos.org with any questions.
%
% % % % % % % % % % % % % % % % % % % % % % % %

\documentclass[10pt,letterpaper]{article}
\usepackage[top=0.85in,left=2.75in,footskip=0.75in]{geometry}

% amsmath and amssymb packages, useful for mathematical formulas and symbols
\usepackage{amsmath,amssymb}

% Use adjustwidth environment to exceed column width (see example table in text)
\usepackage{changepage}

% Use Unicode characters when possible
\usepackage[utf8x]{inputenc}

% textcomp package and marvosym package for additional characters
\usepackage{textcomp,marvosym}

% cite package, to clean up citations in the main text. Do not remove.
\usepackage{cite}

% Use nameref to cite supporting information files (see Supporting Information section for more info)
\usepackage{nameref,hyperref}

% line numbers
\usepackage[right]{lineno}

% ligatures disabled
\usepackage{microtype}
\DisableLigatures[f]{encoding = *, family = * }

% color can be used to apply background shading to table cells only
\usepackage[table]{xcolor}

% array package and thick rules for tables
\usepackage{array}

% create "+" rule type for thick vertical lines
\newcolumntype{+}{!{\vrule width 2pt}}

% create \thickcline for thick horizontal lines of variable length
\newlength\savedwidth
\newcommand\thickcline[1]{%
  \noalign{\global\savedwidth\arrayrulewidth\global\arrayrulewidth 2pt}%
  \cline{#1}%
  \noalign{\vskip\arrayrulewidth}%
  \noalign{\global\arrayrulewidth\savedwidth}%
}

% \thickhline command for thick horizontal lines that span the table
\newcommand\thickhline{\noalign{\global\savedwidth\arrayrulewidth\global\arrayrulewidth 2pt}%
\hline
\noalign{\global\arrayrulewidth\savedwidth}}


% Remove comment for double spacing
%\usepackage{setspace} 
%\doublespacing

% Text layout
\raggedright
\setlength{\parindent}{0.5cm}
\textwidth 5.25in 
\textheight 8.75in

% Bold the 'Figure #' in the caption and separate it from the title/caption with a period
% Captions will be left justified
\usepackage[aboveskip=1pt,labelfont=bf,labelsep=period,justification=raggedright,singlelinecheck=off]{caption}
\renewcommand{\figurename}{Fig}

% Use the PLoS provided BiBTeX style
\bibliographystyle{plos2015}

% Remove brackets from numbering in List of References
\makeatletter
\renewcommand{\@biblabel}[1]{\quad#1.}
\makeatother

% Leave date blank
\date{}

% Header and Footer with logo
\usepackage{lastpage,fancyhdr,graphicx}
\usepackage{epstopdf}
\pagestyle{myheadings}
\pagestyle{fancy}
\fancyhf{}
\setlength{\headheight}{27.023pt}
\lhead{\includegraphics[width=2.0in]{PLOS-submission.eps}}
\rfoot{\thepage/\pageref{LastPage}}
\renewcommand{\footrule}{\hrule height 2pt \vspace{2mm}}
\fancyheadoffset[L]{2.25in}
\fancyfootoffset[L]{2.25in}
\lfoot{\sf PLOS}

%% Include all macros below

\newcommand{\lorem}{{\bf LOREM}}
\newcommand{\ipsum}{{\bf IPSUM}}

%% END MACROS SECTION


\begin{document}
\vspace*{0.2in}

% Title must be 250 characters or less.
\begin{flushleft}
{\Large
\textbf\newline{Linking mosquito surveillance to dengue risk through Bayesian mechanistic modeling}
}
\newline
% Insert author names, affiliations and corresponding author email (do not include titles, positions, or degrees).
\\
Clinton Leach\textsuperscript{1*},
Colleen Webb\textsuperscript{1},
Kim Pepin\textsuperscript{2},
Alvaro Eiras\textsuperscript{3},
Mevin Hooten\textsuperscript{4},
Jennifer Hoeting\textsuperscript{4}

\bigskip
\textbf{1} Affiliation Dept/Program/Center, Institution Name, City, State, Country
\\
\textbf{2} Affiliation Dept/Program/Center, Institution Name, City, State, Country
\\
\textbf{3} Affiliation Dept/Program/Center, Institution Name, City, State, Country
\\
\bigskip

% Insert additional author notes using the symbols described below. Insert symbol callouts after author names as necessary.
% 
% Remove or comment out the author notes below if they aren't used.


% Use the asterisk to denote corresponding authorship and provide email address in note below.
* clint.leach@colostate.edu

\end{flushleft}
% Please keep the abstract below 300 words
\section*{Abstract}

Mosquito abundance alone is not a strong predictor of dengue fever risk, but provides information that, when coupled with case data, allows us to estimate mosquito demographic rates.

% Please keep the Author Summary between 150 and 200 words
% Use first person. PLOS ONE authors please skip this step. 
% Author Summary not valid for PLOS ONE submissions.   
\section*{Author summary}

Mosquitoes are dumb.

\linenumbers

% Use "Eq" instead of "Equation" for equation citations.
\section*{Introduction}

Dengue fever is a massive global public health burden, with millions of cases per year in Brazil alone \cite{Bhatt2013}.  
Since the dengue virus (DENV) is vectored by the mosquito \textit{Aedes aegypti}, dengue fever is prevented primarily through mosquito control programs \cite{Achee2015}.
To be effective, control efforts should be guided by consistent, high resolution mosquito surveillance \cite{Morrison2008}.
The "MI-Dengue" system, implemented by the company Ecovec and deployed throughout cities in Brazil, uses a city-wide grid of mosquito sticky-traps (MosquiTRAPs) for spatial prioritization of source reduction and larvicide application \cite{Eiras2009}.
Comparing cities with and without this system, \cite{Pepin2013} estimated that it prevented 27,191 cases of dengue fever from 2007 - 2011.
Though this system has been effective, the mosquito indices and thresholds used to prioritize control efforts remain somewhat ad-hoc.
As noted in \cite{Morrison2008}, entomological indices are only relevant in the context of local epidemiology, and thus control policies will be most effective when they synthesize both entomological and clinical surveillance.

Many of the attempts to link these two sources of information have found a weak relationship between mosquito surveillance disease risk (CITATIONS).
These efforts have often relied on tests of the ability of mosquito abundance measures to predict human disease risk in regression models.
For example, \cite{Pepin2015} found that models that included MosquiTRAP surveillance data failed to predict weekly dengue cases any better than models that included lagged case data alone.
However, such tests fail to account for the mechanistic, non-linear relationship between mosquito population dynamics and dengue epidemiology.
The cycle of transmission between humans and dengue is controlled not just by mosquito abundance, but also by mosquito survival relative to the virus' incubation period \cite{Achee2015}.
In fact, the probability that an exposed mosquito will survive long enough to become infectious provides a more direct measure of transmission potential than mosquito abundance alone.

Adult \textit{Aedes aegypti} surveillance data (e.g. time series of mosquito trap counts) is insufficient on its own to estimate mosquito survival rates and thus transmission potential.
However, we can borrow from the field of 'integrated population modeling' and combine multiple sources of information to estimate mosquito demographic rates (CITATIONS).
Since mosquito survival is a critical bottleneck in the transmission process, it can be informed from the observed transmission dynamics (i.e. reports of human cases), given an appropriate underlying mechanistic model.

In this paper, we propose a mechanistic model to connect mosquito abundance and survival with dengue epidemiology.
We then embed this model into a Bayesian statistical framework that allows us to estimate latent time series of mosquito demographic parameters from time series of mosquito trap counts and reported cases of dengue fever.
Lastly, we explore the utility of these estimated transmission risks for guiding targeted control efforts relative to using mosquito abundance alone.

\section*{Methods}

\subsection*{Study system and data}

Vitoria is a coastal city and the capital of the state of Espirito Santo, Brazil, with a population of 327,801 in the city proper, and an additional 1.5 million people in the greater metropolitan area.  
Since 2008, the company Ecovec has monitored mosquito abundance for the city using approximately 1327 sticky traps (MosquiTRAP, \cite{Eiras2009}) arranged in a grid across the city.
Each trap is checked weekly and the mosquitoes inside counted and identified, providing us with 243 weeks (2008 through week 34 of 2012) of counts of gravid female \emph{Aedes aegypti}.
Dengue is a mandatory notifiable disease, and thus the city's Ministry of Health Secretary maintains a database of weekly notified probable dengue cases (i.e. medical care sought for dengue-like symptoms) for this same time period.

\subsection*{Process Model}

Dengue epidemiology is complicated considerably by the presence of four simultaneously circulating serotypes (referred to as DENV-1, DENV-2, DENV-3, DENV-4).
Infection with one serotype confers life-long immunity to that serotype, along with temporary immunity to other serotypes.  
As this cross-immunity wanes, antibodies from the previous infection can actually result in antibody-dependent enhancement (ADE), wherein hosts are more susceptible to infection with the other serotypes and more likely to develop severe symptoms (i.e. dengue haemorrhagic fever or dengue shock syndrome).
The strength and duration of these different inter-serotype interactions are not well understood, though different models suggest that temporary cross-immunity alone (without ADE) is sufficient to reproduce observed multi-annual dynamics in Thailand \cite{Wearing2006,Reich2013}.
Similarly, \cite{Aguiar2013} find that capturing both primary and secondary infections and the period of cross-immunity is critical, but that explicitly including all four serotypes (at great cost to model complexity) does not perform much better than a two serotype model.

Explicitly accounting for interactions between serotypes, even only two of the four, leads to a large and complex mechanistic model.
Moreover, since dengue case reports do not include information on serotype, we do not have enough information to inform the dynamics of individual serotypes.
As such, we simplify the model of \cite{Wearing2006} to an SEIRS framework, which drops serotype-specific dynamics but preserves the period of cross-immunity and the possibility of reinfection.
In this framework, susceptible individuals ($S$) become exposed ($E$) through contact with infectious mosquitoes ($V_I$).
Following a latent period ($\frac{1}{\rho}$), exposed individuals become infectious ($I$) at which point they can infect susceptible mosquitoes ($V_S$).
Infectious individuals recover at rate $\gamma$ and subsequently remain immune for a period ($\frac{1}{\delta}$) after which they re-enter the susceptible class.

Similarly, susceptible mosquitoes ($V_S$) become exposed by biting infectious humans and pass through a temperature-dependent latent period ($\frac{1}{\rho_{vt}}$) before becoming infectious ($V_I$).
Total mosquito population size ($V_N$) is controlled by stochastic, seasonally varying growth rate ($r(t)$) and death rate ($d(t)$).
Captured mosquitoes ($V_C$) accumulate at rate $\phi_q \tau_t$, where $\phi_q$ is the per-trap capture rate, and $\tau_t$ is the number of traps deployed in week $t$.

The complete differential equation model is then given as:
\begin{align} 
\frac{dS}{dt} &= bN - bS - \lambda \frac{V_{I}}{N} S + \delta R\\
\frac{dE}{dt} &= \lambda \frac{V_{I}}{N} S - (\rho + b)E\\
\frac{dI}{dt} &= \rho E - (\gamma + b)I\\
\frac{dR}{dt} &= \gamma I - (\delta + b)R\\
\frac{dV_N}{dt} & = r V_N - \phi_q \tau_t V_N \\
\frac{dV_E}{dt} &= \lambda \frac{I}{N} V_S - (\rho_{vt} + d + \phi_q \tau_t)V_E\\
\frac{dV_I}{dt} &= \rho_{vt} V_E - (d + \phi_q \tau_t) V_I\\
\frac{dV_C}{dt} & = \phi_q \tau_t V_N\\
V_S &= V_N - V_E - V_I.
\end{align}

We model the centered and log-transformed mosquito mortality rate ($\nu$) and the per-capita growth rate ($r$) as forced harmonic oscillators with natural periods of one year:
\begin{align}
\frac{d^2\nu}{dt^2} &= -\omega^2 \nu + \epsilon_{\nu t}\\
\frac{d^2 r}{dt^2} &= -\omega^2 r + \epsilon_{rt},
\end{align}
where $\omega = 2\pi / 52$, $d(t) = d_0 \exp(\nu(t))$, and
\begin{align}
\epsilon_{\nu i} & \sim \text{Normal}(0, \sigma_{\nu})\\
\epsilon_{ri} & \sim \text{Normal}(0, \sigma_r),
\end{align}
for $i = 1 \dots 243$.
These stochastically-forced harmonic oscillators provide a flexible framework for generating smooth seasonal oscillations in the latent mosquito processes (Ramsay citation?).

\subsection*{Data model}

To connect the differential equation model the the observed case reports, we add an extra state, $C$, that collects the cumulative number of transitions from the exposed to infectious class (assuming that case reporting coincides with the onset of symptoms).
We then model the number of new cases reported in week $t$ ($y_t$) as:
\begin{equation}
y_t  \sim \text{NegBin}(\phi_y (C_t - C_{t-1}), \eta_y),
\end{equation}
where $\phi_y$ is the reporting probability, $C_t - C_{t-1}$ is the number of new infectious individuals in week $t$, and $\eta_y$ controls the overdispersion relative to the Poisson.

We similarly model the number of mosquitoes trapped in week $t$ ($q_t$) as:
\begin{equation}
q_t \sim \text{NegBin}(V_{Ct} - V_{Ct-1}, \eta_q),
\end{equation}
where $V_{Ct} - V_{Ct-1}$ is the number of new mosquitoes captured in week $t$, and $\eta_q$ controls overdispersion relative to the Poisson.

\subsection*{Parameterization and priors}

Several of the parameters in this model are assumed to be fixed and known (\ref{parameters}).
The population size and average life span (which we use to parameterize the birth/death rate) for Vitoria are taken from the 2010 census.
To maintain identifiability, the transmission rate ($\lambda$) and case reporting probability ($\phi_y$) are also fixed at literature values.
Lastly, the extrinsic incubation period in mosquitoes is modeled as a function of weekly mean temperature and forced with weather station data obtained from WeatherUnderground.

The remaining parameters include the epidemiological parameters ($\rho$, $\gamma$, $\delta$), the initial conditions of the model ($S_0$, $E_0$, $I_0$, $R_0$, $V_{N0}, \nu_0, r_0$), the variances of the latent mosquito processes ($\sigma^2_r$, $\sigma^2_{\nu}$), and the remaining measurement parameters ($\phi_q$, $\eta_y$, $\eta_q$).  
Where possible, we place informative priors on these parameters based on existing laboratory and field studies (see Table \ref{parameters} for means and Supplemental Material for detailed explanations).

\begin{table}[!ht]
\label{parameters}
\begin{adjustwidth}{-2.25in}{0in} 
\begin{center}
\caption{Model parameters and their values.  Parameters above the rule are fixed, while parameters below the rule are random, with the value giving the prior mean.}
\begin{tabular}{llll}
Parameter & Description & Value & Citation\\
\hline
$N$ & Population size & 327801 & Vitoria Census\\
$1/d$ & Human life-span & 76 years & Vitoria Census\\
$\lambda$ & Transmission rate & 4.87 week$^{-1}$ & \cite{Scott2000}\\
$\phi_y$ & Reporting probability & 0.083 & \cite{Silva2016}\\
$1/\rho_{vt}$ & Extrinsic incubation period & $7\exp \left( 0.2 T_t - 8 \right)$ & \cite{Chan2012}\\
$\exp(\mu)$ & Mean mosquito death rate & 1.47 week$^{-1}$ & \cite{Brady2013} \\
$V_{E0}$ & Initial exposed mosquitoes &  0 & \\
$V_{I0}$ & Initial infectious mosquitoes & 0 & \\
\hline
$1/\rho$ & Latent period in host & 0.87 weeks  & \cite{Chan2012}\\
$\gamma$ & Rate of loss of infectiousness & 3.5 week$^{-1}$ & \cite{Nguyet2013}\\
$1/\delta$ & Period of cross-immunity & 97 weeks &  \cite{Reich2013}\\
$S_0$ & Proportion initially susceptible & 0.4 & \cite{Cardoso2011a} \\
$E_0$ & Proportion initially exposed & $8\times 10 ^ {-5}$ & \\
$I_0$ & Proportion initially infectious & $8\times 10 ^ {-5}$ & \\
$\phi_q$ & Per-trap mosquito capture rate & $2 \times 10^{-6}$ week$^{-1}$ & 
\end{tabular}
\end{center}
\end{adjustwidth}
\end{table}

\subsection*{Implementation}
 
Samples from the posterior distribution were generated using Hamiltonian Monte Carlo implemented in the rstan package \cite{Carpenter2016, Rstan2017} for R (\cite{R2016}). 
We ran 3 chains with different starting values for 10,000 iterations each, discarding the first 5,000 as burn-in.
Within stan, the solution to the differential equation model was approximated with an Euler scheme with a time step of 1 day.  

\subsection*{Control simulations}

Given samples from the posterior as obtained above, we can simulate the efficacy of different control strategies.
In particular, we investigate the optimal timing of a single pulse of control within the year (e.g., if the city only has enough resource to fog for adult mosquitoes once a year, in what week is that control going to prevent the most cases?).
To obtain the posterior distribution of the number of cases prevented by implementing control in week $i$ each year, for each posterior sample $k$, we simulate an increase in the latent mosquito death rate during each week $i$, e.g.,
$\hat{d}^{(k)}_{yi} = (1 + \Delta) d^{(k)}_{yi}$, where $y$ indexes the year, and $i$ indexes the week of the year.
Then we sum the total number of cases reported in this scenario and compare to the data to get an estimate of the number of cases prevented.
To keep our simulations conservative, and to avoid pushing the model into the unrealistic range of dengue eradication, we set $\Delta = 0.05$.
We also only simulate the effect of control in the middle three years, as the effectiveness of control in the first year is strongly affected by initial conditions, and we lack complete data for the last year.

\section*{Results}

The model is able to capture the observed dynamics of both case reports and mosquito trap counts (Figure \ref{timeseries}).
The estimated posterior median case reports explain 91\% of the variation in the observed time series, while the posterior median mosquito trap counts explain 45\% of the variation in the observed time series. 
In addition, posterior predictive checks show that the model reproduces the the total number of cases reported and mosquitoes captured as well as the autocorrelation structure of both time series (with the exception of slightly underestimating the autocorrelation for short lags, Supplemental Fig XX - XX).
Posterior means for the three epidemiological parameters ($\rho, \gamma, \delta$) are close to the literature values used to set their prior means (Supplemental Fig XX).

The estimated latent mosquito demographic rates are smooth and strongly seasonal, with fairly narrow credible intervals (Figure \ref{latent}).
Mosquito death rates fluctuate within the range estimated by \cite{Brady2013} and generally increase with temperature, as expected from laboratory studies \cite{Brady2013}(Figure \ref{temp}).
The underlying process variances are small ($\sigma_nu = 0.01$, $\sigma_r = 0.0005$), and there is little residual structure in the process noise series, though there is a weak periodic signal in the $\epsilon_{\nu}$ (Supplemental Fig XX).

Modifying these estimated demographic rates to simulate targeted control efforts, we find that control applied in the third week of the year is the most effective at preventing cases of dengue fever (though weeks 2 and 4 are also very close, Figure \ref{control}A).
In fact, increasing the mosquito death rate by just 5\% in the third week of 2009, 2010, and 2011 would prevent 13,988 (reported) cases of dengue fever (with 80\% credible interval of 11,130 to 15,811).
Most of these gains come by way of substantially reducing the size of the large 2011 outbreak, while also further damping the smaller 2010 and 2012 outbreaks (Figure \ref{control}B).
Targeting larvae instead of adults (simulated by reducing the birth rate ($b_t = r_t + d_t$) by 5\%) produces nearly identical results.

The third week of the year is generally very early in the dengue season, well before cases peak (Figure \ref{timing}).
In the two years with large outbreaks (2009 and 2011), the third week of the year is near the inflection point where the disease burden begins to grow rapidly.  
The third week is also generally before the estimated seasonal peak in mosquito abundance, which occurs between weeks 6 and 9 (Figure \ref{timing}).
Annual peaks in estimated mosquito death rate are more variable, but do tend to fall between weeks 2 and 7.
Though mosquito death rates tend to be high early in the year, the transmission risk (i.e. the probability of an exposed mosquito surviving long enough to be infectious, $\frac{\rho_v}{\rho_v + d}$) does not pick up until later in the year.

\begin{figure}[!h]
\includegraphics[angle = 270, scale = 1]{figures/fig1.eps}
\caption{{\bf Vitoria data and model fits.}
Weekly observations from 2008 to week 34 of 2012 (points), with corresponding median posterior prediction (line) and 80\% credible interval (gray band). A: reported dengue cases. B: total number of trapped mosquitoes.
}
\label{timeseries}
\end{figure}

\begin{figure}[!h]
\includegraphics[angle = 270, scale = 1]{figures/fig2.eps}
\caption{{\bf Estimated latent mosquito growth and death rates.}
Median posterior estimate (line) and 80\% credible interval (gray band).  A: weekly mosquito population growth rate ($r_t$). B: weekly mosquito death rate ($d_t$).
}
\label{latent}
\end{figure}

\begin{figure}[!h]
\includegraphics[angle = 270, scale = 1]{figures/fig3.eps}
\caption{{\bf Estimated mosquito death rate as a function of temperature.}
Weekly median posterior death rate ($d_t$) plotted against weekly mean temperature (Celsius) in Vitoria, Brazil.
}
\label{temp}
\end{figure}

\begin{figure}[!h]
\includegraphics[angle = 270, scale = 1]{figures/fig4.eps}
\caption{{\bf Cases prevented by mosquito control.}
A: Number of reported cases of dengue fever prevented by increasing the mosquito death rate by 5\% during a given week each year in 2009, 2010, and 2011. Median posterior prediction (line) and 80\% credible interval. B: Weekly predicted case reports resulting from implementing control in week 3.  Median posterior case reports from model without (dashed line) and with control (solid line) and 80\% credible interval of controlled dynamics (gray band).
}
\label{control}
\end{figure}

\begin{figure}[!h]
\includegraphics[angle = 270, scale = 1]{figures/fig5.eps}
\caption{{\bf Timing of optimal control relative to disease and mosquito dynamics.}
The vertical black line indicates week 3, during which mosquito control produces the largest decline in predicted dengue case reports. Each colored line represents a different year: salmon? = 2008, green-yellow = 2009, green = 2010, blue = 2011, purple = 2012. A: posterior median estimate of dengue case reports per week. B: posterior median estimate of trapped mosquitoes per week. C: posterior median estimate of weekly mosquito death rate. D: posterior median estimate of weekly transmission risk ($\frac{\rho_v}{\rho_v + d}$).
}
\label{timing}
\end{figure}

\section*{Discussion}

Targeted mosquito control can be an effective tool in reducing the disease burden within a city.
When faced with limited resources with which to deploy that control, these results suggest that focusing efforts early in the year, specifically in the second to fourth weeks of the year, will lead to the largest reductions in disease.
Because adult control and larval control produce the same results, the effect of control is likely driven by reducing the standing population of adult mosquitoes, rather than by limiting the transmission potential of exposed or infectious mosquitoes.
Targeting mosquitoes early in the year as dengue just begins to spread can disrupt the transmission cycle and prevent further amplification and feedback.

The importance of reducing the population of adult mosquitoes agrees with the results of \cite{Burattini2008} whose model of dengue dynamics in Singapore suggests that killing adult mosquitoes is the most effective way to control an epidemic.
However, they find that larval control is most effective at preventing resurgence of dengue epidemics, and thus that a mixed strategy that includes both larval and adult control is likely to be most effective in the long term.
Fully exploring such combined strategies requires a more complete model of mosquito life history and the lags induced by the mosquito's aquatic larval stage.
In addition, the lack of a larval phase in our model might somewhat exaggerate the effectiveness of adult control, as decreasing the adult population leads to an immediate reduction in reproduction, rather than a lagged reduction.
Temperature-dependent development \cite{Focks1993}.
Models of development can be very complex \cite{Marori2009}.
\cite{Pinho2010} find that adult control alone cannot reduce $R_0$ below 1 and that aquatic control is necessary.

As implemented, mosquito mortality is the only source of uncertainty/stochasticity in the transmission process.
Related to the idea of empirical forcing functions of \cite{Hooker2015} and dynamic smoothing from Ramsay.
The fact that death rate increases with temperature helps to validate this approach.
Whether or not the death rate is actually interpretable as a mortality rate, it nonetheless captures the critical bottleneck in the transmission process.
\cite{Maciel-de-Freitas2008}

Establishing that there is likely spatial heterogeneity.
Processes that might drive spatial heterogeneity (human movement, socioeconomic factors)
Structured human movement within the city is likely to be a major driver of heterogeneous human-mosquito mixing (\cite{Adams2009, Cosner2009a, Stoddard2009, Dalziel2013}.
In addition variability in socioeconomic factors within the city also modulate the extent to which mosquitoes in different parts of the city contribute to disease spread \cite{Mondini2008, Honorio2009, Hu2012, DeMattosAlmeida2007}.
What are the implications of this heterogeneity and how might spatial heterogeneity modify our conclusions? Introducing heterogeneous mixing increases $R_0$ relative to homogeneous mixing, potentially leading us to overestimate the effectiveness of control (\cite{Dye1986, Hasibeder1988}.
This heterogeneity suggests that disease risk might be driven by high levels of contact with a relatively small proportion of the mosquito population \cite{Canyon1999, Yoon2012}.
What can we gain by accounting for heterogeneity?
Though the presence of this heterogeneity suggests that our model may overestimate the efficacy of mosquito control, it also points the way towards more efficient methods.
Most generally, control efforts should target mosquitoes that are most directly implicated in transmission.
One way to do this is through the use of contact-tracing \cite{Vazquez-Prokopec2017}.
How would we incorporate that heterogeneity?  Embedding the mechanistic model explicitly into a metapopulation framework leads to a very large, unwieldy model that likely exacerbates existing difficulties inherent in fitting ODE models to data.

Statistical novelty and connections to literature
Integrated population modeling \cite{Schaub2010}
Posterior sampling for ode models \cite{Girolami2008, Calderhead2011}

\subsection*{Conclusions}

Mosquito surveillance is a valuable tool in managing dengue fever, but is most powerful when integrated into a larger modeling framework.


\section*{Supporting information}

% Include only the SI item label in the paragraph heading. Use the \nameref{label} command to cite SI items in the text.
\paragraph*{S1 Text.}
\label{S1_Diag}
{\bf Full description of model and priors.}

\section*{Acknowledgements}

Thanks!

\nolinenumbers

% Either type in your references using
% \begin{thebibliography}{}
% \bibitem{}
% Text
% \end{thebibliography}
%
% or
%
% Compile your BiBTeX database using our plos2015.bst
% style file and paste the contents of your .bbl file
% here. See http://journals.plos.org/plosone/s/latex for 
% step-by-step instructions.
% 

\bibliographystyle{plos2015}
\bibliography{dengue}


\end{document}

