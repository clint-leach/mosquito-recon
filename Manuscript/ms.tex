% Template for PLoS
% Version 3.4 January 2017
%
% % % % % % % % % % % % % % % % % % % % % %
%
% -- IMPORTANT NOTE
%
% This template contains comments intended 
% to minimize problems and delays during our production 
% process. Please follow the template instructions
% whenever possible.
%
% % % % % % % % % % % % % % % % % % % % % % % 
%
% Once your paper is accepted for publication, 
% PLEASE REMOVE ALL TRACKED CHANGES in this file 
% and leave only the final text of your manuscript. 
% PLOS recommends the use of latexdiff to track changes during review, as this will help to maintain a clean tex file.
% Visit https://www.ctan.org/pkg/latexdiff?lang=en for info or contact us at latex@plos.org.
%
%
% There are no restrictions on package use within the LaTeX files except that 
% no packages listed in the template may be deleted.
%
% Please do not include colors or graphics in the text.
%
% The manuscript LaTeX source should be contained within a single file (do not use \input, \externaldocument, or similar commands).
%
% % % % % % % % % % % % % % % % % % % % % % %
%
% -- FIGURES AND TABLES
%
% Please include tables/figure captions directly after the paragraph where they are first cited in the text.
%
% DO NOT INCLUDE GRAPHICS IN YOUR MANUSCRIPT
% - Figures should be uploaded separately from your manuscript file. 
% - Figures generated using LaTeX should be extracted and removed from the PDF before submission. 
% - Figures containing multiple panels/subfigures must be combined into one image file before submission.
% For figure citations, please use "Fig" instead of "Figure".
% See http://journals.plos.org/plosone/s/figures for PLOS figure guidelines.
%
% Tables should be cell-based and may not contain:
% - spacing/line breaks within cells to alter layout or alignment
% - do not nest tabular environments (no tabular environments within tabular environments)
% - no graphics or colored text (cell background color/shading OK)
% See http://journals.plos.org/plosone/s/tables for table guidelines.
%
% For tables that exceed the width of the text column, use the adjustwidth environment as illustrated in the example table in text below.
%
% % % % % % % % % % % % % % % % % % % % % % % %
%
% -- EQUATIONS, MATH SYMBOLS, SUBSCRIPTS, AND SUPERSCRIPTS
%
% IMPORTANT
% Below are a few tips to help format your equations and other special characters according to our specifications. For more tips to help reduce the possibility of formatting errors during conversion, please see our LaTeX guidelines at http://journals.plos.org/plosone/s/latex
%
% For inline equations, please be sure to include all portions of an equation in the math environment.  For example, x$^2$ is incorrect; this should be formatted as $x^2$ (or $\mathrm{x}^2$ if the romanized font is desired).
%
% Do not include text that is not math in the math environment. For example, CO2 should be written as CO\textsubscript{2} instead of CO$_2$.
%
% Please add line breaks to long display equations when possible in order to fit size of the column. 
%
% For inline equations, please do not include punctuation (commas, etc) within the math environment unless this is part of the equation.
%
% When adding superscript or subscripts outside of brackets/braces, please group using {}.  For example, change "[U(D,E,\gamma)]^2" to "{[U(D,E,\gamma)]}^2". 
%
% Do not use \cal for caligraphic font.  Instead, use \mathcal{}
%
% % % % % % % % % % % % % % % % % % % % % % % % 
%
% Please contact latex@plos.org with any questions.
%
% % % % % % % % % % % % % % % % % % % % % % % %

\documentclass[10pt,letterpaper]{article}
\usepackage[top=0.85in,left=2.75in,footskip=0.75in]{geometry}

% amsmath and amssymb packages, useful for mathematical formulas and symbols
\usepackage{amsmath,amssymb}

% Use adjustwidth environment to exceed column width (see example table in text)
\usepackage{changepage}

% Use Unicode characters when possible
\usepackage[utf8x]{inputenc}

% textcomp package and marvosym package for additional characters
\usepackage{textcomp,marvosym}

% cite package, to clean up citations in the main text. Do not remove.
\usepackage{cite}

% Use nameref to cite supporting information files (see Supporting Information section for more info)
\usepackage{nameref,hyperref}

% line numbers
\usepackage[right]{lineno}

% ligatures disabled
\usepackage{microtype}
\DisableLigatures[f]{encoding = *, family = * }

% color can be used to apply background shading to table cells only
\usepackage[table]{xcolor}

% array package and thick rules for tables
\usepackage{array}

% create "+" rule type for thick vertical lines
\newcolumntype{+}{!{\vrule width 2pt}}

% create \thickcline for thick horizontal lines of variable length
\newlength\savedwidth
\newcommand\thickcline[1]{%
  \noalign{\global\savedwidth\arrayrulewidth\global\arrayrulewidth 2pt}%
  \cline{#1}%
  \noalign{\vskip\arrayrulewidth}%
  \noalign{\global\arrayrulewidth\savedwidth}%
}

% \thickhline command for thick horizontal lines that span the table
\newcommand\thickhline{\noalign{\global\savedwidth\arrayrulewidth\global\arrayrulewidth 2pt}%
\hline
\noalign{\global\arrayrulewidth\savedwidth}}


% Remove comment for double spacing
%\usepackage{setspace} 
%\doublespacing

% Text layout
\raggedright
\setlength{\parindent}{0.5cm}
\textwidth 5.25in 
\textheight 8.75in

% Bold the 'Figure #' in the caption and separate it from the title/caption with a period
% Captions will be left justified
\usepackage[aboveskip=1pt,labelfont=bf,labelsep=period,justification=raggedright,singlelinecheck=off]{caption}
\renewcommand{\figurename}{Fig}

% Use the PLoS provided BiBTeX style
\bibliographystyle{plos2015}

% Remove brackets from numbering in List of References
\makeatletter
\renewcommand{\@biblabel}[1]{\quad#1.}
\makeatother

% Leave date blank
\date{}

% Header and Footer with logo
\usepackage{lastpage,fancyhdr,graphicx}
\usepackage{epstopdf}
\pagestyle{myheadings}
\pagestyle{fancy}
\fancyhf{}
\setlength{\headheight}{27.023pt}
\lhead{\includegraphics[width=2.0in]{PLOS-submission.eps}}
\rfoot{\thepage/\pageref{LastPage}}
\renewcommand{\footrule}{\hrule height 2pt \vspace{2mm}}
\fancyheadoffset[L]{2.25in}
\fancyfootoffset[L]{2.25in}
\lfoot{\sf PLOS}

%% Include all macros below

\newcommand{\lorem}{{\bf LOREM}}
\newcommand{\ipsum}{{\bf IPSUM}}

%% END MACROS SECTION


\begin{document}
\vspace*{0.2in}

% Title must be 250 characters or less.
\begin{flushleft}
{\Large
\textbf\newline{Linking mosquito surveillance to dengue fever through Bayesian mechanistic modeling}
}
\newline
% Insert author names, affiliations and corresponding author email (do not include titles, positions, or degrees).
\\
Clinton Leach\textsuperscript{1,2*},
Colleen Webb\textsuperscript{1},
Kim Pepin\textsuperscript{3},
Alvaro Eiras\textsuperscript{4},
Mevin Hooten\textsuperscript{2},
Jennifer Hoeting\textsuperscript{2}

\bigskip
\textbf{1} Graduate Degree Program in Ecology, Colorado State University, Fort Collins, CO, USA
\\
\textbf{2} Department of Statistics, Colorado State University, Fort Collins, CO, USA
\\
\textbf{3} National Wildlife Research Center, United States Department of Agriculture, Wildlife Services, Fort Collins, CO, USA
\\
\textbf{4} Departamento de Parasitologia, Universidade Federal de Minas Gerais, Belo Horizonte, MG, Brazil
\bigskip

% Insert additional author notes using the symbols described below. Insert symbol callouts after author names as necessary.
% 
% Remove or comment out the author notes below if they aren't used.


% Use the asterisk to denote corresponding authorship and provide email address in note below.
* clint.leach@colostate.edu

\end{flushleft}
% Please keep the abstract below 300 words
\section*{Abstract}

Our ability to effectively prevent the transmission of the dengue virus through targeted control of its vector, \emph{Aedes aegypti}, depends critically on our understanding of the link between mosquito abundance and human disease risk.
The mosquito and clinical surveillance data necessary to elucidate this link are widely collected, but have yet to be coupled with a modeling framework that accounts for the complex non-linear mechanisms involved in transmission, in particular the critical bottleneck imposed by mosquito mortality.
We developed a differential equation model of dengue transmission and embedded it in a Bayesian hierarchical framework that allowed us to estimate latent time series of mosquito demographic rates from mosquito trap counts and dengue case reports from the city of Vitoria, Brazil.
We used the fitted model to explore the optimal timing of targeted adult control during the year.
We found that control was most effective when deployed in the third week of the year, as mosquito abundance neared its peak and dengue transmission began to ramp up.
Intervening at this time disrupts transmission and prevents the amplification and feedback that leads to large outbreaks of disease, an insight not possible without the underlying mechanistic model.
Though the ability of mosquito surveillance to predict human disease is often limited, this highlights the utility of this surveillance when integrated into a larger modeling framework.
Grounding this modeling famework in the actual mechanisms of transmission will help to establish more effective and efficient dengue control policies that allow us to better target mosquitoes that are most responsible for continuing disease spread. 

% Please keep the Author Summary between 150 and 200 words
% Use first person. PLOS ONE authors please skip this step. 
% Author Summary not valid for PLOS ONE submissions.   
%\section*{Author summary}

\linenumbers

% Use "Eq" instead of "Equation" for equation citations.
\section*{Introduction}

Dengue fever is a massive global public health burden, with millions of cases per year \cite{Bhatt2013}.  
Because the dengue virus (DENV) is vectored by the mosquito \textit{Aedes aegypti}, dengue fever is prevented primarily through mosquito control programs \cite{Achee2015}.
Though there have been documented successes, there is limited evidence for the long-term sustainability and effectiveness of these control programs \cite{Morrison2008}.
Because of this, there is a growing recognition that effective control needs to be guided by high quality vector surveillance, together with quantitative tools that synthesize vector surveillance with clinical surveillance, account for local epidemiology, and connect easily to local decision making \cite{Morrison2008, Scott2010b}.
Moreover, mosquito control needs to be guided by an understanding of the link between mosquito abundance and disease risk so that the mosquitoes most responsible for transmission can be targeted \cite{Scott2010a, Scott2010b}.

Many of the attempts to establish this link have found a weak relationship between mosquito surveillance and incidence of dengue fever \cite{Bowman2014, Pepin2015, Cromwell2017}.
However, these attempts often do not account for the complex, non-linear interactions that mediate the relationship between mosquito abundance and human disease.
In particular, host immunity is a key intrinsic driver of infectious disease dynamics, and conditions favorable for transmission can only lead to an outbreak of disease when there is a sufficiently large population of susceptible hosts \cite{Koelle2004, Koelle2005}.
As such, the ability of mosquitoes to contribute to DENV transmission depends critically on the level of prior immunity in the human population \cite{Scott2010a}.
Further, the cycle of transmission between humans and mosquitoes is influenced not just by mosquito abundance, but also by mosquito survival relative to the virus' incubation period in mosquitoes \cite{Smith2012}.
In fact, whether or not an exposed mosquito will survive long enough to become infectious represents a critical bottleneck in the transmission process and leads to nonlinear dependence of transmission on mosquito survival \cite{Smith2012}.

The importance of intrinsic nonlinearities, potentially alongside seasonality and stochastic forcing \cite{Ellner1998, Koelle2004, Grenfell2002}, in governing human disease risk highlight the need to integrate mechanistic modeling into the quantitative tools used to understand the effects of control interventions.
Such mechanistic models can often perform better than complex autoregressive statistical models in describing and forecasting population dynamics \cite{Reilly2005}.
Moreover, in the absence of case-control studies, mechanistic models can provide scenario-based tools that can be used to predict the effect of management actions \cite{Buckland2007}.

Differential equation models provide a natural way to encode the relevant biological mechanisms, though it is important to also account for sources of uncertainty in those models \cite{Hotelling1927, Wikle2010}.
In particular, the values of parameters (e.g. the length of time for which a host is infectious) are often uncertain, which can lead to large uncertainty about the effects of management actions \cite{Elderd2006}.
In addition, the structure of the processes themselves can be uncertain \cite{Ellner1998}, and need to be informed by available, often noisy, data.
Bayesian hierarchical modeling provides a coherent framework to account for and integrate this uncertainty across the three levels of the model (data, process, and parameters \cite{Berliner1996, Cressie2009}).

In this paper, we aim to integrate these elements -- a detailed mechanistic model of dengue transmission with a full Bayesian accounting of uncertainty -- in order to better understand the interplay of forces governing dengue dynamics and their interaction with potential vector control interventions.
We apply and test this framework on clinical and entomological surveillance data from the city of Vitoria, Brazil.
These data allow us to estimate a latent time series of mosquito mortality rates that modulate the transmission process and link between mosquito abundance to human disease.  
We then use the fitted model to explore how perturbations to the mosquito population propagate and interact with the complex epidemiology and nonlinearities involved in dengue transmission.

\section*{Methods}

\subsection*{Study system and data}

Vitoria is a coastal city and the capital of the state of Espirito Santo, Brazil, with a population of 327,801 \cite{vitpop}.
Since 2008, the company Ecovec has monitored mosquito abundance for the city using approximately 1327 sticky traps (MosquiTRAP, \cite{Eiras2009}) arranged in a grid across the city.
Each trap is checked weekly and the mosquitoes inside counted and identified, with the results sent to a central database that city managers then use to map mosquito infestations and target control.
These data provide us with 243 weeks (2008 through week 34 of 2012) of total city-wide counts of trapped gravid female \emph{Aedes aegypti}.
It is important to note that this time series reflects both natural fluctuations in mosquito density and fluctuations driven by the city's existing mosquito control program.
In addition, dengue fever is a mandatory notifiable disease, and thus the city's Ministry of Health Secretary maintains a database of weekly notified probable dengue cases (i.e., medical care sought for dengue-like symptoms) for the same time period.

\subsection*{Process Model}

Dengue epidemiology is complicated considerably by the presence of four simultaneously circulating serotypes.
Infection with one serotype confers life-long immunity to that serotype, along with temporary immunity to other serotypes \cite{Wearing2006}.  
As this cross-immunity wanes, antibodies from the previous infection can result in antibody-dependent enhancement (ADE), wherein human hosts are more susceptible to infection with the other serotypes and more likely to develop severe symptoms (i.e., dengue haemorrhagic fever or dengue shock syndrome)\cite{Wearing2006}.
The strength and duration of these different inter-serotype interactions are not well understood, though different models suggest that temporary cross-immunity alone (without ADE) is sufficient to reproduce observed multi-annual dynamics in Thailand \cite{Wearing2006,Reich2013}.
Similarly, Aguiar \emph{et al.} \cite{Aguiar2013} found that capturing primary and secondary infections and the period of cross-immunity is critical, but that explicitly including all four serotypes (at great cost to model complexity) does not perform better than a two serotype model.

Explicitly accounting for interactions between serotypes, even only two of the four, leads to a large and complex mechanistic model.
Moreover, because dengue case reports do not include information on serotype, we do not have enough information to inform the dynamics of individual serotypes.
As such, we develop an SEIRS compartment model, similar to \cite{Newton1992, Burattini2008, Pinho2010} which drops serotype-specific dynamics but preserves the period of cross-immunity and the possibility of reinfection.
In this framework, susceptible humans ($S$) become exposed ($E$) through contact with infectious mosquitoes ($V_I$).
Following a latent period ($\frac{1}{\rho}$), exposed humans become infectious ($I$) at which point they can infect susceptible mosquitoes ($V_S$).
Infectious humans recover at rate $\gamma$ and subsequently remain immune for a period ($\frac{1}{\delta}$) after which they re-enter the susceptible class.

Similarly, susceptible mosquitoes ($V_S$) become exposed by biting infectious humans and pass through a temperature-dependent incubation period ($\frac{1}{\rho_{vt}}$) before becoming infectious ($V_I$).
Total mosquito population size ($V_N$) is controlled by stochastic, seasonally varying growth rate ($r(t)$) and death rate ($d(t)$).
Captured mosquitoes ($V_C$) accumulate at rate $\phi_q \tau_t$, where $\phi_q$ is the per-trap capture rate, and $\tau_t$ is the number of traps deployed in week $t$.

The differential equations governing the human population are then given as:
\begin{align} 
\frac{dS}{dt} &= bN - bS - \lambda \frac{V_{I}}{N} S + \delta R\\
\frac{dE}{dt} &= \lambda \frac{V_{I}}{N} S - (\rho + b)E\\
\frac{dI}{dt} &= \rho E - (\gamma + b)I\\
\frac{dR}{dt} &= \gamma I - (\delta + b)R
\end{align}
while the equations governing the mosquito (vector) population are:
\begin{align}
\frac{dV_N}{dt} & = r(t) V_N - \phi_q \tau(t) V_N \\
\frac{dV_{E1}}{dt} &= \lambda \frac{I}{N} V_S - (\rho_{v}(t) + d(t) + \phi_q \tau(t))V_{E1}\\
\frac{dV_{E2}}{dt} &= \rho_{v}(t) V_{E1} - (\rho_{v}(t) + d(t) + \phi_q \tau(t))V_{E2}\\
\frac{dV_{E3}}{dt} &= \rho_{v}(t) V_{E2}  - (\rho_{v}(t) + d(t) + \phi_q \tau(t))V_{E3}\\
\frac{dV_{E4}}{dt} &= \rho_{v}(t) V_{E3}  - (\rho_{v}(t) + d(t) + \phi_q \tau(t))V_{E4}\\
\frac{dV_I}{dt} &= \rho_{v}(t) V_{E4} - (d(t) + \phi_q \tau(t)) V_I\\
\frac{dV_C}{dt} & = \phi_q \tau(t) V_N\\
V_S &= V_N - V_E - V_I.
\end{align}

We model the centered and log-transformed mosquito mortality rate ($\nu$) and the per-capita mosquito growth rate ($r$) as forced harmonic oscillators with natural periods of one year:
\begin{align}
\frac{d^2\nu}{dt^2} &= -\omega^2 \nu + \epsilon_{\nu t}\\
\frac{d^2 r}{dt^2} &= -\omega^2 r + \epsilon_{rt},
\end{align}
where the angular frequency of the oscillator, $\omega = 2\pi / 52$, the mosquito death rate $d(t) = d_0 \exp(\nu(t))$, and
\begin{align}
\epsilon_{\nu i} & \sim \text{Normal}(0, \sigma^2_{\nu})\\
\epsilon_{ri} & \sim \text{Normal}(0, \sigma^2_r),
\end{align}
for $i = 1 \dots 243$.
These stochastically-forced harmonic oscillators provide a flexible framework for generating smooth seasonal oscillations in the latent mosquito processes \cite{Ramsay2017}.

\subsection*{Data model}

To connect the differential equation model to the observed case reports, we add an extra state, $C$, that collects the cumulative number of transitions from the exposed to infectious class (assuming that case reporting coincides with the onset of symptoms).
We then model the number of new cases reported in week $t$ ($y_t$) as:
\begin{equation}
y_t  \sim \text{NegBin}(\phi_y (C(t) - C(t-1)), \eta_y),
\end{equation}
where $\phi_y$ is the reporting probability, $C(t) - C(t-1)$ is the number of new infectious humans in week $t$, and $\eta_y$ controls the overdispersion relative to the Poisson distribution.

We similarly model the number of mosquitoes trapped in week $t$ ($q_t$) as:
\begin{equation}
q_t \sim \text{NegBin}(V_{C}(t) - V_{C}(t-1), \eta_q),
\end{equation}
where $V_{C}(t) - V_{C}(t-1)$ is the number of new mosquitoes captured in week $t$, and $\eta_q$ controls overdispersion relative to the Poisson distribution.

\subsection*{Parameterization and priors}

Several of the parameters in this model are assumed to be fixed and known (Table 1).
The human population size and average life span (which we use to parameterize the birth/death rate) for Vitoria are taken from the 2010 census.
To maintain identifiability, the transmission rate ($\lambda$) and case reporting probability ($\phi_y$) are also fixed at literature values.
Lastly, the extrinsic incubation period in mosquitoes is modeled as a function of weekly mean temperature and forced with weather station data obtained from WeatherUnderground.

The remaining parameters include the epidemiological parameters ($\rho$, $\gamma$, $\delta$, $d_0$), the initial conditions of the model ($S_0$, $E_0$, $I_0$, $R_0$, $V_{N0}, \nu_0, r_0$), the variances of the latent mosquito processes ($\sigma^2_r$, $\sigma^2_{\nu}$), and the remaining measurement parameters ($\phi_q$, $\eta_y$, $\eta_q$).  
Where possible, we specified informative priors on these parameters based on existing laboratory and field studies (see Table 1 for means and Supplemental Material for detailed explanations).

\begin{table}[!ht]
\label{parameters}
\begin{adjustwidth}{-2.25in}{0in} 
\begin{center}
\caption{Model parameters and their values.  Parameters above the rule are fixed, while parameters below the rule are random, with the value giving the prior mean.}
\begin{tabular}{llll}
Parameter & Description & Value & Citation\\
\hline
$N$ & Human population size & 327801 & \cite{vitpop} \\
$1/d$ & Human life-span & 76 years & \cite{vitlong} \\
$\lambda$ & Transmission rate & 4.87 week$^{-1}$ & \cite{Scott2000}\\
$\phi_y$ & Reporting probability & 0.083 & \cite{Silva2016}\\
$1/\rho_{vt}$ & Extrinsic incubation period & $7\exp \left( 0.2 T_t - 8 \right)$ & \cite{Chan2012}\\
$V_{E0}$ & Initial exposed mosquitoes &  0 & \\
$V_{I0}$ & Initial infectious mosquitoes & 0 & \\
\hline
$d_0$ & Baseline mosquito mortality rate & 1.47 week$^{-1}$ & \cite{Brady2013} \\
$1/\rho$ & Latent period in host & 0.87 weeks  & \cite{Chan2012}\\
$\gamma$ & Rate of loss of infectiousness & 3.5 week$^{-1}$ & \cite{Nguyet2013}\\
$1/\delta$ & Period of cross-immunity & 97 weeks &  \cite{Reich2013}\\
$\sigma_r$ & Standard deviation of mosquito growth rate forcing & $0^*$ & \\
$\sigma_{\nu}$ & Standard deviation of mosquito mortality rate forcing & $0^*$ & \\
$S_0$ & Proportion initially susceptible & 0.4 & \cite{Cardoso2011a} \\
$E_0$ & Proportion initially exposed & $8\times 10 ^ {-5}$ & \\
$I_0$ & Proportion initially infectious & $8\times 10 ^ {-5}$ & \\
$r_0$ & Initial mosquito population growth rate & 0 & \\
$V_{N0}$ & Initial mosquito population size & $2N$ & \\
$\nu_0$ & Initial unconstrained mosquito mortality rate & 0 & \\
$\phi_q$ & Per-trap mosquito capture rate & $2 \times 10^{-6}$ week$^{-1}$ & \\
$\eta_y$ & Overdispersion of case reports & $0^*$ & \\
$\eta_q$ & Overdispersion of mosquito trap counts & $0^*$ & \\
\end{tabular}
\end{center}
\end{adjustwidth}
$^*$ indicates prior mode, rather than mean.
\end{table}

\subsection*{Implementation}

Combining the data, process, and parameter models \cite{Berliner1996}, we can summarize the full hierarchical model as:
\begin{align}
y_t | \cdot & \sim \text{NegBin}(\phi_y (C(t) - C(t-1)), \eta_y),
\\
q_t | \cdot &\sim \text{NegBin}(V_{C}(t) - V_{C}(t-1), \eta_q)\\
(\mathbf{C}, \mathbf{V_C}) & = \mathcal{M}(\boldsymbol{\epsilon_r}, \boldsymbol{\epsilon_{\nu}},\boldsymbol{\theta})\\
\epsilon_{rt} & \sim \text{Normal}(0, \sigma^2_r)\\
\epsilon_{\nu t} & \sim \text{Normal}(0, \sigma^2_{\nu})\\
\boldsymbol{\theta} & \sim [\boldsymbol{\theta}]
\end{align}
where $\boldsymbol{\theta}$ is a vector of all the model parameters and initial conditions, and $\mathcal{M}(\boldsymbol{\epsilon_r}, \boldsymbol{\epsilon_{\nu}},\boldsymbol{\theta})$ represents the (numeric) solution to the differential equation as a function of $\boldsymbol{\theta}$ and the weekly stochastic forcing terms ($\boldsymbol{\epsilon_{r}}, \boldsymbol{\epsilon_{\nu}}$).
Sampling from the posterior distribution of the parameters and forcing terms in this model is difficult due to multimodality, variable parameter sensitivities (e.g., small changes in one parameter may lead to large changes in output, while similar changes in another parameter may have little effect), and potentially strong posterior correlations induced by the nonlinearity of the the differential equation model \cite{Reilly2005, Girolami2008, Calderhead2011}.
In addition, the computational burden of numerically solving the differential equation model at each iteration means that sampling efficiency is crucial.
These problems can be mitigated somewhat by introducing stochasticity through the $\epsilon_t$ in the mosquito demographic processes.
This allows the data to pull the latent states closer even when the parameters are far from optimal values \cite{Leander2014}.
This reduces multimodality and allows gradient-based methods like Hamiltonain Monte Carlo (HMC) to more easily and efficiently traverse the posterior.
Samples from the posterior distribution were thus generated using HMC implemented in the rstan package \cite{Carpenter2016, Rstan2017} for R \cite{R2016}. 
We ran 3 chains with different starting values for 10,000 iterations each, discarding the first 5,000 as burn-in.
Within stan, the solution to the differential equation model was approximated with an Euler scheme with a time step of 1 day.  

\subsection*{Control simulations}

Given samples from the posterior distribution as obtained above, we simulated the effects of a single pulse of control applied in each week of the first three years of the time series.
Since the city already implements responsive, targeted control with the aim of reducing local mosquito density during an existing outbreak, we focused our simulations on exploring the longer-term feedbacks induced by mosquito control and the ability of an intervention to reduce the disease burden in both the current and future years.
To obtain the posterior distribution of the proportion of cases prevented in calendar year $j$ by implementing control in week $i$, for each posterior sample $k$, we simulated the dynamics resulting from reducing the mosquito population by $5\%$ at the beginning of week $i$ (affecting susceptible, exposed, and infectious mosquitoes equally, and leaving the estimated mosquito birth and mortality rates unchanged) .
We then compared the number of cases produced during year $j$ in the control scenario to the same number in the uncontrolled scenario.
A $5\%$ reduction in mosquito abundance was chosen to keep our simulations conservative relative to field estimates of the mortality induced by spraying \cite{Esu2010}, and to avoid pushing the model into the unrealistic range of dengue eradication.

\section*{Results}

The model captured the observed dynamics of both case reports and mosquito trap counts (Figure \ref{timeseries}).
The estimated posterior median case reports explained 90\% of the variation in the observed time series, while the posterior median mosquito trap counts explained 46\% of the variation in the observed time series. 
In addition, posterior predictive checks showed that the model reproduced the total number of cases reported and mosquitoes captured as well as the autocorrelation structure of both time series (with the exception of slightly underestimating the autocorrelation for short lags, Supplemental Figures 6 -8).
The posterior distributions of the rate of infectious decay ($\gamma$) and the period of cross-immunity ($1/\delta$) did not differ substantially from their priors, suggesting that these data contain little additional information about these parameters (Supplemental Figure 3).
The estimated latent period in a human host ($1/\rho$, the expected time it takes for an exposed human to become infectious to biting mosquitoes) is influenced more strongly by the data, with a posterior mean of 1.53 weeks compared to a prior mean of 0.87 weeks.

The estimated weekly mosquito mortality rate varied seasonally, and was positively correlated with temperature \ref{mortality}.
The estimated oscillations in mosquito mortality were more complex than purely sinusoidal, with variations in shape from year to year.
The posterior mean of the baseline mortality rate, around which the time series oscillated, was 0.7/week, roughly half the prior mean of 1.47/week.
The posterior means of the $\epsilon_{\nu t}$ forcing the mosquito mortality process exhibited a higher-frequency periodic oscillation (Supplementary Figure 1), though the full posterior distribution of each $\epsilon_{\nu t}$ overlapped zero.
In addition, the standard deviation of the mortality forcing terms was small relative to the weak prior ($E(\sigma_{\nu}|\mathbf{y}) = 0.01$, Supplemental Figure 2).

\begin{figure}[!h]
\includegraphics[angle = 270, scale = 1]{figures/fig1.eps}
\caption{{\bf Vitoria data and model fits.}
Weekly observations from 2008 to week 34 of 2012 (points), with corresponding median posterior prediction (line) and 80\% credible interval (gray band). A: reported dengue cases. B: total number of trapped mosquitoes.
}
\label{timeseries}
\end{figure}

\begin{figure}[!h]
\includegraphics[angle = 270, scale = 1]{figures/fig2.eps}
\caption{{\bf Estimated latent mosquito mortality rate.}
A: Time series of mortality rate, showing median posterior estimate (line) and 80\% credible interval (gray band). B: weekly mosquito mortality rate as a function of weekly mean temperature..
}
\label{mortality}
\end{figure}

\begin{figure}[!h]
%\includegraphics[angle = 270, scale = 1]{figures/fig3.eps}
\caption{{\bf The effect of cross-immunity on effect and timing of control.}
Each point represents a sample from the posterior and the results of the corresponding control simulation, in which we evaluated the effects of a $5\%$ reduction in mosquito abundance at each of the 52 weeks from week 16 of the previous year to week 15 of the target year. A, B, C: the maximum proportion of cases prevented target years 2009, 2010, and 2011, respectively.  D, E, F: the epidemiological week in which control achieved the maximum reduction in cases in 2009, 2010, and 2011, respectively.
}
\label{immunity}
\end{figure}

\begin{figure}[!h]
%\includegraphics[angle = 270, scale = 1]{figures/fig4.eps}
\caption{{\bf Timing of optimal control relative to disease and mosquito dynamics.}
The vertical gray lines indicate the median week at which control was most effective in preventing cases in the following calendar year. A: posterior median estimate of new dengue cases per week. B: posterior median estimate of number of mosquitoes per person each week. C: posterior median estimate of the proportion of susceptible individuals.
}
\label{timing}
\end{figure}

\section*{Discussion}


\subsection*{Summary}

Targeted mosquito control can be an effective tool in reducing the disease burden within a city.
When faced with limited resources with which to deploy that control, our results suggest that focusing efforts early in the year, specifically in the second to fourth weeks, will lead to the largest reductions in disease.
The lag between optimal control and both mosquito population dynamics and human disease highlights the importance of early intervention and the need to account for the nonlinear processes governing transmission.
Targeting the mosquito population early in the year as dengue just begins to spread can disrupt the transmission cycle and prevent further amplification and feedback later in the year.

\subsection*{Model validation and interpretation of $d_v$}

In addition, there is relatively little uncertainty on our estimates of weekly mosquito mortality (Fig. \ref{mortality}) and the variance of the forcing terms is small, indicating that mosquito mortality is tightly constrained by the case reports.

Given the central role that mosquito lifespan plays in driving dengue dynamics, we modeled fluctuations in the mosquito mortality rate using a flexible, forced harmonic oscillator.
This encodes our prior expectation that the mosquito mortality rate should be driven by seasonal environmental forcing (i.e., it should oscillate with a period of one year), while allowing the flexibility to diverge from the simple oscillator and learn from the case reports data.
The fact that the $\epsilon_{\nu}$ forcing terms are the only source of variability in the transmission process, however, means that the estimated mosquito mortality time series could be soaking up other sources of stochasticity or model misspecification.
Hooker and Ellner \cite{Hooker2015} provide a framework for diagnosing such model misspecification in differential equation models using forcing functions similar to our implementation of the $\epsilon_{\nu}$.
In that framework, Hooker and Ellner \cite{Hooker2015} estimate nonparametric forcing functions that modify a fitted differential equation model to provide a good fit to the data.
These forcing functions serve as residuals on the time derivatives, and can be more readily interpreted as indicators of lack-of-fit than residuals on the state variables.
We do not employ the same explicit goodness-of-fit testing framework as \cite{Hooker2015}, but we can inspect our estimated $\epsilon_{\nu}$ forcing terms in the same spirit.

In particular, the periodic structure in the posterior means of the $\epsilon_{\nu t}$ suggests that these residuals are accounting for more than just noise, and there may be some unmodeled process influencing fluctuations in mosquito mortality and/or transmission.
Following Hooker and Ellner, we can explore whether this process results from misspecification of the rates of change of the existing state variables, or from missing state variables altogether. 
The lack of any apparent relationship between any of the state variables and the forcing function, combined with the dependence of $\epsilon_{\nu t}$ on previous values (as apparent through the periodic structure), suggest that unmodeled state variables may be the more likely driver of model misspecification.
These unmodeled components could include additional mosquito population dynamic processes (e.g. aquatic stage dynamics, environmental drivers, or control interventions), or epidiemiological processes (e.g. multiple circulationg serotypes of the dengue virus).
Additional work developing and testing new models would be needed to distinguish among these.

Despite this suggestion of potential model misspecification, our estimated mosquito mortalities nonetheless fell within the reasonable range from the literature \cite{Maciel-de-Freitas2008, Brady2013}.
Moreover, the fact that our estimated mortality rate increased with temperature also broadly agrees with the empirical literature on mosquito survival \cite{Yang2009, Brady2013}.
This suggests that whatever structure remains unexplained in the forcing terms, the pattern of case reports is still very well described by realistic fluctuations in the mosquito mortality rate.
Given the substantial increase in model complexity that, say, resolving serotype dynamics would entail, we suggest that allowing a relatively small amount of phenomenological forcing to persist is a fair price to pay for parsimony.

\subsection*{Processes driving dynamics and implications for control}

Infectious diseases \cite{Ellner1998,Koelle2004} and ecological systems \cite{Bjornstad2001}, are often driven by the combined efforts of intrinsic non-linearities and feedbacks, seasonality, and stochasticity.
In this paper, we have highlighted the importance of these processes in driving the dynamics of dengue fever and shaping the relationship between mosquito abundance and human disease risk.

\subsubsection*{Adult control}

The importance of mosquito longevity in driving disease dynamics highlights the potential effectiveness of control efforts that target adult mosquitoes and disrupt transmission by preventing mosquitoes from living long enough to progress through the extrinsic incubation period to the infectious state.
In fact, this forms the basis for much of the theory of adult control \cite{Burattini2008, Morrison2008, Smith2012}.
Our results suggest that such control efforts (i.e., those that remove adult mosquitoes from the population) are most effective when applied late in the outbreak (week 24), after the number of cases has peaked.
If applied at the peak of the epidemic, the this control is likely less effective, as there are still a large number of infectious human hosts available \cite{Newton1992, Burattini2008}.
Control applied late in the outbreak pushes the system in the direction it is already heading.
Thus control that interacts with the existing seasonality is more effective.
In measles, this is also generally when the effect of stochasticity is most limited and the dynamics are pulled most strongly towards the deterministic attractor.
This is also generally when immunity is highest.

\subsection*{Caveats and extensions}

The case data to which we fit the model represents reports of "dengue-like illness", without laboratory confirmation, and as such could include cases of other diseases with similar symptoms (e.g. chikungunya or Zika).
However, neither chikungunya nor Zika had emerged as substantial public health threats in Brazil by the end of our time series in late 2012 \cite{PAHO2014, PAHO2015}.
In addition, given the high underreporting rate expected for dengue fever \cite{Silva2016}, and the uncertainty incorporated into the measurement model, we expect misreported cases to have a small effect on our analyses.

In addition, it is important to note that that the data to which we fit our model implicitly reflect the control efforts already enacted by the city.
These control efforts are guided by the MI-Dengue system, which uses the trap-level mosquito surveillance data to target areas of high mosquito infestation for control (including source reduction, larvacide, and adulticide)\cite{Eiras2009}.
These local interventions are effective at reducing disease burden in the cities in which they are implemented \cite{Pepin2013}, but long-term epidemiological processes (e.g. fluctuations in human immunity) still seem to be driving disease spread.
Our control simulations account for these processes and offer a suggestion for how an additional city-wide pulse of control might be deployed to supplement existing efforts to reduce mosquito density throughout the city.

That said, we do not anticipate that a single, simultaneous, city-wide pulse of control could be feasibly implemented on top of the existing control efforts.
However, contrary to our assumption of homogeneous mixing between mosquitoes and humans, all mosquitoes may not contribute equally to dengue spread.
In fact, a relatively small proportion of the mosquito population may be responsible for a large proportion of disease spread \cite{Yoon2012}
In particular, structured human movement within the city is likely to induce heterogeneous human-mosquito mixing \cite{Adams2009, Cosner2009a, Stoddard2009}.
In addition, spatial variability in socioeconomic factors within the city may also modulate the extent to which mosquitoes in different parts of the city contribute to disease spread \cite{Mondini2008, Honorio2009, Hu2012, DeMattosAlmeida2007}.
The mechanistic model presented here may be able to suggest when an intervention is likely to be effective, but in order to make the best use of limited resources, spatial prioritization may still be necessary.
This should include not just prioritization of areas with high mosquito abundance, but also areas with disproprtionate risk of disease spread, either due to hetereogeneities in the mixing process or in the distribution of human immunity.

\subsection*{Conclusions}

Mosquito surveillance is a valuable tool in managing dengue fever, but is most powerful when integrated into a larger modeling framework.
We have developed a simple yet realistic mechanistic model of dengue fever spread that allows us to combine mosquito and clinical surveillance data to estimate the latent mosquito demographic rates that connect mosquito abundance and human disease.
The mechanistic framework allows us to capture important lags and feedbacks in the transmission process and to identify critical intervention points that would not be apparent otherwise.
The fully hierarchical Bayesian framework in which we embedded the mechanistic model allows for a thorough accounting of uncertainty that is carried through to the evaluation of different control strategies.
This combination of model features helps to meet the need for more effective, biologically grounded, and data-driven dengue control policies and offers a building block on which these tools can be further developed in the future.

\section*{Supporting information}

% Include only the SI item label in the paragraph heading. Use the \nameref{label} command to cite SI items in the text.
\paragraph*{S1 Text.}
\label{S1}
{Full description of model and priors, along with all supplemental plots.}

\section*{Acknowledgements}

Thanks!

\nolinenumbers

% Either type in your references using
% \begin{thebibliography}{}
% \bibitem{}
% Text
% \end{thebibliography}
%
% or
%
% Compile your BiBTeX database using our plos2015.bst
% style file and paste the contents of your .bbl file
% here. See http://journals.plos.org/plosone/s/latex for 
% step-by-step instructions.
% 

\bibliographystyle{plos2015}
\bibliography{dengue}


\end{document}

