% Template for PLoS
% Version 3.4 January 2017
%
% % % % % % % % % % % % % % % % % % % % % %
%
% -- IMPORTANT NOTE
%
% This template contains comments intended 
% to minimize problems and delays during our production 
% process. Please follow the template instructions
% whenever possible.
%
% % % % % % % % % % % % % % % % % % % % % % % 
%
% Once your paper is accepted for publication, 
% PLEASE REMOVE ALL TRACKED CHANGES in this file 
% and leave only the final text of your manuscript. 
% PLOS recommends the use of latexdiff to track changes during review, as this will help to maintain a clean tex file.
% Visit https://www.ctan.org/pkg/latexdiff?lang=en for info or contact us at latex@plos.org.
%
%
% There are no restrictions on package use within the LaTeX files except that 
% no packages listed in the template may be deleted.
%
% Please do not include colors or graphics in the text.
%
% The manuscript LaTeX source should be contained within a single file (do not use \input, \externaldocument, or similar commands).
%
% % % % % % % % % % % % % % % % % % % % % % %
%
% -- FIGURES AND TABLES
%
% Please include tables/figure captions directly after the paragraph where they are first cited in the text.
%
% DO NOT INCLUDE GRAPHICS IN YOUR MANUSCRIPT
% - Figures should be uploaded separately from your manuscript file. 
% - Figures generated using LaTeX should be extracted and removed from the PDF before submission. 
% - Figures containing multiple panels/subfigures must be combined into one image file before submission.
% For figure citations, please use "Fig" instead of "Figure".
% See http://journals.plos.org/plosone/s/figures for PLOS figure guidelines.
%
% Tables should be cell-based and may not contain:
% - spacing/line breaks within cells to alter layout or alignment
% - do not nest tabular environments (no tabular environments within tabular environments)
% - no graphics or colored text (cell background color/shading OK)
% See http://journals.plos.org/plosone/s/tables for table guidelines.
%
% For tables that exceed the width of the text column, use the adjustwidth environment as illustrated in the example table in text below.
%
% % % % % % % % % % % % % % % % % % % % % % % %
%
% -- EQUATIONS, MATH SYMBOLS, SUBSCRIPTS, AND SUPERSCRIPTS
%
% IMPORTANT
% Below are a few tips to help format your equations and other special characters according to our specifications. For more tips to help reduce the possibility of formatting errors during conversion, please see our LaTeX guidelines at http://journals.plos.org/plosone/s/latex
%
% For inline equations, please be sure to include all portions of an equation in the math environment.  For example, x$^2$ is incorrect; this should be formatted as $x^2$ (or $\mathrm{x}^2$ if the romanized font is desired).
%
% Do not include text that is not math in the math environment. For example, CO2 should be written as CO\textsubscript{2} instead of CO$_2$.
%
% Please add line breaks to long display equations when possible in order to fit size of the column. 
%
% For inline equations, please do not include punctuation (commas, etc) within the math environment unless this is part of the equation.
%
% When adding superscript or subscripts outside of brackets/braces, please group using {}.  For example, change "[U(D,E,\gamma)]^2" to "{[U(D,E,\gamma)]}^2". 
%
% Do not use \cal for caligraphic font.  Instead, use \mathcal{}
%
% % % % % % % % % % % % % % % % % % % % % % % % 
%
% Please contact latex@plos.org with any questions.
%
% % % % % % % % % % % % % % % % % % % % % % % %

\documentclass[10pt,letterpaper]{article}
\usepackage[top=0.85in,left=2.75in,footskip=0.75in]{geometry}

% amsmath and amssymb packages, useful for mathematical formulas and symbols
\usepackage{amsmath,amssymb}

% Use adjustwidth environment to exceed column width (see example table in text)
\usepackage{changepage}

% Use Unicode characters when possible
\usepackage[utf8x]{inputenc}

% textcomp package and marvosym package for additional characters
\usepackage{textcomp,marvosym}

% cite package, to clean up citations in the main text. Do not remove.
\usepackage{cite}

% Use nameref to cite supporting information files (see Supporting Information section for more info)
\usepackage{nameref,hyperref}

% line numbers
\usepackage[right]{lineno}

% ligatures disabled
\usepackage{microtype}
\DisableLigatures[f]{encoding = *, family = * }

% color can be used to apply background shading to table cells only
\usepackage[table]{xcolor}

% array package and thick rules for tables
\usepackage{array}

% create "+" rule type for thick vertical lines
\newcolumntype{+}{!{\vrule width 2pt}}

% create \thickcline for thick horizontal lines of variable length
\newlength\savedwidth
\newcommand\thickcline[1]{%
  \noalign{\global\savedwidth\arrayrulewidth\global\arrayrulewidth 2pt}%
  \cline{#1}%
  \noalign{\vskip\arrayrulewidth}%
  \noalign{\global\arrayrulewidth\savedwidth}%
}

% \thickhline command for thick horizontal lines that span the table
\newcommand\thickhline{\noalign{\global\savedwidth\arrayrulewidth\global\arrayrulewidth 2pt}%
\hline
\noalign{\global\arrayrulewidth\savedwidth}}


% Remove comment for double spacing
%\usepackage{setspace} 
%\doublespacing

% Text layout
\raggedright
\setlength{\parindent}{0.5cm}
\textwidth 5.25in 
\textheight 8.75in

% Bold the 'Figure #' in the caption and separate it from the title/caption with a period
% Captions will be left justified
\usepackage[aboveskip=1pt,labelfont=bf,labelsep=period,justification=raggedright,singlelinecheck=off]{caption}
\renewcommand{\figurename}{Fig}

% Use the PLoS provided BiBTeX style
\bibliographystyle{plos2015}

% Remove brackets from numbering in List of References
\makeatletter
\renewcommand{\@biblabel}[1]{\quad#1.}
\makeatother

% Leave date blank
\date{}

% Header and Footer with logo
\usepackage{lastpage,fancyhdr,graphicx}
\usepackage{epstopdf}
\pagestyle{myheadings}
\pagestyle{fancy}
\fancyhf{}
\setlength{\headheight}{27.023pt}
\lhead{\includegraphics[width=2.0in]{PLOS-submission.eps}}
\rfoot{\thepage/\pageref{LastPage}}
\renewcommand{\footrule}{\hrule height 2pt \vspace{2mm}}
\fancyheadoffset[L]{2.25in}
\fancyfootoffset[L]{2.25in}
\lfoot{\sf PLOS}

%% Include all macros below

\newcommand{\lorem}{{\bf LOREM}}
\newcommand{\ipsum}{{\bf IPSUM}}

%% END MACROS SECTION


\begin{document}
\vspace*{0.2in}

% Title must be 250 characters or less.
\begin{flushleft}
{\Large
\textbf\newline{Linking mosquito surveillance to dengue risk through Bayesian mechanistic modeling}
}
\newline
% Insert author names, affiliations and corresponding author email (do not include titles, positions, or degrees).
\\
Clinton Leach\textsuperscript{1*},
Colleen Webb\textsuperscript{1},
Kim Pepin\textsuperscript{2},
Alvaro Eiras\textsuperscript{3},
Mevin Hooten\textsuperscript{4},
Jennifer Hoeting\textsuperscript{4}

\bigskip
\textbf{1} Affiliation Dept/Program/Center, Institution Name, City, State, Country
\\
\textbf{2} Affiliation Dept/Program/Center, Institution Name, City, State, Country
\\
\textbf{3} Affiliation Dept/Program/Center, Institution Name, City, State, Country
\\
\bigskip

% Insert additional author notes using the symbols described below. Insert symbol callouts after author names as necessary.
% 
% Remove or comment out the author notes below if they aren't used.


% Use the asterisk to denote corresponding authorship and provide email address in note below.
* clint.leach@colostate.edu

\end{flushleft}
% Please keep the abstract below 300 words
\section*{Abstract}

We know mosquitoes spread dengue, but mosquito abundance, as estimated from trap-based surveillance data, is not causally linked to human disease risk.

% Please keep the Author Summary between 150 and 200 words
% Use first person. PLOS ONE authors please skip this step. 
% Author Summary not valid for PLOS ONE submissions.   
\section*{Author summary}

We know mosquitoes spread dengue, but mosquito abundance, as estimated from trap-based surveillance data, is not causally linked to human disease risk.


\linenumbers

% Use "Eq" instead of "Equation" for equation citations.
\section*{Introduction}

Dengue fever is a massive global public health burden, with millions of cases per year in Brazil alone (CITATION).  
Since the dengue virus (DENV) is vectored by the mosquito \textit{Aedes aegypti}, dengue fever is prevented primarily through mosquito control programs \cite{Achee2015}.
Even if a vaccine is developed, mosquito control is likely to remain an important component of dengue management \cite{Achee2015}.
Studies have shown that targeted mosquito control efforts can be effective in reducing disease burden.
For example contact tracing and residential spraying reduced disease risk in Cairns, Australia \cite{Vazquez-Prokopec2017}.
Though this approach was successful, it requires detailed data following individual cases, potentially limiting its utility for large-scale deployment.

Other efforts have sought to guide the deployment of control efforts through the use of fine-scale mosquito surveillance.
For instance, the "MI-Dengue" system deployed throughout several cities in Brazil uses a grid of mosquito sticky-traps (MosquiTRAPs) for spatial prioritization of source reduction and larvicide application \cite{Eiras2009}.
Comparing cities with and without this system, \cite{Pepin2013} estimated that it prevented X cases and saved X dollars in economic losses from 20XX - 20XX.
Though this system has been effective, the mosquito indices and thresholds used to prioritize control efforts remain somewhat ad-hoc.
Using such surveillance data to efficiently target limited control resources requires a detailed understanding of the relationship between the components of mosquito abundances measured by the trap and disease risk.

However, even though mosquitoes are necessary for dengue transmission, establishing a clear link between surveillance-based indices of mosquito abundance and disease risk has proven difficult thus far (CITATIONS).
Many of the attempts to establish this link have relied on tests of the ability of mosquito abundance measures to predict human disease risk in regression models.
For example, \cite{Pepin2015} found that models that included MosquiTRAP surveillance data failed to predict weekly dengue cases any better than autoregressive models that include lagged case data alone.
Evaluating the influence of mosquito abundance in this way is analogous to a test of Granger causality \cite{Granger1969}, which states that some variable $X$ (e.g. mosquito abundance) \emph{causes} $Y$ (e.g. human dengue cases) if $X$ contains unique information about $Y$ (i.e. if a model with $X$ and $Y$ provides better predictions of $Y$ than a model with $Y$ alone).
By this definition, the results of \cite{Pepin2015} suggest that mosquito abundance does not cause dengue cases (in the Granger sense).

However, as noted by Sugihara \emph{et al.} \cite{Sugihara2012a}, Granger causality assumes that $X$ and $Y$ are separable, which is not the case if $X$ and $Y$ are linked in a non-linear dynamical system.
In this case, a model with $X$ and $Y$ will not outperform the model with only $Y$, as the information contained in $X$ is already embedded in $Y$, a consequence of Taken's theorem (CITATION). 
Since dengue transmission is governed by such a non-linear dynamical system, information about the driving component of mosquito abundance should be already embedded in the time series of human case reports.
As a result, predictive utility is not an appropriate way to test the causal influence of mosquito abundance on dengue risk.
Instead, Sugihara \emph{et al.} argue that if $X$ is a driver $Y$, then you should be able to reconstruct $X$ from $Y$.

In this paper, we apply this perspective to the problem of linking mosquito abundance to disease risk.
To test whether or not mosquito abundance, as measured through surveillance efforts, drives dengue dynamics, we ask whether we can reconstruct a time series of mosquito surveillance data from a time series of case reports.
We approach this as an inverse problem: we first posit a mechanistic model of dengue transmission and then develop a Bayesian statistical framework to estimate the mosquito abundance time series that is consistent with that model and the time series of case reports.
We then evaluate the causal link between mosquito abundance and case reports by using posterior predictive checks to compare the estimated and observed mosquito time series.  


\section*{Methods}

\subsection*{Study system and data}

Vitoria is a coastal city and the capital of the state of Espirito Santo, Brazil, with a population of 327,801 in the city proper, and an additional 1.5 million people in the greater metropolitan area.  
The city has an area of approximately 93 sq. km., with most of that area taken up by a large island surrounded by the river Rio Santa Maria and bay Baia de Vitoria.  

Since 2008, the city has monitored mosquito abundance using 1327 sticky traps (MosquiTRAP, \cite{Eiras2009}) arranged in a grid across the city.
Each trap is checked weekly, providing us with 243 weeks (2008 through week 34 of 2012) of counts of gravid female \emph{Aedes aegypti}.
Dengue is a mandatory notifiable disease, and thus the city's Ministry of Health Secretary maintains a database of weekly notified probable dengue cases (i.e. medical care sought for dengue-like symptoms).

Governador Valladares is also a city for which we have data.

Sete Lagoas is also a city for which we have data.


\subsection*{Process Model}

Dengue epidemiology is complicated considerably by the presence of four simultaneously circulating serotypes (referred to as DENV-1, DENV-2, DENV-3, DENV-4).
Infection with one serotype confers life-long immunity to that serotype, along with temporary immunity to other serotypes.  
As this cross-immunity wanes, antibodies from the previous infection can actually result in antibody-dependent enhancement (ADE), wherein hosts are more susceptible to infection with the other serotypes and more likely to develop severe symptoms (i.e. dengue haemorrhagic fever or dengue shock syndrome).
The strength and duration of these different inter-serotype interactions are not well understood, though different models suggest that temporary cross-immunity alone (without ADE) is sufficient to reproduce observed multi-annual dynamics in Thailand \cite{Wearing2006,Reich2013}.
Similarly, \cite{Aguiar2013} find that capturing both primary and secondary infections and the period of cross-immunity is critical, but that explicitly including all four serotypes (at great cost to model complexity) does not perform much better than a two serotype model, and indeed that the number of hospital admissions caused by a third or fourth infection is very low \cite{Aguiar2011a}.

Explicitly accounting for interactions between serotypes, even only two of the four, leads to a large and complex mechanistic model.
Moreover, since dengue case reports do not include information on serotype, we do not have enough information to inform the dynamics of individual serotypes.
As such, we simplify the model of \cite{Wearing2006} to an SEIRS framework, which drops serotype-specific dynamics but preserves the period of cross-immunity and the possibility of reinfection.
In this framework, susceptible individuals ($S$) become exposed ($E$) through contact with infectious mosquitoes ($V_I$).
Following a latent period ($\frac{1}{\rho}$), exposed individuals become infectious ($I$) at which point they can infect susceptible mosquitoes ($V_S$).
Infectious individuals recover at rate $\gamma$ and remain immune for a period ($\frac{1}{\delta}$) after which they re-enter the susceptible class.

Similarly, susceptible mosquitoes ($V_S$) become exposed by biting infectious humans and pass through a temperature-dependent latent period ($\frac{1}{\rho_{vt}}$) before becoming infectious ($V_I$).

The complete differential equation model is then given as:
\begin{align} 
\frac{dS}{dt} &= bN - bS - \alpha \frac{V_{I}}{N} S + \delta R\\
\frac{dE}{dt} &= \alpha \frac{V_{I}}{N} S - (\rho + b)E\\
\frac{dI}{dt} &= \rho E - (\gamma + b)I\\
\frac{dR}{dt} &= \gamma I - (\delta + b)R\\
V_{St} & = V_{Nt} - V_E(t) - V_I(t)\\
\frac{V_E}{dt} &= \alpha \frac{I}{N} V_S - (\rho_{vt} + d_v)V_E\\
\frac{V_I}{dt} &= \rho_{vt} V_E - d_v V_I
\end{align}

The unobserved total mosquito abundance $V_N$, and thus the size of the available susceptible pool, is modeled as an AR(1) process on the log-scale:
\begin{equation}
\log(V_{Nt}) \sim \text{Normal}(\beta \log(V_{Nt-1}), \sigma).
\end{equation}

\subsection*{Data model}

To connect the differential equation model the the observed case reports, we add an extra state, $X$, that collects the cumulative number of transitions from the exposed to infectious class (assuming that case reporting coincides with the onset of symptoms).
We then model the number of new cases reported in week $t$ ($y_t$) as:
\begin{equation}
y_t  \sim \text{NegBin}(\phi_y (X_t - X_{t-1}), \eta_y),
\end{equation}
where $\phi_y$ is the reporting probability, $X_t - X_{t-1}$ is the number of new infectious individuals in week $t$, and $\eta_y$ controls the overdispersion relative to the Poisson.

We also model the number of mosquitoes trapped in week $t$ ($q_t$) as:
\begin{equation}
q_t \sim \text{NegBin}(\phi_q \tau_t V_{Nt}, \eta_q),
\end{equation}
where $\phi_q$ is the per-trap capture probability, $\tau_t$ is the number of traps inspected in week $t$, and $\eta_q$ controls overdispersion relative to the Poisson.

\subsection*{Parameterization and priors}

Several of the parameters in this model are assumed to fixed and known. 
The population size and average life span (which we use to parameterize the birth/death rate) for Vitoria are taken from the 2010 census.
To maintain identifiability, the transmission rate and case reporting probability are also fixed at literature values, while the standard deviation of the mosquito process is fixed at a level that allows reasonably flexible behavior.
Lastly, the extrinsic incubation period in mosquitoes is modeled as a function of weekly mean temperature and forced with data pulled from WeatherUnderground.

The remaining parameters include the epidemiological parameters that control the flows between the compartments of the differential equation model ($d_v$, $\rho$, $\gamma$, $\delta$), the initial conditions of the model ($S_0$, $E_0$, $I_0$, $R_0$, $V_{N0}$), the autoregressive parameter in the mosquito process model ($\beta$), and the remaining measurement parameters ($\phi_q$, $\eta_y$, $\eta_q$).  
Where possible, we place informative priors on these parameters based on existing laboratory and field studies (see Table \ref{fixedparms} for means and Supplemental Material for detailed explanations).

For any given values of these parameters, the mosquito process model should be flexible enough to produce a good fit to the case reports data.
As a result, we expect that these parameters will be informed primarily by the mosquito surveillance data (i.e. these parameters add degrees of freedom that let the mosquito process model fit the surveillance data better while maintaining a good fit to the case data).

\begin{table}[!ht]
\label{fixedparms}
\begin{adjustwidth}{-2.25in}{0in} 
\begin{center}
\caption{Model parameters and their values.  Parameters above the rule are fixed, while parameters below the rule are random, with the value giving the prior mean.}
\begin{tabular}{llll}
Parameter & Description & Value & Citation\\
\hline
$N$ & Population size & 327801 & Vitoria Census\\
$1/d$ & Human life-span & 76 years & Vitoria Census\\
$\alpha$ & Transmission rate & 4.87 week$^{-1}$ & \cite{Scott2000}\\
$\phi_y$ & Reporting probability & 0.083 & \cite{Silva2016}\\
$1/\rho_{vt}$ & Extrinsic incubation period & $\exp \left(\exp(1.9 - 0.04 T_t) + \frac{1}{14}\right)$ & \cite{Chan2012}\\
$\sigma$  & Standard dev. of AR(1) process & 0.2 & \\
$V_{E0}$ & Initial exposed mosquitoes &  0 & \\
$V_{I0}$ & Initial infectious mosquitoes & 0 & \\
\hline
$d_v$ & Vector death rate & 1.47 week$^{-1}$ & \cite{Brady2013} \\
$1/\rho$ & Latent period in host & 0.87 weeks  & \cite{Chan2012}\\
$\gamma$ & Rate of loss of infectiousness & 3.5 week$^{-1}$ & \cite{Nguyet2013}\\
$1/\delta$ & Period of cross-immunity & 97 weeks &  \cite{Reich2013}\\
$S_0$ & Proportion initially susceptible & 0.4 & \cite{Cardoso2011a} \\
$E_0$ & Proportion initially exposed & $8\times 10 ^ {-5}$ & \\
$I_0$ & Proportion initially infectious & $8\times 10 ^ {-5}$ & \\
$\phi_q$ & Per-trap mosquito capture probability & $2 \times 10^{-6}$ & 
\end{tabular}
\end{center}
\end{adjustwidth}
\end{table}


\subsection*{Implementation}
 
Samples from the posterior distribution were generated using Hamiltonian Monte Carlo (CITATIONS) implemented in the rstan package for R. 
We ran 3 chains with different starting values for 10,000 iterations each, discarding the first 5,000 as burn-in.
Within stan, the solution to the differential equation model was approximated with an Euler scheme with a time step of 1 day.  

\subsection*{Posterior predictive checks}

We evaluate the goodness-of-fit of the estimated mosquito abundance time series through the use of posterior predictive checks \cite{Gelman1996,Gelman2003}.
Specifically, for each iteration, $i$, of MCMC, we sample replicate mosquito surveillance data:
\begin{equation}
q^{rep(i)}_t \sim \text{NegBin}(\phi^{(i)}_q \tau_t V^{(i)}_{Nt}, \eta^{(i)}_q)
\end{equation}
and compute
\begin{equation}
p_t = P(q^{rep}_t > q_t) = \frac{1}{M}\sum I(q^{rep(i)}_t > q_t).
\end{equation}
If the reconstructed mosquito trajectories are a good fit for the observed mosquito trap counts, then we would expect the time series $p_t$ to be random in time (CITATION?).
Any systematic trends in $p_t$ indicate lack-of-fit.

In addition to looking at weekly comparisons, we also develop posterior checks to compare the temporal variability and correlation structures of the observed and estimated time series.
To test whether the modeled mosquito time series capture the seasonal features of the observed mosquito data, we compute the posterior auto-correlation function of the replicate data:
\begin{equation}
R^{rep(i)}(l) = \text{corr}(q^{rep(i)}_t, q^{rep(i)}_{t-l}),
\end{equation}
for lags ranging from 1 to 55 weeks.
We also compute posterior distributions of $\min(q^{rep}_t)$ , $\max(q^{rep}_t)$, and $\sigma(q^{rep}_t - q^{rep}_{t-1})$ as measures of the smoothness of the modeled process.

\section*{Results}

Questions:
\begin{itemize}
	\item To what extent do the case reports inform our estimates of mosquito abundance? How do we separate the information coming from the case reports from the information coming from the mosquito data itself?
	\item Do our estimates of epidemiological parameters differ from their prior (literature) values?
	\item Do the case reports reproduce the observed mosquito surveillance time series?
	\item If no, how does it fail?
\end{itemize}

The model is able to effectively capture the observed weekly case reports for Vitoria (Fig. \ref{timeseries}A), as expected given the degrees of freedom in the mosquito process model.
In addition, most of the observed mosquito trap counts fall within the 80\% posterior credible interval (Fig. \ref{timeseries}B).
We expect the posterior distribution of mosquito abundance time series to be informed primarily by the observed case data.
However, the mapping from the case reports to mosquito abundance is influenced by the parameters of the mechanistic model and deviations of the posterior distribution of these parameters from their prior reflect the influence of the mosquito trap count data.
Of these parameters, the posterior estimate of mosquito mortality rate (blah) was lower than its prior mean (1.47 week$^{-1}$), while the posterior estimate of the intrinsic incubation period (blah) was larger than its prior mean (0.87 weeks).
That these two parameters differ from their literature values reflects the fact that they are critical in controlling the interaction between mosquito abundance and human disease risk, as the intrinsic incubation period controls how quickly humans are able to begin transmitting to mosquitoes, and the mortality rate controls for how long mosquitoes are able to remain infectious and transmit to humans.

However, there is a seasonal trend in the weekly posterior p-values for the replicate mosquito surveillance series (Fig. \ref{pseries}), indicating a tendency to overestimate the number of mosquitoes trapped during the downswing of the annual epidemic and underestimate the number of mosquitoes trapped during the upswing. 
This seasonal trend indicates a systematic lack-of-fit.

This lack-of-fit is due in large part to the failure of the estimated mosquito trajectories to capture the periodic seasonal structure in the observed mosquito surveillance data (Fig. \ref{autocorr}).
In particular, the autocorrelation in the posterior replicate mosquito surveillance series decays much faster than in the observed data and remains near zero for lags near a year, whereas the observed data have positive autocorrelation at yearly lags (\ref{autocorr}).
Moreover, the posterior standard deviations of 1 week differences are substantially larger than the observed (Supplemental?), indicating that the observed time series is much smoother than the posterior replicates.
These posterior checks indicate that the case data and mechanistic model are not able to reproduce the observed time series of mosquito trap counts.

\begin{figure}[!h]
\includegraphics[angle = 270, scale = 1]{figures/fig1.eps}
\caption{{\bf Vitoria data and model fits.}
Weekly observations from 2008 to week 34 of 2012 (points), with corresponding median posterior prediction (line) and 80\% credible inverval (gray band). A: reported dengue cases. B: total number of trapped mosquiotes.
}
\label{timeseries}
\end{figure}

\begin{figure}[!h]
%\includegraphics[angle = 270, scale=1]{figures/fig2.eps}
\caption{{\bf Posterior densities of epidemiological parameters.}
A: Intrinsic incubation period. B: Period of cross-immunity. C: Decay rate of infectiousness. D: Mosquito mortality rate.
}
\label{pseries}
\end{figure}

\begin{figure}[!h]
\includegraphics[angle = 270, scale=1]{figures/fig2.eps}
\caption{{\bf Weekly posterior predictive p-values.}
$P(q^{rep}_t > q_t)$ for each week of the year for 2008 through week 34 of 2012. Gray lines show loess-smoothed curves of the p-values for each year. Based on 15000 replicate mosquito surveillance data sets estimated from case reports.
}
\label{pseries}
\end{figure}

\begin{figure}[!h]
\includegraphics[angle = 270, scale=1]{figures/fig3.eps}
\caption{{\bf Posterior autocorrelation function.}
Autocorrelation of observed mosquito trap counts (points) and replicate time series estimated from case reports (vertical lines, indicating 90\% posterior credible interval). 
}
\label{autocorr}
\end{figure}

\section*{Discussion}

\begin{itemize}
\item Human movement \cite{Adams2009, Cosner2009a, Stoddard2009, Dalziel2013}
\item Disease risk possibly driven by high levels of contact with a relatively small proportion of the mosquito population \cite{Canyon1999}.

\end{itemize}

\section*{Conclusions}

\section*{Supporting information}

\subsection*{Full model}


\begin{align}
y_t & \sim \text{NegBin}(\phi_y  (X_t - X_{t-1})N, \eta_y)\\
q_t & \sim \text{NegBin}(\phi_q \tau_t V_{Nt}N, \eta_q)\\
\\ 
\frac{dX}{dt} &= \rho E \\
\frac{dS}{dt} &= b - bS - \alpha V_{I} S + \delta R\\
\frac{dE}{dt} &= \alpha V_{I}S - (\rho + b)E\\
\frac{dI}{dt} &= \rho E - (\gamma + b)I\\
\frac{dR}{dt} &= \gamma I - (\delta + b)R\\
\\
V_{St} & = V_{Nt} - V_E(t) - V_I(t)\\
\frac{dV_E}{dt} &= \alpha I V_S - (\rho_{vt} + d_v)V_E\\
\frac{dV_I}{dt} &= \rho_{vt} V_E - d_v V_I\\
\\
\log(V_{Nt}) & \sim \text{Normal}(\beta \log(V_{Nt-1}), \sigma^2)\\
\log(V_{N0}) & \sim \text{Normal}(0.7, 2)\\
\beta & \sim \text{Normal}(0, 0.4)\\
-\log(\phi_q) & \sim \text{Half-normal}(13, 0.5)\\
\eta_y &\sim \text{Half-normal}(0, 5)\\
\eta_q & \sim \text{Half-normal}(0, 5)\\
T_{\rho} & \sim \text{Gamma}(8.3, 8.3)\\
\rho & = (0.87 T_{\rho})^{-1}\\
T_{\delta} & \sim \text{Gamma}(10, 10)\\
\delta & = (97 T_{\delta})^{-1}\\
\gamma' & \sim \text{Gamma}(100, 100)\\
\gamma & = 3.5\gamma'\\
d_{v}' & \sim \text{Gamma}(100, 100)\\
d_v & = 1.47 d_v'\\
S_0' & \sim \text{Normal}(-0.4, 0.2)\\
E_0' & \sim \text{Normal}(-9, 0.6)\\
I_0' & \sim \text{Normal}(-9, 0.6)\\
R_0' & = 0\\
S_0 & = \frac{\exp(S_0)}{\exp(S_0' + E_0' + I_0' + R_0')} \\
\end{align}

\subsection*{Details of prior specification}

\begin{itemize}
\item \textbf{Period of cross-immunity} ($T_{\delta}$): Reich \emph{et al.} \cite{Reich2013} estimate a mean of 1.88 years (97 weeks) with a 95\% confidence interval of (0.88, 4.31).
We specify a gamma prior on the multiplicative scale to have a mean of one and 0.01 and 0.99 quantiles at roughly $0.88/1.88 = 0.47$ and $4.31/1.88 = 2.3$,, respectively:
\begin{equation}
T_{\delta} \sim \text{Gamma}(10, 10)
\end{equation}
where $\delta = (97 T_\delta)^{-1}$.
Note that we use a slightly more conservative interval than Reich \emph{et al.} to avoid numerical instability at extreme values.
\\
\item \textbf{Intrinsic incubation period} ($T_{\rho}$): Chan and Johansson \cite{Chan2012} estimate a mean of 6.1 days with a 95\% credible interval of (3, 10).
Thus, we specify a gamma prior on the multiplicative scale to have a mean of one, and 0.05 and 0.95 quantiles of $3 / 6.1 = 0.49$ and $10 / 6.1 = 1.6$, respectively:
\begin{equation}
T_{\rho} \sim \text{Gamma}(8.3, 8.3)
\end{equation}
where $\rho = (0.87 T_{\rho})^{-1}$.
\\
\item \textbf{Decay rate of infectiousness} ($\gamma$): Nguyet \emph{et al.} \cite{Nguyet2013} find that there is very little transmission of dengue from humans to mosquitoes after 5-6 days since the onset of fever (with roughly an extra of infectiousness prior to onset of fever).
In reality, an individual's infectiousness decays with time, but the ODE model assumes infectiousness is constant for the (exponentially distributed) duration of the infection period.    
To parameterize the model then, we assume a rough equivalence between the probability of infection from a single individual at a given time and the proportion of individuals still infectious at a given time.
Thus we choose our prior for $\gamma$ to keep the time at which only 5\% of individuals are still infectious between 5 and 7 days, with a mean 5\% threshold of 6 days.
Again we specify the prior on the multiplicative scale to have a mean of one and 0.05 and 0.95 quantiles at $2.99/3.5 = 0.85$ and $4.19 / 3.5 = 1.2$, respectively:
\begin{equation}
\gamma' \sim \text{Gamma}(100, 100)
\end{equation}
where $\gamma = 3.5\gamma'$.
\\
\item \textbf{Mosquito mortality rate} ($d_v$): Brady \textit{et al.} \cite{Brady2013} estimated \textit{Aedes aegypti} temperature-dependent field survivorship curves from over 190 laboratory and field experiments.
They found that increased mortality in the field generally washes out the effect of temperature, so we use their estimates for 25 degrees C to construct our prior for $d_v$.  
We use nonlinear least squares to fit the exponential cdf with rate $d_v$ to the estimated mortality curve provided by Brady \textit{et al.}, which gives $d_v = 1.47$.
We use the same procedure to estimate $d_v$ for the 0.25 and 0.75 quantiles of the estimated mortality curves (assuming symmetry around the mean), which gives 1.6 and 1.37.
To specify the prior, we again scale $d_v'$ to have a mean of one, and 0.25 and 0.75 quantiles of $1.37 / 1.47 = 0.93$ and $1.6 / 1.47 = 1.08$, respectively:
\begin{equation}
d_v' \sim \text{Gamma}(100, 100)
\end{equation}
where $d_v = 1.47d_v'$.
\\
\item \textbf{Initial conditions} ($S_0, E_0, I_0, R_0$): The SEIRS model assumes a constant population size and is scaled so that $S_0+E_0+I_0+R_0 = 1$.
We specify our prior on an unconstrained scale and transform to the simplex with the softmax transformation.
To maintain identifiability of this transformation, we fix $R_0' = 0$.
We choose the prior for $S_0'$  so that $S_0$ has a mean of roughly  0.4, with the mass concentrated between 0.2 and 0.6.
This range produces dynamics with recurring annual outbreaks and avoids the extremes where the disease either does not spread or spreads far too quickly and then dies out.
This is also roughly consistent with \cite{Cardoso2011a} who report notified cases for Vitoria from 1995 to 2008. 
They report that the 2008 outbreak (where our data begins) was driven by serotypes 2 and 3, and 2009 by serotype 2 only, with 13590 reported cases in previous serotype 2 years.
Assuming 12 unreported cases for every reported case, 163,080 total cases, or about 50\% of the population immune to serotype 2 at the start of 2008.
We choose the priors for $E_0'$ and $I_0'$ so that the initial sizes of these compartments were less than 100 individuals, but with a mode slightly off from zero.
This was again chosen primarily to ensure sustained multi-annual transmission.
\\
\item \textbf{Mosquito population size and capture probability} ($\log(V_{N0}), \phi_q$): Wearing and Rohani \cite{Wearing2006} assume 2 mosquitoes per person in their model of dengue transmission, so we use that to inform our prior on $\log(V_{N0})$.
Since $V_{N}$ has been scaled by the human population size, we center the prior for $\log(V_{N0})$ at $\log(2) = 0.7$ and set the variance to concentrate mass between 0.5 and 5 mosquitoes per person.
Maybe see Focks (2000) for further justification of range of mosquitoes/person.
The per-trap mosquito capture probability, $\phi_q$ follows $V_{N0}$ closely and scales the estimated mosquito abundance down to magnitude of the trap counts.
Since $ E[q_t] = \phi_q \tau V_{Nt} N$, we set the prior on $\log(\phi_q)$ to have a mean of $\log(q_1) - \log(\tau_1) - \log(V_{N0}) - \log(N) = -13$ and a variance to match the prior on $\log(V_{N0})$.
\\
\item \textbf{Overdispersion parameters} ($\eta_y, \eta_q$): The negative binomial likelihoods were parameterized as:
\begin{equation}
\text{NegBin}(y | \mu, \eta) = {y + 1/\eta -1 \choose y}\left(\frac{\mu\eta}{1 + \mu\eta}\right)^y\left(\frac{1}{1+\eta\mu}\right)^{1/\eta}
\end{equation}
such that
\begin{align}
E[y] & = \mu\\
\text{Var}[y] &= \mu + \eta\mu^2
\end{align}
Thus, $\eta = 0$ corresponds to Poisson, and $\eta > 0$ leads to overdispersion.
We put a relatively conservative truncated normal prior on both $\eta_y$ and $\eta_q$ that concentrates mass near zero but allows for larger values.
\\
\item \textbf{Autoregressive parameter} ($\beta$): Though the autoregressive framework requires that $|\beta| < 1$, for computational efficiency, we avoid hard bounds and instead use a normal prior centered at zero, with variance such that the majority of mass falls between -1 and 1.
\end{itemize}

\subsection*{Fixed parameters}

\begin{itemize}
\item \textbf{Transmission rate} ($\alpha$): The transmission rate only appears multiplied by a term from the mosquito model (either $V_I$ or $V_S$) and is thus effectively non-identifiable when we are also estimating the overall size of the mosquito population.
Because of this, we fix $\alpha$ at a reasonable value from the literature (though our results should not be sensitive to the exact value used).
The transmission rate is the product of the mosquito feeding rate and the probability of transmission given feeding.
Scott \textit{et al.} \cite{Scott2000} estimate that \textit{Aedes aegypti} females take 0.63 blood meals per day in Puerto Rico, and 0.76 blood meals per day in Thailand.
We take the average of these and set the biting rate at 0.69 per day, or 4.87 per week.
There are relatively few reliable estimates of the transmission probability of DENV, but Lambrechts \textit{et al.} \cite{Lambrechts2011} estimate transmission probabilities close to one  for several related flaviviruses at 25 degrees C (roughly the mean temperature for Vitoria).
\\
\item \textbf{Reporting probability} ($\phi_y$): Silva \textit{et al.} \cite{Silva2016} estimate that there are 12 cases of dengue for every case reported to Brazil's Notifiable Diseases Information System (SINAN, the source of our data), so we fix $\phi_y = 1/12$.
\\
\item \textbf{Variance of mosquito process} ($\sigma$): We found that trying to estimate $\sigma$ introduces substantial computational difficulties, so we choose to fix $\sigma = 0.2$ which allows for maximum weekly fluctuations of roughly 50\%.
This is sufficiently large to allow enough flexibility to fit both the case and mosquito data while guarding against unreasonably large fluctuations in modeled mosquito abundance.
\\
\item \textbf{Extrinsic incubation period} ($1/\rho_{vt}$): Temperature is thought to play an important role in the seasonality of dengue, in particular through its influence on the extrinsic incubation period of the virus in the mosquito  (the amount of time until a mosquito becomes infectious after taking an infectious blood meal).
To account for this, we make $1/\rho_v$ a function of temperature, and force it with weekly mean temperature records for Vitoria downloaded from Weather Underground (URL/citation).
We use the temperature relationship from the recent meta-analysis of \cite{Chan2012}, which incorporates a large number of records and accounts for the interval-censored nature of the data usually available from mosquito transmission studies.
We use a slightly more conservative fitted relationship from a log-normal model that excludes right-censored data and produces a less extreme temperature dependence:
\begin{align}
\frac{1}{\rho_v} &= \exp \left(\mu + \frac{1}{14}\right)\\
\mu & = \exp(1.9 - 0.04 T)
\end{align}
This relationship is similar to those reported in \cite{Focks1995} and \cite{Tjaden2013}, though there is uncertainty about the exact shape of this relationship.
\end{itemize}

% Include only the SI item label in the paragraph heading. Use the \nameref{label} command to cite SI items in the text.
\paragraph*{S1 HMC Diagnostics.}
\label{S1_Diag}
{\bf Trace plots and convergence tests.}

\section*{Acknowledgments}

Thanks!

\nolinenumbers

% Either type in your references using
% \begin{thebibliography}{}
% \bibitem{}
% Text
% \end{thebibliography}
%
% or
%
% Compile your BiBTeX database using our plos2015.bst
% style file and paste the contents of your .bbl file
% here. See http://journals.plos.org/plosone/s/latex for 
% step-by-step instructions.
% 

\bibliographystyle{plos2015}
\bibliography{dengue}


\end{document}

