% Template for PLoS
% Version 3.4 January 2017
%
% % % % % % % % % % % % % % % % % % % % % %
%
% -- IMPORTANT NOTE
%
% This template contains comments intended 
% to minimize problems and delays during our production 
% process. Please follow the template instructions
% whenever possible.
%
% % % % % % % % % % % % % % % % % % % % % % % 
%
% Once your paper is accepted for publication, 
% PLEASE REMOVE ALL TRACKED CHANGES in this file 
% and leave only the final text of your manuscript. 
% PLOS recommends the use of latexdiff to track changes during review, as this will help to maintain a clean tex file.
% Visit https://www.ctan.org/pkg/latexdiff?lang=en for info or contact us at latex@plos.org.
%
%
% There are no restrictions on package use within the LaTeX files except that 
% no packages listed in the template may be deleted.
%
% Please do not include colors or graphics in the text.
%
% The manuscript LaTeX source should be contained within a single file (do not use \input, \externaldocument, or similar commands).
%
% % % % % % % % % % % % % % % % % % % % % % %
%
% -- FIGURES AND TABLES
%
% Please include tables/figure captions directly after the paragraph where they are first cited in the text.
%
% DO NOT INCLUDE GRAPHICS IN YOUR MANUSCRIPT
% - Figures should be uploaded separately from your manuscript file. 
% - Figures generated using LaTeX should be extracted and removed from the PDF before submission. 
% - Figures containing multiple panels/subfigures must be combined into one image file before submission.
% For figure citations, please use "Fig" instead of "Figure".
% See http://journals.plos.org/plosone/s/figures for PLOS figure guidelines.
%
% Tables should be cell-based and may not contain:
% - spacing/line breaks within cells to alter layout or alignment
% - do not nest tabular environments (no tabular environments within tabular environments)
% - no graphics or colored text (cell background color/shading OK)
% See http://journals.plos.org/plosone/s/tables for table guidelines.
%
% For tables that exceed the width of the text column, use the adjustwidth environment as illustrated in the example table in text below.
%
% % % % % % % % % % % % % % % % % % % % % % % %
%
% -- EQUATIONS, MATH SYMBOLS, SUBSCRIPTS, AND SUPERSCRIPTS
%
% IMPORTANT
% Below are a few tips to help format your equations and other special characters according to our specifications. For more tips to help reduce the possibility of formatting errors during conversion, please see our LaTeX guidelines at http://journals.plos.org/plosone/s/latex
%
% For inline equations, please be sure to include all portions of an equation in the math environment.  For example, x$^2$ is incorrect; this should be formatted as $x^2$ (or $\mathrm{x}^2$ if the romanized font is desired).
%
% Do not include text that is not math in the math environment. For example, CO2 should be written as CO\textsubscript{2} instead of CO$_2$.
%
% Please add line breaks to long display equations when possible in order to fit size of the column. 
%
% For inline equations, please do not include punctuation (commas, etc) within the math environment unless this is part of the equation.
%
% When adding superscript or subscripts outside of brackets/braces, please group using {}.  For example, change "[U(D,E,\gamma)]^2" to "{[U(D,E,\gamma)]}^2". 
%
% Do not use \cal for caligraphic font.  Instead, use \mathcal{}
%
% % % % % % % % % % % % % % % % % % % % % % % % 
%
% Please contact latex@plos.org with any questions.
%
% % % % % % % % % % % % % % % % % % % % % % % %

\documentclass[10pt,letterpaper]{article}
\usepackage[top=0.85in,left=2.75in,footskip=0.75in]{geometry}

% amsmath and amssymb packages, useful for mathematical formulas and symbols
\usepackage{amsmath,amssymb}

% Use adjustwidth environment to exceed column width (see example table in text)
\usepackage{changepage}

% Use Unicode characters when possible
\usepackage[utf8x]{inputenc}

% textcomp package and marvosym package for additional characters
\usepackage{textcomp,marvosym}

% cite package, to clean up citations in the main text. Do not remove.
\usepackage{cite}

% Use nameref to cite supporting information files (see Supporting Information section for more info)
\usepackage{nameref,hyperref}

% line numbers
\usepackage[right]{lineno}

% ligatures disabled
\usepackage{microtype}
\DisableLigatures[f]{encoding = *, family = * }

% color can be used to apply background shading to table cells only
\usepackage[table]{xcolor}

% array package and thick rules for tables
\usepackage{array}

% create "+" rule type for thick vertical lines
\newcolumntype{+}{!{\vrule width 2pt}}

% create \thickcline for thick horizontal lines of variable length
\newlength\savedwidth
\newcommand\thickcline[1]{%
  \noalign{\global\savedwidth\arrayrulewidth\global\arrayrulewidth 2pt}%
  \cline{#1}%
  \noalign{\vskip\arrayrulewidth}%
  \noalign{\global\arrayrulewidth\savedwidth}%
}

% \thickhline command for thick horizontal lines that span the table
\newcommand\thickhline{\noalign{\global\savedwidth\arrayrulewidth\global\arrayrulewidth 2pt}%
\hline
\noalign{\global\arrayrulewidth\savedwidth}}


% Remove comment for double spacing
%\usepackage{setspace} 
%\doublespacing

% Text layout
\raggedright
\setlength{\parindent}{0.5cm}
\textwidth 5.25in 
\textheight 8.75in

% Bold the 'Figure #' in the caption and separate it from the title/caption with a period
% Captions will be left justified
\usepackage[aboveskip=1pt,labelfont=bf,labelsep=period,justification=raggedright,singlelinecheck=off]{caption}
\renewcommand{\figurename}{Fig}

% Use the PLoS provided BiBTeX style
\bibliographystyle{plos2015}

% Remove brackets from numbering in List of References
\makeatletter
\renewcommand{\@biblabel}[1]{\quad#1.}
\makeatother

% Leave date blank
\date{}

% Header and Footer with logo
\usepackage{lastpage,fancyhdr,graphicx}
\usepackage{epstopdf}
\pagestyle{myheadings}
\pagestyle{fancy}
\fancyhf{}
\setlength{\headheight}{27.023pt}
\lhead{\includegraphics[width=2.0in]{PLOS-submission.eps}}
\rfoot{\thepage/\pageref{LastPage}}
\renewcommand{\footrule}{\hrule height 2pt \vspace{2mm}}
\fancyheadoffset[L]{2.25in}
\fancyfootoffset[L]{2.25in}
\lfoot{\sf PLOS}

%% Include all macros below

\newcommand{\lorem}{{\bf LOREM}}
\newcommand{\ipsum}{{\bf IPSUM}}

%% END MACROS SECTION


\begin{document}
\vspace*{0.2in}

% Title must be 250 characters or less.
\begin{flushleft}
{\Large
\textbf\newline{Linking mosquito surveillance to dengue risk through Bayesian mechanistic modeling}
}
\newline
% Insert author names, affiliations and corresponding author email (do not include titles, positions, or degrees).
\\
Clinton Leach\textsuperscript{1*},
Colleen Webb\textsuperscript{1},
Kim Pepin\textsuperscript{2},
Alvaro Eiras\textsuperscript{3},
Mevin Hooten\textsuperscript{4},
Jennifer Hoeting\textsuperscript{4}

\bigskip
\textbf{1} Affiliation Dept/Program/Center, Institution Name, City, State, Country
\\
\textbf{2} Affiliation Dept/Program/Center, Institution Name, City, State, Country
\\
\textbf{3} Affiliation Dept/Program/Center, Institution Name, City, State, Country
\\
\bigskip

% Insert additional author notes using the symbols described below. Insert symbol callouts after author names as necessary.
% 
% Remove or comment out the author notes below if they aren't used.


% Use the asterisk to denote corresponding authorship and provide email address in note below.
* clint.leach@colostate.edu

\end{flushleft}
% Please keep the abstract below 300 words
\section*{Abstract}

We know mosquitoes spread dengue, but mosquito abundance, as estimated from trap-based surveillance data, is not causally linked to human disease risk.

% Please keep the Author Summary between 150 and 200 words
% Use first person. PLOS ONE authors please skip this step. 
% Author Summary not valid for PLOS ONE submissions.   
\section*{Author summary}

We know mosquitoes spread dengue, but mosquito abundance, as estimated from trap-based surveillance data, is not causally linked to human disease risk.


\linenumbers

% Use "Eq" instead of "Equation" for equation citations.
\section*{Introduction}

Dengue fever is a massive global public health burden, with millions of cases per year in Brazil alone (CITATION).  
Since the dengue virus (DENV) is vectored by the mosquito \textit{Aedes aegypti}, dengue fever is prevented primarily through mosquito control programs \cite{Achee2015}.
Though eradication of \textit{Aedes aegypti} is infeasible, effectively targeted control interventions may still substantially reduce disease risk.

Cities deploy mosquito control based on trap-based mosquito surveillance data.
For instance, the "MI-Dengue" system deployed throughout several cities in Brazil uses a grid of mosquito sticky-traps (MosquiTRAPs) for spatial prioritization of source reduction and larvicide application \cite{Eiras2009}.
Comparing cities with and without this system, \cite{Pepin2013} estimated that it prevented X cases and saved X dollars in economic losses from 20XX - 20XX.
Though this system has been effective, the mosquito indices and thresholds used to prioritize control efforts remain somewhat ad-hoc.
Using such surveillance data to efficiently target limited control resources requires a more detailed understanding of the relationship between the components of mosquito abundance measured by traps and disease risk.

Many of the attempts to establish this relationship statistically have relied on tests of the ability of mosquito abundance measures to predict human disease risk in regression models.
For example, \cite{Pepin2015} found that models that included MosquiTRAP surveillance data failed to predict weekly dengue cases any better than models that include lagged case data alone.
However, these models fail to account for the mechanistic, non-linear relationship between mosquito population dynamics and dengue epidemiology.
In particular, the cycle of transmission between humans and dengue is controlled not just by mosquito population size, but also by the emergence of new susceptible adults and the mortality rate.

In this paper, we seek to connect mosquito surveillance data with human case reports through the use of a Bayesian, mechanistic model.
We first propose a differential equation model for dengue epidemiology.
We then embed this model into a Bayesian statistical framework that allows us to estimate latent time series of mosquito demographic parameters from time series of mosquito trap counts and reported cases of dengue fever.
Lastly, we explore the ability of this fitted model to predict annual dengue outbreaks and its utility for targeting control efforts.

\section*{Methods}

\subsection*{Study system and data}

Vitoria is a coastal city and the capital of the state of Espirito Santo, Brazil, with a population of 327,801 in the city proper, and an additional 1.5 million people in the greater metropolitan area.  
Since 2008, the company Ecovec has monitored mosquito abundance for the city using approximately 1327 sticky traps (MosquiTRAP, \cite{Eiras2009}) arranged in a grid across the city.
Each trap is checked weekly and the mosquitoes inside counted and identified, providing us with 243 weeks (2008 through week 34 of 2012) of counts of gravid female \emph{Aedes aegypti}.
Dengue is a mandatory notifiable disease, and thus the city's Ministry of Health Secretary maintains a database of weekly notified probable dengue cases (i.e. medical care sought for dengue-like symptoms) for this same time period.

\subsection*{Process Model}

Dengue epidemiology is complicated considerably by the presence of four simultaneously circulating serotypes (referred to as DENV-1, DENV-2, DENV-3, DENV-4).
Infection with one serotype confers life-long immunity to that serotype, along with temporary immunity to other serotypes.  
As this cross-immunity wanes, antibodies from the previous infection can actually result in antibody-dependent enhancement (ADE), wherein hosts are more susceptible to infection with the other serotypes and more likely to develop severe symptoms (i.e. dengue haemorrhagic fever or dengue shock syndrome).
The strength and duration of these different inter-serotype interactions are not well understood, though different models suggest that temporary cross-immunity alone (without ADE) is sufficient to reproduce observed multi-annual dynamics in Thailand \cite{Wearing2006,Reich2013}.
Similarly, \cite{Aguiar2013} find that capturing both primary and secondary infections and the period of cross-immunity is critical, but that explicitly including all four serotypes (at great cost to model complexity) does not perform much better than a two serotype model.

Explicitly accounting for interactions between serotypes, even only two of the four, leads to a large and complex mechanistic model.
Moreover, since dengue case reports do not include information on serotype, we do not have enough information to inform the dynamics of individual serotypes.
As such, we simplify the model of \cite{Wearing2006} to an SEIRS framework, which drops serotype-specific dynamics but preserves the period of cross-immunity and the possibility of reinfection.
In this framework, susceptible individuals ($S$) become exposed ($E$) through contact with infectious mosquitoes ($V_I$).
Following a latent period ($\frac{1}{\rho}$), exposed individuals become infectious ($I$) at which point they can infect susceptible mosquitoes ($V_S$).
Infectious individuals recover at rate $\gamma$ and subsequenty remain immune for a period ($\frac{1}{\delta}$) after which they re-enter the susceptible class.

Similarly, susceptible mosquitoes ($V_S$) become exposed by biting infectious humans and pass through a temperature-dependent latent period ($\frac{1}{\rho_{vt}}$) before becoming infectious ($V_I$).
Total mosquito population size ($V_N$) is controlled by a periodically varying net emergence rate, $b_t$, and a stochastic, weather dependent death rate, $d_t$.
Captured mosquitoes ($V_C$) accumulate at rate $\phi_q \tau_t$, where $\phi_q$ is the per-trap capture rate, and $\tau_t$ is the number of traps deployed in week $t$.

The complete differential equation model is then given as:
\begin{align} 
\frac{dS}{dt} &= bN - bS - \lambda \frac{V_{I}}{N} S + \delta R\\
\frac{dE}{dt} &= \lambda \frac{V_{I}}{N} S - (\rho + b)E\\
\frac{dI}{dt} &= \rho E - (\gamma + b)I\\
\frac{dR}{dt} &= \gamma I - (\delta + b)R\\
\frac{dV_S}{dt} & = b_t - \lambda \frac{I}{N} V_S - (d_t + \phi_q \tau_t) V_S \\
\frac{dV_E}{dt} &= \lambda \frac{I}{N} V_S - (\rho_{vt} + d_t + \phi_q \tau_t)V_E\\
\frac{dV_I}{dt} &= \rho_{vt} V_E - (d_t + \phi_q \tau_t) V_I\\
\frac{dV_C}{dt} & = \phi_q \tau_t V_N\\
V_N &= V_S + V_E + V_I,
\end{align}
where
\begin{equation}
\log(b_t) = \alpha_{0} + \alpha_1 \sin(2\pi t / 52) + \alpha_2 \cos(2\pi t / 52),
\end{equation}
and
\begin{align}
\log(d_{t}) & = \beta_{0} + X\beta + \epsilon_t\\
\epsilon_t & \sim \text{Normal}(0, \sigma).
\end{align}

\subsection*{Data model}

To connect the differential equation model the the observed case reports, we add an extra state, $C$, that collects the cumulative number of transitions from the exposed to infectious class (assuming that case reporting coincides with the onset of symptoms).
We then model the number of new cases reported in week $t$ ($y_t$) as:
\begin{equation}
y_t  \sim \text{NegBin}(\phi_y (C_t - C_{t-1}), \eta_y),
\end{equation}
where $\phi_y$ is the reporting probability, $C_t - C_{t-1}$ is the number of new infectious individuals in week $t$, and $\eta_y$ controls the overdispersion relative to the Poisson.

We similarly model the number of mosquitoes trapped in week $t$ ($q_t$) as:
\begin{equation}
q_t \sim \text{NegBin}(V_{Ct} - V_{Ct-1}, \eta_q),
\end{equation}
where $V_{Ct} - V_{Ct-1}$ is the number of new mosquitoes captured in week $t$, and $\eta_q$ controls overdispersion relative to the Poisson.

\subsection*{Parameterization and priors}

Several of the parameters in this model are assumed to fixed and known. 
The population size and average life span (which we use to parameterize the birth/death rate) for Vitoria are taken from the 2010 census.
To maintain identifiability, the transmission rate ($\lambda$) and case reporting probability ($\phi_y$) are also fixed at literature values.
Lastly, the extrinsic incubation period in mosquitoes is modeled as a function of weekly mean temperature and forced with weather station data obtained from WeatherUnderground.

The remaining parameters include the epidemiological parameters ($\rho$, $\gamma$, $\delta$), the initial conditions of the model ($S_0$, $E_0$, $I_0$, $R_0$, $V_{N0}$), the regression parameters controlling mean, phase, and amplitude of the emergence rate ($\alpha_0,\alpha_1,\alpha_2$), the regression and variance parameters for the mosquito death rate process ($\beta_0,\mathbf{\beta},\sigma$), and the remaining measurement parameters ($\phi_q$, $\eta_y$, $\eta_q$).  
Where possible, we place informative priors on these parameters based on existing laboratory and field studies (see Table \ref{fixedparms} for means and Supplemental Material for detailed explanations).

For any given values of these parameters, the mosquito process model should be flexible enough to produce a good fit to the case reports data.
As a result, we expect that these parameters will be informed primarily by the mosquito surveillance data (i.e. these parameters add degrees of freedom that let the mosquito process model fit the surveillance data better while maintaining a good fit to the case data).

\begin{table}[!ht]
\label{fixedparms}
\begin{adjustwidth}{-2.25in}{0in} 
\begin{center}
\caption{Model parameters and their values.  Parameters above the rule are fixed, while parameters below the rule are random, with the value giving the prior mean.}
\begin{tabular}{llll}
Parameter & Description & Value & Citation\\
\hline
$N$ & Population size & 327801 & Vitoria Census\\
$1/d$ & Human life-span & 76 years & Vitoria Census\\
$\lambda$ & Transmission rate & 4.87 week$^{-1}$ & \cite{Scott2000}\\
$\phi_y$ & Reporting probability & 0.083 & \cite{Silva2016}\\
$1/\rho_{vt}$ & Extrinsic incubation period & $\exp \left(\exp(1.9 - 0.04 T_t) + \frac{1}{14}\right)$ & \cite{Chan2012}\\
$V_{E0}$ & Initial exposed mosquitoes &  0 & \\
$V_{I0}$ & Initial infectious mosquitoes & 0 & \\
\hline
$\exp(\beta_0)$ & Mean vector death rate & 1.47 week$^{-1}$ & \cite{Brady2013} \\
$\exp(\alpha_0)$ & Mean vector net emergence rate & ?? & \\
$1/\rho$ & Latent period in host & 0.87 weeks  & \cite{Chan2012}\\
$\gamma$ & Rate of loss of infectiousness & 3.5 week$^{-1}$ & \cite{Nguyet2013}\\
$1/\delta$ & Period of cross-immunity & 97 weeks &  \cite{Reich2013}\\
$S_0$ & Proportion initially susceptible & 0.4 & \cite{Cardoso2011a} \\
$E_0$ & Proportion initially exposed & $8\times 10 ^ {-5}$ & \\
$I_0$ & Proportion initially infectious & $8\times 10 ^ {-5}$ & \\
$\phi_q$ & Per-trap mosquito capture rate & $2 \times 10^{-6}$ week$^{-1}$ & 
\end{tabular}
\end{center}
\end{adjustwidth}
\end{table}

\subsection*{Model validation}

We evaluate the model's predictive ability by fitting to the first 191 weeks of data (up to the trough following the fourth outbreak), and predicting case reports for the subsequent year.
In order to make these predictions, we project weather covariates into the future using weekly averages of the first 191 weeks.

\subsection*{Implementation}
 
Samples from the posterior distribution were generated using Hamiltonian Monte Carlo (CITATIONS) implemented in the rstan package for R (CITATIONS). 
We ran 3 chains with different starting values for 10,000 iterations each, discarding the first 5,000 as burn-in.
Within stan, the solution to the differential equation model was approximated with an Euler scheme with a time step of 1 day.  

\section*{Results}

Questions:
\begin{itemize}
\item How wells does the model fit the case reports and mosquito surveillance data?
\item What do the estimated time series of demographic rates look like? How smooth are they? How strong is the periodic signal relative to noise?
\item How far/how well can we make predictions?
\item What do the estimated demographic rates tell us about efficient implementation of control?
\end{itemize}


\begin{figure}[!h]
%\includegraphics[angle = 270, scale = 1]{figures/fig1.eps}
\caption{{\bf Vitoria data and model fits.}
Weekly observations from 2008 to week 34 of 2012 (points), with corresponding median posterior prediction (line) and 80\% credible inverval (gray band). A: reported dengue cases. B: total number of trapped mosquiotes.
}
\label{timeseries}
\end{figure}


\section*{Discussion}

\begin{itemize}
\item Human movement \cite{Adams2009, Cosner2009a, Stoddard2009, Dalziel2013}
\item Disease risk possibly driven by high levels of contact with a relatively small proportion of the mosquito population \cite{Canyon1999}.
\item Explicitly targeting mosquitoes that have been implicated in transmission through the use of contact tracing \cite{Vazquez-Prokopec2017}
\end{itemize}

\section*{Conclusions}

\section*{Supporting information}

% Include only the SI item label in the paragraph heading. Use the \nameref{label} command to cite SI items in the text.
\paragraph*{S1 Text.}
\label{S1_Diag}
{\bf Full description of model and priors.}

\section*{Acknowledgments}

Thanks!

\nolinenumbers

% Either type in your references using
% \begin{thebibliography}{}
% \bibitem{}
% Text
% \end{thebibliography}
%
% or
%
% Compile your BiBTeX database using our plos2015.bst
% style file and paste the contents of your .bbl file
% here. See http://journals.plos.org/plosone/s/latex for 
% step-by-step instructions.
% 

\bibliographystyle{plos2015}
\bibliography{dengue}


\end{document}

