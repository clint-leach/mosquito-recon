% Template for PLoS
% Version 3.4 January 2017
%
% % % % % % % % % % % % % % % % % % % % % %
%
% -- IMPORTANT NOTE
%
% This template contains comments intended 
% to minimize problems and delays during our production 
% process. Please follow the template instructions
% whenever possible.
%
% % % % % % % % % % % % % % % % % % % % % % % 
%
% Once your paper is accepted for publication, 
% PLEASE REMOVE ALL TRACKED CHANGES in this file 
% and leave only the final text of your manuscript. 
% PLOS recommends the use of latexdiff to track changes during review, as this will help to maintain a clean tex file.
% Visit https://www.ctan.org/pkg/latexdiff?lang=en for info or contact us at latex@plos.org.
%
%
% There are no restrictions on package use within the LaTeX files except that 
% no packages listed in the template may be deleted.
%
% Please do not include colors or graphics in the text.
%
% The manuscript LaTeX source should be contained within a single file (do not use \input, \externaldocument, or similar commands).
%
% % % % % % % % % % % % % % % % % % % % % % %
%
% -- FIGURES AND TABLES
%
% Please include tables/figure captions directly after the paragraph where they are first cited in the text.
%
% DO NOT INCLUDE GRAPHICS IN YOUR MANUSCRIPT
% - Figures should be uploaded separately from your manuscript file. 
% - Figures generated using LaTeX should be extracted and removed from the PDF before submission. 
% - Figures containing multiple panels/subfigures must be combined into one image file before submission.
% For figure citations, please use "Fig" instead of "Figure".
% See http://journals.plos.org/plosone/s/figures for PLOS figure guidelines.
%
% Tables should be cell-based and may not contain:
% - spacing/line breaks within cells to alter layout or alignment
% - do not nest tabular environments (no tabular environments within tabular environments)
% - no graphics or colored text (cell background color/shading OK)
% See http://journals.plos.org/plosone/s/tables for table guidelines.
%
% For tables that exceed the width of the text column, use the adjustwidth environment as illustrated in the example table in text below.
%
% % % % % % % % % % % % % % % % % % % % % % % %
%
% -- EQUATIONS, MATH SYMBOLS, SUBSCRIPTS, AND SUPERSCRIPTS
%
% IMPORTANT
% Below are a few tips to help format your equations and other special characters according to our specifications. For more tips to help reduce the possibility of formatting errors during conversion, please see our LaTeX guidelines at http://journals.plos.org/plosone/s/latex
%
% For inline equations, please be sure to include all portions of an equation in the math environment.  For example, x$^2$ is incorrect; this should be formatted as $x^2$ (or $\mathrm{x}^2$ if the romanized font is desired).
%
% Do not include text that is not math in the math environment. For example, CO2 should be written as CO\textsubscript{2} instead of CO$_2$.
%
% Please add line breaks to long display equations when possible in order to fit size of the column. 
%
% For inline equations, please do not include punctuation (commas, etc) within the math environment unless this is part of the equation.
%
% When adding superscript or subscripts outside of brackets/braces, please group using {}.  For example, change "[U(D,E,\gamma)]^2" to "{[U(D,E,\gamma)]}^2". 
%
% Do not use \cal for caligraphic font.  Instead, use \mathcal{}
%
% % % % % % % % % % % % % % % % % % % % % % % % 
%
% Please contact latex@plos.org with any questions.
%
% % % % % % % % % % % % % % % % % % % % % % % %

\documentclass[10pt,letterpaper]{article}
\usepackage[top=0.85in,left=2.75in,footskip=0.75in]{geometry}

% amsmath and amssymb packages, useful for mathematical formulas and symbols
\usepackage{amsmath,amssymb}

% Use adjustwidth environment to exceed column width (see example table in text)
\usepackage{changepage}

% Use Unicode characters when possible
\usepackage[utf8x]{inputenc}

% textcomp package and marvosym package for additional characters
\usepackage{textcomp,marvosym}

% cite package, to clean up citations in the main text. Do not remove.
\usepackage{cite}

% Use nameref to cite supporting information files (see Supporting Information section for more info)
\usepackage{nameref,hyperref}

% line numbers
\usepackage[right]{lineno}

% ligatures disabled
\usepackage{microtype}
\DisableLigatures[f]{encoding = *, family = * }

% color can be used to apply background shading to table cells only
\usepackage[table]{xcolor}

% array package and thick rules for tables
\usepackage{array}

% create "+" rule type for thick vertical lines
\newcolumntype{+}{!{\vrule width 2pt}}

% create \thickcline for thick horizontal lines of variable length
\newlength\savedwidth
\newcommand\thickcline[1]{%
  \noalign{\global\savedwidth\arrayrulewidth\global\arrayrulewidth 2pt}%
  \cline{#1}%
  \noalign{\vskip\arrayrulewidth}%
  \noalign{\global\arrayrulewidth\savedwidth}%
}

% \thickhline command for thick horizontal lines that span the table
\newcommand\thickhline{\noalign{\global\savedwidth\arrayrulewidth\global\arrayrulewidth 2pt}%
\hline
\noalign{\global\arrayrulewidth\savedwidth}}


% Remove comment for double spacing
%\usepackage{setspace} 
%\doublespacing

% Text layout
\raggedright
\setlength{\parindent}{0.5cm}
\textwidth 5.25in 
\textheight 8.75in

% Bold the 'Figure #' in the caption and separate it from the title/caption with a period
% Captions will be left justified
\usepackage[aboveskip=1pt,labelfont=bf,labelsep=period,justification=raggedright,singlelinecheck=off]{caption}
\renewcommand{\figurename}{Fig}

% Use the PLoS provided BiBTeX style
\bibliographystyle{plos2015}

% Remove brackets from numbering in List of References
\makeatletter
\renewcommand{\@biblabel}[1]{\quad#1.}
\makeatother

% Leave date blank
\date{}

% Header and Footer with logo
\usepackage{lastpage,fancyhdr,graphicx}
\usepackage{epstopdf}
\pagestyle{myheadings}
\pagestyle{fancy}
\fancyhf{}
\setlength{\headheight}{27.023pt}
\lhead{\includegraphics[width=2.0in]{PLOS-submission.eps}}
\rfoot{\thepage/\pageref{LastPage}}
\renewcommand{\footrule}{\hrule height 2pt \vspace{2mm}}
\fancyheadoffset[L]{2.25in}
\fancyfootoffset[L]{2.25in}
\lfoot{\sf PLOS}

%% Include all macros below

\newcommand{\lorem}{{\bf LOREM}}
\newcommand{\ipsum}{{\bf IPSUM}}

%% END MACROS SECTION


\begin{document}
\vspace*{0.2in}

% Title must be 250 characters or less.
\begin{flushleft}
{\Large
\textbf\newline{Linking mosquito surveillance to dengue fever through Bayesian mechanistic modeling}
}
\newline
% Insert author names, affiliations and corresponding author email (do not include titles, positions, or degrees).
\\
Clinton B. Leach\textsuperscript{1,2*},
Jennifer A. Hoeting\textsuperscript{2}
Kim M. Pepin\textsuperscript{3},
Alvaro E. Eiras\textsuperscript{4},
Mevin B. Hooten\textsuperscript{5,6,2},
Colleen T. Webb\textsuperscript{1}


\bigskip
\textbf{1} Graduate Degree Program in Ecology, Colorado State University, Fort Collins, CO, USA
\\
\textbf{2} Department of Statistics, Colorado State University, Fort Collins, CO, USA
\\
\textbf{3} National Wildlife Research Center, United States Department of Agriculture, Wildlife Services, Fort Collins, CO, USA
\\
\textbf{4} Departamento de Parasitologia, Universidade Federal de Minas Gerais, Belo Horizonte, MG, Brazil
\\
\textbf{5} U.S. Geological Survey, Colorado Cooperative Fish and Wildlife Research Unit
\\
\textbf{6} Department of Fish, Wildlife, and Conservation Biology, Colorado State University
\bigskip

% Insert additional author notes using the symbols described below. Insert symbol callouts after author names as necessary.
% 
% Remove or comment out the author notes below if they aren't used.


% Use the asterisk to denote corresponding authorship and provide email address in note below.
* clint.leach@colostate.edu

\end{flushleft}
% Please keep the abstract below 300 words
\section*{Abstract}

Our ability to effectively prevent the transmission of the dengue virus through targeted control of its vector, \emph{Aedes aegypti}, depends critically on our understanding of the link between mosquito abundance and human disease risk.
Mosquito and clinical surveillance data are widely collected, but linking them requires a modeling framework that accounts for the complex non-linear mechanisms involved in transmission.
Most critical are the bottleneck in transmission imposed by mosquito lifespan relative to the virus' extrinsic incubation period, and the dynamics of human immunity.
We developed a differential equation model of dengue transmission and embedded it in a Bayesian hierarchical framework that allowed us to estimate latent time series of mosquito demographic rates from mosquito trap counts and dengue case reports from the city of Vit\'oria, Brazil.
We used the fitted model to explore how the timing of a pulse of adult mosquito control influences its effect on the human disease burden in the following year.
We found that control was generally more effective when implemented in periods of relatively low mosquito mortality (when mosquito abundance was also generally low).
In particular, control implemented in early September (week 34 of the year) produced the largest reduction in predicted human case reports over the following year.
This highlights the potential long-term utility of broad, off-peak-season mosquito control in addition to existing, locally targeted within-season efforts.
Further, uncertainty in the effectiveness of control interventions was driven largely by posterior variation in the average mosquito mortality rate (closely tied to total mosquito abundance) with lower mosquito mortality generating systems more vulnerable to control.
Broadly, these correlations suggest that mosquito control is most effective in situations in which transmission is already limited by mosquito abundance.

% Please keep the Author Summary between 150 and 200 words
% Use first person. PLOS ONE authors please skip this step. 
% Author Summary not valid for PLOS ONE submissions.   
\section*{Author summary}

The contribution of the mosquito vector \emph{Aedes aegypti} to the spread of dengue fever depends not only on their abundance, but also on the likelihood of an exposed mosquito living long enough to incubate the dengue virus and subsequently transmit it to a susceptible human host.
We developed a mechanistic model that accounts for the role of this process in the dynamics of dengue fever and fit the model to a time series of human case reports and mosquito trap counts from the city of Vit\'oria, Brazil.
We then used this fitted model to simulate the effect of mosquito control implemented at different times of the year and found that mosquito control leads to the largest reduction in human dengue cases over the following year when implemented in early September, during the dengue off-season.
Further, the effectiveness of mosquito control was strongly negatively correlated with the overall average abundance of mosquitoes.
Together with the timing of effective control, these results suggest that mosquito control is most effective when mosquitoes are already limiting to transmission.

\linenumbers

% Use "Eq" instead of "Equation" for equation citations.

\section*{Introduction}

Dengue fever is a massive global public health burden, with millions of cases per year \cite{Bhatt2013}.  
Because the dengue virus (DENV) is transmitted by the mosquito \textit{Aedes aegypti}, dengue fever is prevented primarily through mosquito control programs \cite{Achee2015}.
Though there have been documented successes, there is limited evidence for the long-term sustainability and effectiveness of these control programs \cite{Morrison2008}.
As a result, there is a growing recognition that effective control needs to be guided by high quality vector surveillance, together with quantitative tools that synthesize vector surveillance with clinical surveillance, account for local epidemiology, and facilitate local decision making \cite{Morrison2008, Scott2010b}.
Moreover, mosquito control needs to be guided by an understanding of the link between mosquito abundance and disease risk so that the mosquitoes most responsible for transmission can be targeted \cite{Scott2010a, Scott2010b}.

Many of the attempts to establish this link have found a weak relationship between mosquito abundance indices and incidence of disease in humans \cite{Bowman2014, Pepin2015, Cromwell2017}.
However, these attempts often do not account for the complex, non-linear interactions that mediate the relationship between mosquito abundance and human disease.
In particular, host immunity is a key intrinsic driver of infectious disease dynamics, and conditions favorable for transmission can only lead to an outbreak of disease when there is a sufficiently large population of susceptible hosts \cite{Koelle2004, Koelle2005}.
As such, the ability of mosquitoes to contribute to DENV transmission depends critically on the level of immunity in the human population \cite{Scott2010a}.
Further, the cycle of transmission between humans and mosquitoes is influenced not just by mosquito abundance, but also by mosquito survival relative to the virus incubation period in mosquitoes \cite{Smith2012}.
In fact, whether or not an exposed mosquito will survive long enough to become infectious represents a critical bottleneck in the transmission process and leads to nonlinear dependence of transmission on mosquito survival \cite{Smith2012}.

The importance of intrinsic nonlinearities, potentially alongside seasonality and stochastic forcing \cite{Ellner1998, Koelle2004, Grenfell2002}, in governing human disease risk highlights the need to integrate mechanistic modeling into the quantitative tools used to understand the effects of control interventions.
Such mechanistic models can often perform better than complex autoregressive statistical models in describing and forecasting population dynamics \cite{Reilly2005}.
Moreover, in the absence of case-control studies, mechanistic models can provide scenario-based tools that can be used to predict the effect of management actions \cite{Buckland2007}.

Differential equation models provide a natural way to describe mechanistic processes, but that description must also account for sources of uncertainty \cite{Hotelling1927, Wikle2010}.
In particular, the values of parameters (e.g., the average length of time for which a host is infectious) are often uncertain, which can lead to large uncertainty about the effects of management actions \cite{Elderd2006}.
The structure of the processes themselves can be uncertain \cite{Ellner1998}, and needs to be informed by available, often noisy, data.
Bayesian hierarchical modeling provides a coherent framework to account for and integrate this uncertainty across the three levels of the model (data, process, and parameters \cite{Berliner1996, Cressie2009}).

In what follows, we integrate these elements -- a detailed mechanistic model of dengue transmission with a full Bayesian accounting of uncertainty --  to better understand the interplay of forces governing dengue dynamics and their interaction with potential vector control interventions.
We apply this framework to clinical and entomological surveillance data from the city of Vit\'oria, Brazil.
These data allow us to estimate a latent time series of mosquito mortality rates that modulate the transmission process and link mosquito abundance to human disease.  
We then use the fitted model to explore how perturbations to the mosquito population propagate and interact with the nonlinearities of dengue transmission to better inform mosquito control efforts.

\section*{Methods}

\subsection*{Study system and data}

Vit\'oria is a coastal city and the capital of the state of Esp\'irito Santo, Brazil, with a population of 327,801 as of 2010 \cite{vitpop}.
Since 2008, the company Ecovec has monitored mosquito abundance for the city using approximately 1327 sticky traps (MosquiTRAP, \cite{Eiras2009}) arranged in a roughly 250m grid across the city \cite{Pepin2015, Lana2018}.
Each trap is checked weekly and the mosquitoes inside counted and identified, with the results sent to a central database that city managers then use to map mosquito infestations and target control.
These data comprise 243 weeks (week 1 of 2008 through week 34 of 2012) of total city-wide counts of trapped gravid female \emph{Aedes aegypti}.
It is important to note that this time series reflects both natural fluctuations in mosquito density and fluctuations driven by the city's existing mosquito control program.
In addition, dengue fever is a mandatory notifiable disease, and thus the city's Ministry of Health Secretary maintains a database of weekly notified probable dengue cases (i.e., medical care sought for dengue-like symptoms) for the same time period.
We did not obtain IRB approval for this work as the data we worked with were received by us as aggregated data at the weekly and neighborhood level.  
Hence, this research does not meet the definition of human subjects research requiring IRB approval.  
The data were analyzed in the aggregated form, which protects the anonymity of individuals.

\subsection*{Process Model}

Dengue epidemiology is complicated considerably by the presence of four simultaneously circulating serotypes.
Infection with one serotype confers life-long immunity to that serotype, along with temporary immunity to other serotypes \cite{Wearing2006}.  
As this cross-immunity wanes, antibodies from the previous infection can result in antibody-dependent enhancement (ADE), wherein human hosts are more susceptible to infection with the other serotypes and more likely to develop severe symptoms (i.e., dengue haemorrhagic fever or dengue shock syndrome)\cite{Wearing2006}.
The strength and duration of these different inter-serotype interactions are not well understood, although different models suggest that temporary cross-immunity alone (without ADE) is sufficient to reproduce observed multi-annual dynamics in Thailand \cite{Wearing2006,Reich2013}.

Explicitly capturing the cross-immune interactions among all four serotypes, or even only two of the four \cite{Aguiar2013}, leads to a large and complex mechanistic model.
Moreover, because dengue case reports do not identify serotype, there is not enough information in our data to inform the dynamics of individual serotypes.
As such, we captured temporary cross-immunity, and the potential for multiple sequential infections, as simply and tractably as possible in a susceptible-exposed-infectious-recovered-susceptible (SEIRS) compartment model, similar to \cite{Newton1992, Burattini2008, Pinho2010, Hooten2010}.
Although it captures the critical influence of temporary immunity, this framework does not account for the potential relationship between an individual's infection history and the likelihood that a new infection with be symptomatic (and thus reported).
In particular, secondary infections appear more likely to be symptomatic than primary infections \cite{Imai2016,Clapham2017}, while third and fourth infections appear much less likely to be symptomatic \cite{Olkowski2013} (but see \cite{Montoya2013} who found similar rates of symptomatic cases across infection number).
Despite these potential differences in reporting rate, modeling work has suggested that the dynamics of primary and secondary infections are closely coupled (and thus not dynamically distinct) under many conditions \cite{Schwartz2005}.
Moreover, given the short time scale of our data (5 years) relative to the period of cross-immunity (roughly 2 years \cite{Reich2013}), we expect that third and fourth infections will be relatively rare.
We thus expect the SEIRS framework, and the assumption of equal symptomatic rates across infection number, to be sufficient for capturing the dengue dynamics of Vit\'oria and the relationship between mosquito abundance and human disease.

In the SEIRS framework, the total human population of Vit\'oria ($N$) is divided into susceptible ($S$), exposed ($E$), infectious ($I$), and immune ($R$) classes. 
Susceptible humans ($S$) become exposed ($E$) through contact with infectious mosquitoes ($V_I$).
Following a latent period ($\frac{1}{\rho}$), exposed humans become infectious ($I$) at which point they can infect susceptible mosquitoes ($V_S$).
Infectious humans recover at rate $\gamma$ and subsequently remain immune ($R$) for a period ($\frac{1}{\delta}$) after which they re-enter the susceptible class.
Similarly, susceptible mosquitoes ($V_S$) become exposed ($V_E$) by biting infectious humans and pass through a temperature-dependent incubation period ($\frac{1}{\rho_{vt}}$) before becoming infectious ($V_I$).
Because the assumption of an exponentially distributed incubation period (implicit in the specification of a differential equation model) is a poor fit to laboratory observations \cite{Chan2012}, we instead implemented a gamma-distributed incubation period by chaining together multiple exposed classes ($V_{Ej}$, taking advantage of the fact that a gamma-distributed random variable can be generated through the sum of exponential random variables with the same rate parameter) \cite{Lloyd2001}.
Total mosquito population size ($V_N$) is controlled by a forced, seasonally varying growth rate ($r(t)$), while the transmission bottleneck is captured with a forced, seasonally varying mortality rate ($d(t)$).
Captured mosquitoes ($V_C$) accumulate at rate $\phi_q \tau_t$, where $\phi_q$ is the per-trap capture rate, and $\tau_t$ is the number of traps deployed in week $t$.

We specified the differential equations governing the human population as:
\begin{align} 
\frac{dS}{dt} &= bN - bS - \lambda \frac{V_{I}}{N} S + \delta R\\
\frac{dE}{dt} &= \lambda \frac{V_{I}}{N} S - (\rho + b)E\\
\frac{dI}{dt} &= \rho E - (\gamma + b)I\\
\frac{dR}{dt} &= \gamma I - (\delta + b)R
\end{align}
while the equations governing the mosquito (vector) population are:
\begin{align}
\frac{dV_N}{dt} & = r(t) V_N - \phi_q \tau(t) V_N \\
\frac{dV_{E1}}{dt} &= \lambda \frac{I}{N} V_S - (4\rho_{v}(t) + d(t) + \phi_q \tau(t))V_{E1}\\
\frac{dV_{E2}}{dt} &= 4\rho_{v}(t) V_{E1} - (4\rho_{v}(t) + d(t) + \phi_q \tau(t))V_{E2}\\
\frac{dV_{E3}}{dt} &= 4\rho_{v}(t) V_{E2}  - (4\rho_{v}(t) + d(t) + \phi_q \tau(t))V_{E3}\\
\frac{dV_{E4}}{dt} &= 4\rho_{v}(t) V_{E3}  - (4\rho_{v}(t) + d(t) + \phi_q \tau(t))V_{E4}\\
\frac{dV_I}{dt} &= 4\rho_{v}(t) V_{E4} - (d(t) + \phi_q \tau(t)) V_I\\
\frac{dV_C}{dt} & = \phi_q \tau(t) V_N\\
V_S &= V_N - V_E - V_I.
\end{align}

We modeled the centered and log-transformed mosquito mortality rate ($\nu$) and the per-capita mosquito growth rate ($r$) as forced harmonic oscillators with natural periods of one year:
\begin{align}
\frac{d^2\nu}{dt^2} &= -\omega^2 \nu + \epsilon_{\nu t}\\
\frac{d^2 r}{dt^2} &= -\omega^2 r + \epsilon_{rt},
\end{align}
where the angular frequency of the oscillator, $\omega = 2\pi / 52$, the mosquito death rate $d(t) = d_0 \exp(\nu(t))$, and
\begin{align}
\epsilon_{\nu t} & \sim \text{Normal}(0, \sigma^2_{\nu})\\
\epsilon_{rt} & \sim \text{Normal}(0, \sigma^2_r),
\end{align}
for each week $t = 1, \dots, 243$.
These stochastically-forced harmonic oscillators provide a flexible framework for generating smooth seasonal oscillations in the latent mosquito processes \cite{Ramsay2017}.

\subsection*{Data model}

To connect the differential equation model to the observed case reports, we added an extra state, $C$, that collects the cumulative number of transitions from the exposed to infectious class (assuming that case reporting coincides with the onset of symptoms).
We then modeled the number of new cases reported in week $t$ ($y_t$) as:
\begin{equation}
y_t  \sim \text{NegBin}(\phi_y (C(t) - C(t-1)), \eta_y),
\end{equation}
where $\phi_y$ is the reporting probability, $C(t) - C(t-1)$ is the number of new infectious humans in week $t$, and $\eta_y$ controls the overdispersion relative to the Poisson distribution.

We similarly modeled the number of mosquitoes trapped in week $t$ ($q_t$) as:
\begin{equation}
q_t \sim \text{NegBin}(V_{C}(t) - V_{C}(t-1), \eta_q),
\end{equation}
where $V_{C}(t) - V_{C}(t-1)$ is the number of new mosquitoes captured in week $t$, and $\eta_q$ controls overdispersion relative to the Poisson distribution.

\subsection*{Parameterization and priors}

Several of the parameters in this model are assumed to be fixed and known (Table 1).
The human population size and average life span (which we use to parameterize the birth/death rate) for Vit\'oria were taken from the 2010 census.
To maintain identifiability, the transmission rate ($\lambda$) was also fixed at literature values.
Lastly, the extrinsic incubation period in mosquitoes was modeled as a function of weekly mean temperature and forced with weather station data obtained from WeatherUnderground \cite{weather}.

The remaining parameters include the epidemiological parameters controlling the average latent, infectious, and immune periods ($\rho$, $\gamma$, $\delta$) and average mosquito lifespan ($d_0$), the initial conditions of the model ($S_0$, $E_0$, $I_0$, $R_0$, $V_{N0}, \nu_0, r_0$), the variances of the latent mosquito processes ($\sigma^2_r$, $\sigma^2_{\nu}$), and the remaining measurement parameters ($\phi_y$, $\phi_q$, $\eta_y$, $\eta_q$).  
Where possible, we specified informative prior distributions for these parameters based on existing laboratory and field studies (see Table 1 for means and Supplemental Material for detailed explanations).

\begin{table}[!ht]
\label{parameters}
\begin{adjustwidth}{-2.25in}{0in} 
\begin{center}
\caption{Model parameters and their values.  Parameters above the rule are fixed, while parameters below the rule are random.  The values in parentheses after the posterior means give the 80$\%$ credible interval. See Supplemental Material for a full description of all prior distributions.}
\begin{tabular}{lp{6cm}lll}
Parameter & Description & Prior mean & Posterior mean & Citation\\
\hline
$N$ & Human population size in Vit\'oria, Brazil & 327801 & & \cite{vitpop} \\
$1/d$ & Human life-span & 76 years & &\cite{vitlong} \\
$\lambda$ & Transmission rate & 4.87 week$^{-1}$ & & \cite{Scott2000}\\
$1/\rho_{v}(t)$ & Extrinsic incubation period & $\frac{1}{7}\exp\left(7.9 - 0.21 T(t) \right)$ weeks & & \cite{Chan2012}\\
$V_{E0}$ & Initial exposed mosquitoes &  0 & &\\
$V_{I0}$ & Initial infectious mosquitoes & 0 & &\\
\hline
$d_0$ & Baseline mosquito mortality rate & 1.47 week$^{-1}$ & 0.88 (0.7, 1.1) &  \cite{Brady2013} \\
$1/\rho$ & Latent period in host & 0.87 weeks  & 1.72 (1.2, 2.3) &  \cite{Chan2012}\\
$\gamma$ & Rate of loss of infectiousness & 3.5 week$^{-1}$ & 3.6 (3.2, 4.1) & \cite{Nguyet2013}\\
$1/\delta$ & Period of cross-immunity & 97 weeks &  114 (72, 160) & \cite{Reich2013}\\
$\sigma_r$ & Standard deviation of mosquito growth rate forcing & $0^*$ & 0.013 (0.01, 0.02) &\\
$\sigma_{\nu}$ & Standard deviation of mosquito mortality rate forcing & $0^*$ & 0.0005 (0.0004, 0.0008) & \\
$S_0$ & Proportion initially susceptible & 0.4 & 0.42 (0.28, 0.57)& \cite{Cardoso2011a} \\
$E_0$ & Number initially exposed & $100$ & 148 (104, 196) & \\
$I_0$ & Number initially infectious & $60$ & 79 (45, 116) &\\
$r_0$ & Initial mosquito population growth rate & 0 & 0.008 (-0.01, 0.02) &\\
$\nu_0$ & Initial unconstrained mosquito mortality rate & 0 & -0.16 (-0.4, 0.08) &\\
$V_{N0}$ & Initial mosquito population size & $2N$ & $1.5N$ $(1.1N, 1.9N)$ &\\
$\phi_y$ & Reporting probability & 0.083 & 0.14  (0.1, 0.18) &\cite{Silva2016}\\
$\log(\phi_q)$ & Log per-trap mosquito capture rate & $-13$  & -13.2 (-13.6, -13)  &\\
$\eta_y$ & Overdispersion of case reports & $0^*$ & 0.12 (0.1, 0.13) &\\
$\eta_q$ & Overdispersion of mosquito trap counts & $0^*$ & 0.14 (0.12, 0.16)&\\
\end{tabular}
\end{center}
\end{adjustwidth}
$^*$ indicates prior mode, rather than mean.
\end{table}

\subsection*{Implementation}

Combining the data, process, and parameter models \cite{Berliner1996}, we summarize the full hierarchical model as:
\begin{align}
y_t | \cdot & \sim \text{NegBin}(\phi_y (C(t) - C(t-1)), \eta_y),
\\
q_t | \cdot &\sim \text{NegBin}(V_{C}(t) - V_{C}(t-1), \eta_q)\\
(\mathbf{C}, \mathbf{V_C}) & = \mathcal{M}(\boldsymbol{\epsilon_r}, \boldsymbol{\epsilon_{\nu}},\boldsymbol{\theta})\\
\epsilon_{rt} & \sim \text{Normal}(0, \sigma^2_r)\\
\epsilon_{\nu t} & \sim \text{Normal}(0, \sigma^2_{\nu})\\
\boldsymbol{\theta} & \sim [\boldsymbol{\theta}]
\end{align}
where $\boldsymbol{\theta}$ is a vector of all the model parameters and initial conditions, and $\mathcal{M}(\boldsymbol{\epsilon_r}, \boldsymbol{\epsilon_{\nu}},\boldsymbol{\theta})$ represents the (numeric) solution to the differential equation model (Eq. 1-14) as a function of $\boldsymbol{\theta}$ and the weekly stochastic forcing terms ($\boldsymbol{\epsilon_{r}}, \boldsymbol{\epsilon_{\nu}}$).
Sampling from the posterior distribution of the parameters in a mechanistic model is difficult due to multimodality, variable parameter sensitivities (e.g., small changes in one parameter may lead to large changes in output, while similar changes in another parameter may have little effect), and potentially strong posterior correlations induced by the nonlinearity of the differential equation model \cite{Reilly2005, Girolami2008, Calderhead2011}.
However, the variable $\epsilon_t$ introduces flexibility to the mechanistic model that remedies lack-of-fit when the process parameters are far from optimal \cite{Leander2014}, thereby reducing multimodality and helping to smooth the posterior surface.
Gradient-based methods like Hamiltonian Monte Carlo (HMC) can then more easily and efficiently traverse the posterior.
Samples from the posterior distribution were generated using HMC implemented in the rstan package \cite{Carpenter2016, Rstan2017} for R \cite{R2016}. 
We ran 3 chains with different starting values for 4,000 iterations each, discarding the first 2,000 as burn-in.
Convergence diagnostics and mixing were evaluated using the shinystan package \cite{shinystan}.
In our implementation, the solution to the differential equation model was approximated with an Euler scheme with a time step of 1 day.  
Timesteps as small as $1/8$ of a day were explored and did not qualitatively change the modeled dynamics.
Code is available from \href{https://github.com/clint-leach/mosquito-recon}{https://github.com/clint-leach/mosquito-recon}.

\subsection*{Mosquito control simulations}

Given a subset of the samples from the posterior distribution as obtained above (2000, taken to reduce computation time), we simulated the effects of a single pulse of mosquito control applied in each week of the first three years of the time series.
Because the city already implements responsive, targeted control with the aim of reducing local mosquito density during an existing outbreak, we focused our simulations on exploring the longer-term feedbacks induced by mosquito control and the ability of an intervention to reduce the disease burden over the following year.
For each week and each posterior sample, we simulated the dynamics resulting from a $5\%$ reduction in the mosquito population implemented at the beginning of that week (affecting susceptible, exposed, and infectious mosquitoes equally).
To capture the likely rapid rebound in mosquito abundance following a single pulse of control \cite{Focks1987, Burattini2008}, we also simulated a $5\%$ increase in mosquito birth rate in the following week to return mosquito abundance to its previous trajectory (without this, the $5\%$ reduction in abundance persists indefinitely).
We then compared the number of cases produced over the year following the control intervention in the control scenario to the same number in the uncontrolled scenario.
A $5\%$ reduction in mosquito abundance was chosen to keep our simulations conservative relative to field estimates of the mortality induced by spraying \cite{Esu2010}, and to avoid pushing the model into the unrealistic range of dengue eradication.

\section*{Results}

The model captured the observed dynamics of both case reports and mosquito trap counts (Figure \ref{timeseries}).
The estimated posterior median case reports explained 91\% of the variation in the observed time series, while the posterior median mosquito trap counts explained 46\% of the variation in the observed time series. 
In addition, posterior predictive checks showed that the model reproduced the total number of cases reported and mosquitoes captured as well as the autocorrelation structure of both time series (with the exception of slightly underestimating the autocorrelation for short lags, Supplemental Figures 6 - 8).
The posterior distributions of the rate of infectious decay ($\gamma$) and the period of cross-immunity ($1/\delta$) did not differ substantially from their priors, suggesting that the Vit\'oria data contained little additional information about these parameters (Supplemental Figure 3).
The estimated latent period in a human host ($1/\rho$, the expected time it takes for an exposed human to become infectious to biting mosquitoes) was influenced more strongly by the data, with a posterior mean of 1.73 weeks compared to a prior mean of 0.87 weeks.
Further, the posterior mean case reporting rate ($\phi$) was 0.14, larger than the prior mean of 0.08.

\begin{figure}[!h]
%\includegraphics[scale = 1]{figures/fig1.eps}
\caption{{\bf Vit\'oria data and model estimates.}
A: weekly observed case reports (points), with corresponding posterior median (black line) and 80\% posterior credible interval (gray band). B: weekly mosquito trap counts (points), with posteriod median (black line) and 80\% posterior credible interval (gray band). C: extrinsic incubation period (EIP; weeks), computed from weekly mean temperature data. D: estimated weekly mosquito mortality rate, with the posterior median (black line) and the 80\% posterior credible interval (gray band).
}
\label{timeseries}
\end{figure}

The estimated weekly mosquito mortality rate varied seasonally, with generally high mortality early in the year and low mortality in August to October (Figure \ref{timeseries}).
This seasonal trend broadly tracked seasonal variation in temperature (Figure \ref{mortality}, correlation coefficient of 0.64) and mosquito trap counts (Figure \ref{timeseries}), though the shape of the annual trajectory differed from year to year.
The posterior distribution of the baseline mortality rate ($d_0$) had a mean of 0.88/week, roughly $60\%$ of the prior mean.
The marginal posterior means of the $\epsilon_{\nu t}$ forcing the mosquito mortality process exhibited a higher-frequency periodic oscillation (Supplementary Figure 1), although the marginal posterior distribution of each $\epsilon_{\nu t}$ overlapped zero.
In addition, the standard deviation of the mortality forcing terms was small relative to the weak prior ($E(\sigma_{\nu}|\mathbf{y}) = 0.01$, Supplemental Figure 2).

\begin{figure}[!h]
%\includegraphics[scale = 1]{figures/fig2.eps}
\caption{{\bf Mosquito mortality and temperature.}
Posterior median mosquito mortality rate as a function of weekly mean temperature (degrees Celsius).
}
\label{mortality}
\end{figure}

The effect of a given mosquito control intervention (i.e., the temporary removal of $5\%$ of the adult population in a given week) on the number of cases in the following year (relative to no control) varied both seasonally and interannually (Figure \ref{control}).
This variation in the effect of control was tightly correlated with the estimated mosquito mortality, with a median posterior correlation between the two time series of 0.96.
As such, the seasonal variation in the effectiveness of control followed the same trend as mosquito mortality rate (although the annual minima in the case ratio time series generally fell 1 to 2 weeks before the minima in the mosquito mortality time series), with the largest reductions in case load resulting from interventions during the dengue off-season (early September for 2008 and 2009, and mid-July for 2010).
Summing over the interannual variation to compute the overall effect of control implemented in a given week of the year, we found that mosquito control was most effective when implemented around  week 34 (late August/early September), reducing the estimated case load by roughly $14\%$ (Figure \ref{control}, bottom panel).
This broadly corresponds to the end of the dry season in Vit\'oria, when both dengue case reports and mosquito trap counts are low.

\begin{figure}[!h]
%\includegraphics[scale = 1]{figures/fig3.eps}
\caption{{\bf The effect of mosquito control as a function of the week in which it was applied.}
Summary of the posterior predicted effect of mosquito control implemented in a given week of the year on the number of cases in the following year (relative to the number of cases expected without control).
Black lines indicate the posterior median, while gray ribbons indicate the $80\%$ credible interval.
The first three panels show the results for control implemented in the years 2008-2010, and the last panel shows the overall effect of control impelmented in a given week of the year, summing over all three years.
For example, mosquito control applied in week 37 of 2008 would have prevented about 13\% of the human cases over the following year (i.e., the caseload would have been 87\% of the expectation without control).
}
\label{control}
\end{figure}

The variation associated with the posterior predicted effect of control implemented in week 34 of the year (i.e., the width of ribbon in Figure \ref{control}) was correlated with the baseline mosquito mortality rate ($d_0$, posterior correlation coefficient of 0.65) and the case reporting probability ($\phi$, posterior correlation coefficient of -0.19).
Simulated mosquito control created the largest reduction in case reports in posterior samples with low baseline mosquito mortality rate and/or high reporting probability, while control was relatively less effective in simulations from samples with high mosquito mortality or low reporting probability (Figure \ref{correlations}).
Thus mosquito control was more effective at reducing disease burden in simulations with long average mosquito lifespans (i.e., low mosquito mortality rates) or low overall prevalence (i.e., fewer undetected cases). 


\begin{figure}[!h]
%\includegraphics[scale = 0.9]{figures/fig4.eps}
\caption{{\bf Posterior correlation between system parameters and the effectiveness of control.}
The y-axis represents the effectiveness of optimally timed control, i.e., the effect of control implmented in the 34th week of the year on the relative number of cases in the following year, summed over 2008, 2009, and 2010.
Each point represets a single sample from the posterior distribution, giving the number of cases in the controlled simulation (relative to the number of cases expected without control) as a function of A: the mean mosquito mortality rate, $d_0$, and B: the case reporting probability, $\phi$, from that posterior sample.
}
\label{correlations}
\end{figure}

\section*{Discussion}

\subsection*{Processes driving effect of mosquito control}

The dynamics of dengue fever, like those of many infectious diseases \cite{Ellner1998,Koelle2004} and ecological systems \cite{Bjornstad2001}, are driven by the combined efforts of intrinsic non-linearities, seasonality, and stochasticity.
Seasonality, in particular, is an important factor in capturing the annual cycle of dengue outbreaks \cite{Wearing2006,Aguiar2011,Reich2013}.
However, the observed seasonality in transmission likely emerges from the combined effects of multiple seasonally-varying components that may be driven by different environmental factors and oscillate in different phases (e.g., mosquito abundance seems to lag slightly behind temperature-driven variation in extrinsic incubation period, Figure \ref{timeseries}).
Integrating these seasonally-varying components into a synthetic measure of transmission potential (e.g., a temperature-dependent effective reproduction number, \cite{Codeco2018}), or more specifically, a measure of the transmission potential of mosquitoes, is difficult.

We positioned the latent mosquito mortality as the link between mosquito abundance, the extrinsic incubation period, and human cases.
In this way, the estimated mosquito mortality rate serves as an index of transmission potential, opening or closing the mosquito life history bottleneck \cite{Smith2012} as necessary to fit to the case reports data.
The resulting seasonality in the estimated trajectory suggests that the seasonality in mosquito abundance and the extrinsic incubation period was not sufficient to capture the observed case reports.
In particular, when mosquitoes were relatively scarce and transmission limited, we estimated a  relatively low mosquito mortality, suggesting that long-lived mosquitoes were required to maintain observed levels of transmission through the off-season.  
On the other hand, when mosquitoes were abundant, we estimated relatively high mosquito mortality rates, suggesting that transmission needed to be damped.

The importance of mosquito longevity in driving disease dynamics highlights the potential effectiveness of control efforts that target adult mosquitoes and disrupt transmission by preventing mosquitoes from living long enough to progress through the extrinsic incubation period to the infectious state and bite a susceptible human.
In fact, this forms the basis for much of the theory of adult mosquito control \cite{Burattini2008, Morrison2008, Smith2012}.
The high correlation between our estimated mosquito mortality and the effect of control confirms this theory, suggesting that control is most effective when it targets long-lived mosquitoes.  
Specifically, our simulated mosquito control interventions were most effective at reducing the disease burden when applied around week 34 (i.e., early September), in the dengue off-season.
The effectiveness of this control was likely driven by the fact that transmission during the off-season was already limited by low mosquito abundance and a relatively high extrinsic incubation period.
Given that limitation, transmission was maintained by relatively few long-lived mosquitoes, making the system vulnerable to perturbation.

On the other hand, we found that a single pulse of control was relatively less effective when implemented during an outbreak, when mosquitoes were abundant (and unlikely to be limiting transmission) but short-lived.
Given the relatively high mosquito mortality rates during this time, exposed and infectious mosquitoes were already fairly ephemeral, such that the relatively small disruption induced by control likely made little difference.
Moreover, due to the large number of infectious human hosts available to transmit to the remaining (and rapidly rebounding) mosquito population, the population of exposed mosquitoes likely recovered quickly \cite{Newton1992, Burattini2008}.
Barsante \emph{et al.} \cite{Barsante2015} and Oki \emph{et al.} \cite{Oki2011} similarly found that control was most effective when applied well before peak prevalence, either during the dry season \cite{Barsante2015} or early in the rainy season \cite{Oki2011} (September is near the end of the dry season in Vit\'oria).
While the immediate effects of mosquito control implemented during the decline phase of an outbreak may be masked by the natually fading transmission intensity \cite{Stoddard2014}, our results nonetheless indicated that disrupting inter-seasonal transmission can be an effective longer-term strategy \cite{Hladish2018}.
Further, although large pulses of imported cases could potentially swamp the effects of early control, in additional simulations we found that our results were robust to the import of 10 infectious humans (roughly the same order as the number of locally reported cases) just before the annual outbreak.

The Bayesian framework allowed us to account for uncertainty across the data, process, and parameter levels of our model \cite{Berliner1996}.
We carried this uncertainty through to our simulations of control interventions \cite{Elderd2006} and found that there was substantial uncertainty in the proportion of cases prevented by a control intervention (i.e., the width of the ribbons in Figure \ref{control}).
Much of this uncertainty could be attributed to posterior uncertainty in the case reporting rate ($\phi$) and the average mosquito mortality rate ($d_0$).
Specifically, we found that control implemented at the overall optimum (week 34) had the largest impact (i.e., the lowest case ratio) in simulations with a low average mosquito mortality and/or a high case reporting rate (Figure \ref{correlations}).
Mosquito mortality and the case reporting rate were correlated with the overall level of mosquito abundance and the overall size of the susceptible population, respectively, suggesting that control was most effective in simulations with fewer mosquitoes and more susceptible humans.  
Hladish \emph{et al.} \cite{Hladish2018} similarly found that simulated indoor residual spraying was more effective when the modeled mosquito abundance was already low.
This suggests that control efforts that reduce the ability of mosquitoes to transmit DENV are most effective for situations in which mosquito abundance is already the limiting component to maintaining transmission (relative to other factors like human immunity).

The posterior correlation between the effectiveness of control, the case reporting rate, and the average mosquito mortality rate emphasizes that the impact of mosquito control is jointly regulated by both mosquito population dynamics and human immune processes.
Similar observations were made by ten Bosch \emph{et al.} \cite{TenBosch2016} who found that models with longer periods of cross-immunity (such that susceptibles replenished more slowly) generated systems in which transmission was more difficult to disrupt with control actions.
As a result of these relationships, efficient deployment of mosquito control, and accurate prediction of its effects, is likely to depend in part on our ability to monitor and predict the dynamics of human immunity.
To meet these needs, existing mosquito monitoring efforts need to be paired with more detailed clinical surveillance \cite{Morrison2008} and tighter estimates of the period of cross-immunity \cite{TenBosch2016} and the number of unreported cases \cite{Silva2016}.
In the absence of, or as a supplement to, such data, mechanistic models like the one developed here, or so-called TSIR (Time-series Susceptiple-Infected-Recovered) frameworks that reconstruct the dynamics of the susceptible class \cite{Finkenstadt2000, Reich2013}, need to be further developed to better inform and understand mosquito control efforts.

\subsection*{Interpretation of the estimated mosquito mortality rate}

The positive correlation between the estimated mosquito mortality rate and the simulated effect of control suggests that mosquito control was most effective when it targeted long-lived mosquitoes during the inter-epidemic periods when mosquito abundance was low.  
However, the fact that the mosquito mortality forcing terms ($\epsilon_{\nu}$, Eq. 13 and 16) were the only source of variability in the transmission process implies that the estimated mosquito mortality time series could have absorbed other sources of stochasticity or model misspecification.
Hooker and Ellner \cite{Hooker2015} provide a framework for diagnosing such model misspecification in differential equation models using forcing functions similar to our implementation of the $\epsilon_{\nu}$.
In that framework, Hooker and Ellner \cite{Hooker2015} estimate nonparametric forcing functions that modify a fitted differential equation model to provide a good fit to the data.
These forcing functions serve as residuals on the time derivatives, and can be more readily interpreted as indicators of lack-of-fit than residuals on the state variables \cite{Hotelling1927, Hooker2015}.
We do not employ the same explicit goodness-of-fit testing framework as \cite{Hooker2015}, but we can inspect our estimated $\epsilon_{\nu}$ forcing terms in the same spirit.

The periodic structure in the time series of the posterior means of the $\epsilon_{\nu}$ (Supplemental Figure 1) suggests that these terms were accounting for more than just noise, and there may have been some unmodeled process influencing fluctuations in mosquito mortality and/or transmission.
Following Hooker and Ellner, we can explore whether this process is likely to result from misspecification of the rates of change of the existing state variables (indicated by a dependence of $\epsilon_{\nu t}$ on other state variables), or from missing state variables altogether (indicated by an additional dependence of $\epsilon_{\nu t}$ on its own lagged values). 
The lack of any apparent relationship between the forcing terms and any of the estimated state variables, combined with the dependence of $\epsilon_{\nu t}$ on previous values (as apparent through the periodic structure), suggest that unmodeled state variables may be the more likely driver of model misspecification.
These unmodeled components could include additional mosquito population dynamic processes (e.g., aquatic stage dynamics, environmental drivers, or control intervention), or epidemiological processes (e.g., multiple circulating serotypes of the dengue virus, subsets of the population with different mixing or risk levels).

Despite these potential sources of model misspecification, our estimated mosquito mortalities nonetheless fell within the reasonable range from the literature \cite{Maciel-de-Freitas2008, Brady2013}.
Moreover, the fact that our estimated mortality rate increased with temperature also broadly agrees with the empirical literature on mosquito survival \cite{Yang2009, Brady2013, Morin2013}.
This suggests that regardless of unexplained structure in the forcing terms, the pattern of case reports was still very well described by realistic seasonal fluctuations in the mosquito mortality rate.
As demonstrated by Reiner \emph{et al.} \cite{Reiner2015} for malaria transmission, estimates of transmission potential can be sensitive to fluctuations in mosquito abundance and age structure.
Moreover, given the broad importance of seasonality in understanding dengue epidemiology \cite{TenBosch2016}, and the role of mosquito mortality and age in driving the effect of control interventions, future work should focus on developing a more complete, predictive understanding of the seasonal drivers of mosquito mortality (including, potentially, control itself).

\subsection*{Additional considerations and extensions}

In addition to epidemiological complexity, dengue dynamics are further complicated by the reporting process.
We estimated a relatively long average intrinsic incubation period ($1/\rho$, posterior mean of 1.7 weeks) relative to our prior mean (0.87 weeks), suggesting possible reporting delays \cite{Reich2016}.
Moreover, the case data to which we fit the model represent reports of ``dengue-like illness,'' without laboratory confirmation, and as such could include cases of other diseases with similar symptoms (e.g., chikungunya or Zika).
However, neither chikungunya nor Zika had emerged as substantial public health threats in Brazil by the end of our time series in late 2012 \cite{PAHO2014, PAHO2015}.
In addition, given the high underreporting rate expected for dengue fever \cite{Silva2016}, and the uncertainty incorporated into the measurement model, we expect misreported cases to have a small effect on our analyses.

Mosquito control interventions can prevent cases by acting on any of the components of vectorial capacity.
We focused on the direct effect of killing adult mosquitoes on transmission, but adult control can also act by reducing egg laying and the number of mosquitoes in the next generation \cite{Brady2016}.
Given the relative simplicity of our mosquito model, we were unable to explore the feedbacks that adult control may induce in mosquito population dynamics, and instead assumed that mosquito populations quickly rebound from any perturbations \cite{Focks1987}.
We expect that capturing these feedbacks would likely reinforce our conclusions about the utility of off-season control, as the disruption to mosquito population dynamics would more strongly limit the ability of the mosquito population to maintain transmission through the off-season.

Although our results suggest that a pulse of adult control in the off-season may be an effective tool for preventing human cases, achieving particular control thresholds or policy goals will likely require deploying a combination of control interventions \cite{Brady2016}.
In fact, it is important to note that the data to which we fit our model implicitly reflect the control efforts already enacted by the city (the effects of which may have influenced our estimates of mosquito demographic rates).
Thus, our mosquito control simulations should be interepreted as exploring the effect of an additional pulse of city-wide control in addition to the existing control activities.
These control efforts are guided by the MI-Dengue system, which uses the trap-level mosquito surveillance data to target areas of high mosquito infestation for control (including source reduction, larvacide, and adulticide)\cite{Eiras2009}.
Although these targeted, reactive interventions are necessary to help reduce local disease risk at times and locations of high mosquito abundance \cite{Pepin2013}, our results suggest that an additional pulse of proactive control in the off-season when mosquitoes are less abundant would minimize human cases. 

The frequent use of spatially-targeted mosquito control highlights the potential for spatial heterogeneities in disease risk within a city.
In particular, structured human movement within the city is likely to induce heterogeneous human-mosquito mixing \cite{Adams2009, Cosner2009a, Stoddard2009}.
In addition, spatial variability in socioeconomic factors within the city may also modulate the extent to which mosquitoes in different parts of the city contribute to disease spread \cite{Mondini2008, Honorio2009, Hu2012, DeMattosAlmeida2007}.
Although our model does not account for these heterogeneities, it was nontheless able to capture the city-wide dynamics well, suggesting that there may be sufficient mixing to appear homogeneous at the city scale.
Further, although our mechanistic model may be able to suggest when a city-wide intervention is likely to be effective, to make the best use of limited resources, spatial prioritization may still be necessary.

\subsection*{Conclusions}

Efforts to connect mosquito abundance to human disease are often hampered by the confounding influences of human immunity and mosquito survival.
The challenges presented by these confounding factors highlight the value of mechanistic information in studying the effect of mosquito control on disease spread.
In the Bayesian context we deployed, this mechanistic knowledge, as formalized in the specification of a differential equation model, can be viewed as part of the prior knowledge on the relationship between mosquito abundance and human disease \cite{Ellner1998, Wikle2010}.
As such, we should neither ignore this mechanistic information, nor encode it so rigidly that it overwhelms the signal in our data.

We developed a simple yet realistic mechanistic model of dengue fever spread that represents the fundamental elements of our prior understanding of dengue epidemiology, while also allowing for uncertainty and flexibility in the fluctuations of mosquito demographic rates.
This mechanistic framework allowed us to capture the critical contribution of long-lived, off-season mosquitoes to the maintenance of transmission and to identify critical intervention points that would not be apparent otherwise.
The fully hierarchical Bayesian framework in which we embedded the mechanistic model allowed for a thorough accounting of uncertainty that was carried through to the evaluation of different control strategies.
This combination of model features helps to meet the need for more effective, biologically grounded, and data-driven dengue control policies and offers a building block on which these tools can be further developed in the future.

\section*{Supporting information}

% Include only the SI item label in the paragraph heading. Use the \nameref{label} command to cite SI items in the text.
\paragraph*{S1 Text.}
\label{S1}
{Full description of model and prior distributions, along with all supplemental figures.}

\section*{Acknowledgements}

CBL would like to thank members of the Webb Lab for feedback and support through innumerable previous iterations of this project.
Any use of trade, firm, or product names is for descriptive purposes only and does not imply endorsement by the U.S. Government.

\nolinenumbers

% Either type in your references using
% \begin{thebibliography}{}
% \bibitem{}
% Text
% \end{thebibliography}
%
% or
%
% Compile your BiBTeX database using our plos2015.bst
% style file and paste the contents of your .bbl file
% here. See http://journals.plos.org/plosone/s/latex for 
% step-by-step instructions.
% 

\bibliographystyle{plos2015}
\bibliography{dengue}


\end{document}

