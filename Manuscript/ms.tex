% Template for PLoS
% Version 3.4 January 2017
%
% % % % % % % % % % % % % % % % % % % % % %
%
% -- IMPORTANT NOTE
%
% This template contains comments intended 
% to minimize problems and delays during our production 
% process. Please follow the template instructions
% whenever possible.
%
% % % % % % % % % % % % % % % % % % % % % % % 
%
% Once your paper is accepted for publication, 
% PLEASE REMOVE ALL TRACKED CHANGES in this file 
% and leave only the final text of your manuscript. 
% PLOS recommends the use of latexdiff to track changes during review, as this will help to maintain a clean tex file.
% Visit https://www.ctan.org/pkg/latexdiff?lang=en for info or contact us at latex@plos.org.
%
%
% There are no restrictions on package use within the LaTeX files except that 
% no packages listed in the template may be deleted.
%
% Please do not include colors or graphics in the text.
%
% The manuscript LaTeX source should be contained within a single file (do not use \input, \externaldocument, or similar commands).
%
% % % % % % % % % % % % % % % % % % % % % % %
%
% -- FIGURES AND TABLES
%
% Please include tables/figure captions directly after the paragraph where they are first cited in the text.
%
% DO NOT INCLUDE GRAPHICS IN YOUR MANUSCRIPT
% - Figures should be uploaded separately from your manuscript file. 
% - Figures generated using LaTeX should be extracted and removed from the PDF before submission. 
% - Figures containing multiple panels/subfigures must be combined into one image file before submission.
% For figure citations, please use "Fig" instead of "Figure".
% See http://journals.plos.org/plosone/s/figures for PLOS figure guidelines.
%
% Tables should be cell-based and may not contain:
% - spacing/line breaks within cells to alter layout or alignment
% - do not nest tabular environments (no tabular environments within tabular environments)
% - no graphics or colored text (cell background color/shading OK)
% See http://journals.plos.org/plosone/s/tables for table guidelines.
%
% For tables that exceed the width of the text column, use the adjustwidth environment as illustrated in the example table in text below.
%
% % % % % % % % % % % % % % % % % % % % % % % %
%
% -- EQUATIONS, MATH SYMBOLS, SUBSCRIPTS, AND SUPERSCRIPTS
%
% IMPORTANT
% Below are a few tips to help format your equations and other special characters according to our specifications. For more tips to help reduce the possibility of formatting errors during conversion, please see our LaTeX guidelines at http://journals.plos.org/plosone/s/latex
%
% For inline equations, please be sure to include all portions of an equation in the math environment.  For example, x$^2$ is incorrect; this should be formatted as $x^2$ (or $\mathrm{x}^2$ if the romanized font is desired).
%
% Do not include text that is not math in the math environment. For example, CO2 should be written as CO\textsubscript{2} instead of CO$_2$.
%
% Please add line breaks to long display equations when possible in order to fit size of the column. 
%
% For inline equations, please do not include punctuation (commas, etc) within the math environment unless this is part of the equation.
%
% When adding superscript or subscripts outside of brackets/braces, please group using {}.  For example, change "[U(D,E,\gamma)]^2" to "{[U(D,E,\gamma)]}^2". 
%
% Do not use \cal for caligraphic font.  Instead, use \mathcal{}
%
% % % % % % % % % % % % % % % % % % % % % % % % 
%
% Please contact latex@plos.org with any questions.
%
% % % % % % % % % % % % % % % % % % % % % % % %

\documentclass[10pt,letterpaper]{article}
\usepackage[top=0.85in,left=2.75in,footskip=0.75in]{geometry}

% amsmath and amssymb packages, useful for mathematical formulas and symbols
\usepackage{amsmath,amssymb}

% Use adjustwidth environment to exceed column width (see example table in text)
\usepackage{changepage}

% Use Unicode characters when possible
\usepackage[utf8x]{inputenc}

% textcomp package and marvosym package for additional characters
\usepackage{textcomp,marvosym}

% cite package, to clean up citations in the main text. Do not remove.
\usepackage{cite}

% Use nameref to cite supporting information files (see Supporting Information section for more info)
\usepackage{nameref,hyperref}

% line numbers
\usepackage[right]{lineno}

% ligatures disabled
\usepackage{microtype}
\DisableLigatures[f]{encoding = *, family = * }

% color can be used to apply background shading to table cells only
\usepackage[table]{xcolor}

% array package and thick rules for tables
\usepackage{array}

% create "+" rule type for thick vertical lines
\newcolumntype{+}{!{\vrule width 2pt}}

% create \thickcline for thick horizontal lines of variable length
\newlength\savedwidth
\newcommand\thickcline[1]{%
  \noalign{\global\savedwidth\arrayrulewidth\global\arrayrulewidth 2pt}%
  \cline{#1}%
  \noalign{\vskip\arrayrulewidth}%
  \noalign{\global\arrayrulewidth\savedwidth}%
}

% \thickhline command for thick horizontal lines that span the table
\newcommand\thickhline{\noalign{\global\savedwidth\arrayrulewidth\global\arrayrulewidth 2pt}%
\hline
\noalign{\global\arrayrulewidth\savedwidth}}


% Remove comment for double spacing
%\usepackage{setspace} 
%\doublespacing

% Text layout
\raggedright
\setlength{\parindent}{0.5cm}
\textwidth 5.25in 
\textheight 8.75in

% Bold the 'Figure #' in the caption and separate it from the title/caption with a period
% Captions will be left justified
\usepackage[aboveskip=1pt,labelfont=bf,labelsep=period,justification=raggedright,singlelinecheck=off]{caption}
\renewcommand{\figurename}{Fig}

% Use the PLoS provided BiBTeX style
\bibliographystyle{plos2015}

% Remove brackets from numbering in List of References
\makeatletter
\renewcommand{\@biblabel}[1]{\quad#1.}
\makeatother

% Leave date blank
\date{}

% Header and Footer with logo
\usepackage{lastpage,fancyhdr,graphicx}
\usepackage{epstopdf}
\pagestyle{myheadings}
\pagestyle{fancy}
\fancyhf{}
\setlength{\headheight}{27.023pt}
\lhead{\includegraphics[width=2.0in]{PLOS-submission.eps}}
\rfoot{\thepage/\pageref{LastPage}}
\renewcommand{\footrule}{\hrule height 2pt \vspace{2mm}}
\fancyheadoffset[L]{2.25in}
\fancyfootoffset[L]{2.25in}
\lfoot{\sf PLOS}

%% Include all macros below

\newcommand{\lorem}{{\bf LOREM}}
\newcommand{\ipsum}{{\bf IPSUM}}

%% END MACROS SECTION


\begin{document}
\vspace*{0.2in}

% Title must be 250 characters or less.
\begin{flushleft}
{\Large
\textbf\newline{Linking mosquito surveillance to dengue fever through Bayesian mechanistic modeling}
}
\newline
% Insert author names, affiliations and corresponding author email (do not include titles, positions, or degrees).
\\
Clinton Leach\textsuperscript{1*},
Colleen Webb\textsuperscript{1},
Kim Pepin\textsuperscript{2},
Alvaro Eiras\textsuperscript{3},
Mevin Hooten\textsuperscript{4},
Jennifer Hoeting\textsuperscript{4}

\bigskip
\textbf{1} Affiliation Dept/Program/Center, Institution Name, City, State, Country
\\
\textbf{2} Affiliation Dept/Program/Center, Institution Name, City, State, Country
\\
\textbf{3} Affiliation Dept/Program/Center, Institution Name, City, State, Country
\\
\bigskip

% Insert additional author notes using the symbols described below. Insert symbol callouts after author names as necessary.
% 
% Remove or comment out the author notes below if they aren't used.


% Use the asterisk to denote corresponding authorship and provide email address in note below.
* clint.leach@colostate.edu

\end{flushleft}
% Please keep the abstract below 300 words
\section*{Abstract}

Our ability to effectively prevent the transmission of the dengue virus through targeted control of its vector, \emph{Aedes aegypti}, depends critically on our understanding of the link between mosquito abundance and human disease risk.
The mosquito and clinical surveillance data necessary to elucidate this link are widely collected, but have yet to be coupled with a modeling framework that accounts for the complex non-linear mechanisms involved in transmission, in particular the critical bottleneck imposed by mosquito mortality.
Here we develop a differential equation model of dengue transmission and embed it in a Bayesian hierarchical framework that allows us to estimate latent time series of mosquito demographic rates from mosquito trap counts and dengue case reports from the city of Vitoria, Brazil.
We then use the fitted model to explore the optimal timing of targeted adult control during the year.
We find that control is most effective when deployed in the third week of the year, as mosquito abundance nears its peak and dengue transmission begins to ramp up.
Intervening at this time disrupts transmission and prevents the amplification and feedback that leads to large outbreaks of disease, an insight not possible without the underlying mechanistic model.
Though the ability of mosquito surveillance to predict human disease is often limited, this highlights the utility of this surveillance when integrated into a larger modeling framework.
Grounding this modeling famework in the actual mechanisms of transmission will help to establish more effective and efficient dengue control policies that allow us to better target mosquitoes that are most responsible for continuing disease spread. 

% Please keep the Author Summary between 150 and 200 words
% Use first person. PLOS ONE authors please skip this step. 
% Author Summary not valid for PLOS ONE submissions.   
%\section*{Author summary}

\linenumbers

% Use "Eq" instead of "Equation" for equation citations.
\section*{Introduction}

Dengue fever is a massive global public health burden, with millions of cases per year in Brazil alone \cite{Bhatt2013}.  
Since the dengue virus (DENV) is vectored by the mosquito \textit{Aedes aegypti}, dengue fever is prevented primarily through mosquito control programs \cite{Achee2015}.
There is limited direct evidence of the success of existing control programs \cite{Esu2010}, possibly due to very low mosquito abundance thresholds for sustaining transmission \cite{Scott2010a}.
Because of this, there is a growing recognition that effective control needs to be guided by high quality vector surveillance, together with quantitative tools that synthesize that surveillance with clinical surveillance, account for local epidemiology, and connect easily to decision making \cite{Morrison2008}.
Moreover, mosquito control needs to be guided by an understanding of the link between mosquito surveillance indices and disease risk so that the mosquitoes most responsible for transmission can be targeted.

Many of the attempts to establish this link have found a weak relationship \cite{Bowman2014, Pepin2015, Cromwell2017}.
However, these attempts often fail to account for the complex, non-linear interactions that mediate the relationship between mosquito abundance and human disease.
In particular, the ability of mosquitoes to contribute to DENV transmission depends critically on the level of prior immunity in the human population \cite{Scott2010a} and on mosquito demographic processes themselves.
The cycle of transmission between humans and mosquitoes is controlled not just by mosquito abundance, but also by mosquito survival relative to the virus' incubation period \cite{Achee2015}.
In fact, whether or not an exposed mosquito will survive long enough to become infectious represents a critical bottleneck in the transmission process.

The importance of these processes in governing human disease risk highlight the need to integrate mechanistic modeling into the quantitative tools used to target control efforts.
Such mechanistic models can often perform better than complex autoregressive statistical models in describing and predicting population dynamics \cite{Reilly2005}.
Moreover, mechanistic models can provide "what-if" tools that can be used to predict the effect of management actions \cite{Buckland2007}.
Differential equation models provide a natural way to encode the relevant biological mechanisms and to link human epidemiology to mosquito demographic processes.
As noted above, mosquito mortality is a critical component of this link, but cannot be estimated from mosquito surveillance alone.
Given its role in mediating transmission, however, mosquito mortality can be informed by a combination of mosquito and clinical surveillance, borrowing from the field of integrated population modeling \cite{Schaub2010}.

Many cities in Brazil generate the necessary vector surveillance data through the use of the MI-Dengue system, implemented by the company Ecovec.
This system deploys a city-wide grid of mosquito sticky traps (MosquiTRAPs) that can be used to target control in highly infested areas \cite{Eiras2009}.
This system has been shown to be effective, preventing an estimated 27,191 cases of dengue fever from 2007 - 2011 across the cities in which it was deployed \cite{Pepin2013}.
Despite this, the mosquito indices and thresholds used to prioritize control efforts remain somewhat ad-hoc.

In this paper, we develop a modeling framework to integrate MI-Dengue and clinical surveillance from the city of Vitoria, Brazil in order to provide data-driven and mechanistically-grounded guidance for targeted mosquito control.
Specifically, we develop a mechanistic differential equation model of dengue transmission that we embed in a hierarchical Bayesian statistical framework.
This allows us to estimate latent time series of mosquito demographic parameters from available time series of mosquito trap counts and reported cases of dengue fever.
We then use the fitted model to explore the optimal timing of adult mosquito control interventions.

\section*{Methods}

\subsection*{Study system and data}

Vitoria is a coastal city and the capital of the state of Espirito Santo, Brazil, with a population of 327,801.
Since 2008, the Ecovec has monitored mosquito abundance for the city using approximately 1327 sticky traps (MosquiTRAP, \cite{Eiras2009}) arranged in a grid across the city.
Each trap is checked weekly and the mosquitoes inside counted and identified, providing us with 243 weeks (2008 through week 34 of 2012) of counts of gravid female \emph{Aedes aegypti}.
Dengue is a mandatory notifiable disease, and thus the city's Ministry of Health Secretary maintains a database of weekly notified probable dengue cases (i.e. medical care sought for dengue-like symptoms) for this same time period.

\subsection*{Process Model}

Dengue epidemiology is complicated considerably by the presence of four simultaneously circulating serotypes (referred to as DENV-1, DENV-2, DENV-3, DENV-4).
Infection with one serotype confers life-long immunity to that serotype, along with temporary immunity to other serotypes.  
As this cross-immunity wanes, antibodies from the previous infection can result in antibody-dependent enhancement (ADE), wherein human hosts are more susceptible to infection with the other serotypes and more likely to develop severe symptoms (i.e. dengue haemorrhagic fever or dengue shock syndrome).
The strength and duration of these different inter-serotype interactions are not well understood, though different models suggest that temporary cross-immunity alone (without ADE) is sufficient to reproduce observed multi-annual dynamics in Thailand \cite{Wearing2006,Reich2013}.
Similarly, \cite{Aguiar2013} find that capturing primary and secondary infections and the period of cross-immunity is critical, but that explicitly including all four serotypes (at great cost to model complexity) does not perform much better than a two serotype model.

Explicitly accounting for interactions between serotypes, even only two of the four, leads to a large and complex mechanistic model.
Moreover, since dengue case reports do not include information on serotype, we do not have enough information to inform the dynamics of individual serotypes.
As such, we simplify the model of \cite{Wearing2006} to an SEIRS framework, which drops serotype-specific dynamics but preserves the period of cross-immunity and the possibility of reinfection.
In this framework, susceptible humans ($S$) become exposed ($E$) through contact with infectious mosquitoes ($V_I$).
Following a latent period ($\frac{1}{\rho}$), exposed humans become infectious ($I$) at which point they can infect susceptible mosquitoes ($V_S$).
Infectious humans recover at rate $\gamma$ and subsequently remain immune for a period ($\frac{1}{\delta}$) after which they re-enter the susceptible class.

Similarly, susceptible mosquitoes ($V_S$) become exposed by biting infectious humans and pass through a temperature-dependent incubation period ($\frac{1}{\rho_{vt}}$) before becoming infectious ($V_I$).
Total mosquito population size ($V_N$) is controlled by stochastic, seasonally varying growth rate ($r(t)$) and death rate ($d(t)$).
Captured mosquitoes ($V_C$) accumulate at rate $\phi_q \tau_t$, where $\phi_q$ is the per-trap capture rate, and $\tau_t$ is the number of traps deployed in week $t$.

The differential equations governing the human population are then given as:
\begin{align} 
\frac{dS}{dt} &= bN - bS - \lambda \frac{V_{I}}{N} S + \delta R\\
\frac{dE}{dt} &= \lambda \frac{V_{I}}{N} S - (\rho + b)E\\
\frac{dI}{dt} &= \rho E - (\gamma + b)I\\
\frac{dR}{dt} &= \gamma I - (\delta + b)R
\end{align}
while the equations governing the mosquito (vector) population are:
\begin{align}
\frac{dV_N}{dt} & = r V_N - \phi_q \tau_t V_N \\
\frac{dV_E}{dt} &= \lambda \frac{I}{N} V_S - (\rho_{vt} + d + \phi_q \tau_t)V_E\\
\frac{dV_I}{dt} &= \rho_{vt} V_E - (d + \phi_q \tau_t) V_I\\
\frac{dV_C}{dt} & = \phi_q \tau_t V_N\\
V_S &= V_N - V_E - V_I.
\end{align}

We model the centered and log-transformed mosquito mortality rate ($\nu$) and the per-capita mosquito growth rate ($r$) as forced harmonic oscillators with natural periods of one year:
\begin{align}
\frac{d^2\nu}{dt^2} &= -\omega^2 \nu + \epsilon_{\nu t}\\
\frac{d^2 r}{dt^2} &= -\omega^2 r + \epsilon_{rt},
\end{align}
where the angular frequency of the oscillator, $\omega = 2\pi / 52$, the mosquito death rate $d(t) = d_0 \exp(\nu(t))$, and
\begin{align}
\epsilon_{\nu i} & \sim \text{Normal}(0, \sigma_{\nu})\\
\epsilon_{ri} & \sim \text{Normal}(0, \sigma_r),
\end{align}
for $i = 1 \dots 243$.
These stochastically-forced harmonic oscillators provide a flexible framework for generating smooth seasonal oscillations in the latent mosquito processes \cite{Ramsay2017}.

\subsection*{Data model}

To connect the differential equation model to the observed case reports, we add an extra state, $C$, that collects the cumulative number of transitions from the exposed to infectious class (assuming that case reporting coincides with the onset of symptoms).
We then model the number of new cases reported in week $t$ ($y_t$) as:
\begin{equation}
y_t  \sim \text{NegBin}(\phi_y (C_t - C_{t-1}), \eta_y),
\end{equation}
where $\phi_y$ is the reporting probability, $C_t - C_{t-1}$ is the number of new infectious humans in week $t$, and $\eta_y$ controls the overdispersion relative to the Poisson distribution.

We similarly model the number of mosquitoes trapped in week $t$ ($q_t$) as:
\begin{equation}
q_t \sim \text{NegBin}(V_{Ct} - V_{Ct-1}, \eta_q),
\end{equation}
where $V_{Ct} - V_{Ct-1}$ is the number of new mosquitoes captured in week $t$, and $\eta_q$ controls overdispersion relative to the Poisson distribution.

\subsection*{Parameterization and priors}

Several of the parameters in this model are assumed to be fixed and known (Table 1).
The human population size and average life span (which we use to parameterize the birth/death rate) for Vitoria are taken from the 2010 census.
To maintain identifiability, the transmission rate ($\lambda$) and case reporting probability ($\phi_y$) are also fixed at literature values.
Lastly, the extrinsic incubation period in mosquitoes is modeled as a function of weekly mean temperature and forced with weather station data obtained from WeatherUnderground.

The remaining parameters include the epidemiological parameters ($\rho$, $\gamma$, $\delta$), the initial conditions of the model ($S_0$, $E_0$, $I_0$, $R_0$, $V_{N0}, \nu_0, r_0$), the variances of the latent mosquito processes ($\sigma^2_r$, $\sigma^2_{\nu}$), and the remaining measurement parameters ($\phi_q$, $\eta_y$, $\eta_q$).  
Where possible, we place informative priors on these parameters based on existing laboratory and field studies (see Table 1 for means and Supplemental Material for detailed explanations).

\begin{table}[!ht]
\label{parameters}
\begin{adjustwidth}{-2.25in}{0in} 
\begin{center}
\caption{Model parameters and their values.  Parameters above the rule are fixed, while parameters below the rule are random, with the value giving the prior mean.}
\begin{tabular}{llll}
Parameter & Description & Value & Citation\\
\hline
$N$ & Human population size & 327801 & Vitoria Census\\
$1/d$ & Human life-span & 76 years & Vitoria Census\\
$\lambda$ & Transmission rate & 4.87 week$^{-1}$ & \cite{Scott2000}\\
$\phi_y$ & Reporting probability & 0.083 & \cite{Silva2016}\\
$1/\rho_{vt}$ & Extrinsic incubation period & $7\exp \left( 0.2 T_t - 8 \right)$ & \cite{Chan2012}\\
$d_0$ & Mean mosquito death rate & 1.47 week$^{-1}$ & \cite{Brady2013} \\
$V_{E0}$ & Initial exposed mosquitoes &  0 & \\
$V_{I0}$ & Initial infectious mosquitoes & 0 & \\
\hline
$1/\rho$ & Latent period in host & 0.87 weeks  & \cite{Chan2012}\\
$\gamma$ & Rate of loss of infectiousness & 3.5 week$^{-1}$ & \cite{Nguyet2013}\\
$1/\delta$ & Period of cross-immunity & 97 weeks &  \cite{Reich2013}\\
$S_0$ & Proportion initially susceptible & 0.4 & \cite{Cardoso2011a} \\
$E_0$ & Proportion initially exposed & $8\times 10 ^ {-5}$ & \\
$I_0$ & Proportion initially infectious & $8\times 10 ^ {-5}$ & \\
$\phi_q$ & Per-trap mosquito capture rate & $2 \times 10^{-6}$ week$^{-1}$ & 
\end{tabular}
\end{center}
\end{adjustwidth}
\end{table}

\subsection*{Implementation}

Sampling from the posterior distribution in non-linear differential equation models is often very difficult due to multimodality, variable parameter sensitivities (e.g. small changes in one parameter may lead to large changes in output, while similar changes in another parameter may have little effect), and potentially strong posterior correlations among parameters \cite{Reilly2005, Girolami2008, Calderhead2011}.
In addition, the computational burden of numerically solving the differential equation model at each iteration means that sampling efficiency is crucial.
These problems can be mitigated somewhat by introducing stochasticity through the $\epsilon_t$ in the mosquito demographic processes.
This allows the data to pull the latent states closer even when the parameters are far from optimal values \cite{Leander2014}.
This reduces multimodality and allows gradient-based methods like Hamiltonain Monte Carlo (HMC) to more easily and efficiently traverse the posterior.
Samples from the posterior distribution were thus generated using HMC implemented in the rstan package \cite{Carpenter2016, Rstan2017} for R \cite{R2016}. 
We ran 3 chains with different starting values for 10,000 iterations each, discarding the first 5,000 as burn-in.
Within stan, the solution to the differential equation model was approximated with an Euler scheme with a time step of 1 day.  

\subsection*{Control simulations}

Given samples from the posterior distribution as obtained above, we can simulate the efficacy of different control strategies.
In particular, we investigate the optimal timing of a single pulse of control within the year (e.g., if the city only has enough resource to fog for adult mosquitoes once a year, in what week is that control going to prevent the most cases?).
To obtain the posterior distribution of the number of cases prevented by implementing control in week $i$ each year, for each posterior sample $k$, we simulate an increase in the latent mosquito death rate during each week $i$, e.g.,
$\hat{d}^{(k)}_{ji} = (1 + \Delta) d^{(k)}_{ji}$, where $j$ indexes the year, and $i$ indexes the week of the year.
Then we sum the total number of cases reported in this scenario and compare to the data to get an estimate of the number of cases prevented.
To keep our simulations conservative relative to field estimates of the mortality induced by spraying \cite{Esu2010}, and to avoid pushing the model into the unrealistic range of dengue eradication, we set $\Delta = 0.05$.
We also only simulate the effect of control in the middle three years, as the effectiveness of control in the first year is strongly affected by initial conditions, and we lack complete data for the last year.

\section*{Results}

The model is able to capture the observed dynamics of both case reports and mosquito trap counts (Figure \ref{timeseries}).
The estimated posterior median case reports explain 91\% of the variation in the observed time series, while the posterior median mosquito trap counts explain 45\% of the variation in the observed time series. 
In addition, posterior predictive checks show that the model reproduces the the total number of cases reported and mosquitoes captured as well as the autocorrelation structure of both time series (with the exception of slightly underestimating the autocorrelation for short lags, Supplemental Figures 5 - 7).
The posterior distributions of the three epidemiological parameters ($\rho, \gamma, \delta$) differ from their priors, though they do still cover the literature values used to set their prior means (Supplemental Figure 3).
The estimated latent period in a human host ($1/\rho$) is influenced most strongly by the data, with a posterior mean of 1.4 weeks compared to a prior mean of 0.87 weeks.

The estimated latent mosquito demographic rates are smooth and strongly seasonal, with fairly narrow credible intervals (Figure \ref{latent}).
Mosquito mortality rates fluctuate within the range estimated by \cite{Brady2013} and generally increase with temperature (Figure \ref{temp}).
The underlying process variances are small ($\sigma_{\nu} = 0.01$, $\sigma_r = 0.0005$, Supplemental Figure 2), and there is little residual structure in the process noise series, though there is a weak periodic signal in the $\epsilon_{\nu}$ (Supplemental Figure 1).

Modifying these estimated demographic rates to simulate targeted control efforts, we find that control applied in the third week of the year is the most effective at preventing cases of dengue fever (though weeks 2 and 4 are also very close, Figure \ref{control}A).
In fact, increasing the mosquito mortality rate by just 5\% in the third week of 2009, 2010, and 2011 would prevent 13,988 (reported) cases of dengue fever (with 80\% credible interval of 11,130 to 15,811).
Most of these gains come by way of substantially reducing the size of the large 2011 outbreak, while also further damping the smaller 2010 and 2012 outbreaks (Figure \ref{control}B).
Targeting larvae instead of adults (simulated by reducing the birth rate ($b_t = r_t + d_t$) by 5\%) produces nearly identical results (Supplemental Figure 8).

The third week of the year is generally very early in the dengue season, well before cases peak (Figure \ref{timing}).
In the two years with large outbreaks (2009 and 2011), the third week of the year is near the inflection point where the disease burden begins to grow rapidly.  
The third week is also generally before the estimated seasonal peak in mosquito abundance, which occurs between weeks 6 and 9 (Figure \ref{timing}).
Annual peaks in estimated mosquito mortality rate are more variable, but tend to fall between weeks 2 and 7.

\begin{figure}[!h]
\includegraphics[angle = 270, scale = 1]{figures/fig1.eps}
\caption{{\bf Vitoria data and model fits.}
Weekly observations from 2008 to week 34 of 2012 (points), with corresponding median posterior prediction (line) and 80\% credible interval (gray band). A: reported dengue cases. B: total number of trapped mosquitoes.
}
\label{timeseries}
\end{figure}

\begin{figure}[!h]
\includegraphics[angle = 270, scale = 1]{figures/fig2.eps}
\caption{{\bf Estimated latent mosquito growth and death rates.}
Median posterior estimate (line) and 80\% credible interval (gray band).  A: weekly mosquito population growth rate ($r_t$). B: weekly mosquito death rate ($d_t$).
}
\label{latent}
\end{figure}

\begin{figure}[!h]
\includegraphics[angle = 270, scale = 1]{figures/fig3.eps}
\caption{{\bf Estimated mosquito death rate as a function of temperature.}
Weekly median posterior death rate ($d_t$) plotted against weekly mean temperature (Celsius) in Vitoria, Brazil.
}
\label{temp}
\end{figure}

\begin{figure}[!h]
\includegraphics[angle = 270, scale = 1]{figures/fig4.eps}
\caption{{\bf Cases prevented by mosquito control.}
A: Number of reported cases of dengue fever prevented by increasing the mosquito death rate by 5\% during a given week each year in 2009, 2010, and 2011. Median posterior prediction (line) and 80\% credible interval. B: Weekly predicted case reports resulting from implementing control in week 3 of each year (timing indicated by vertical red lines).  Median posterior case reports from model without (dashed line) and with control (solid line) and 80\% credible interval of controlled dynamics (gray band).
}
\label{control}
\end{figure}

\begin{figure}[!h]
\includegraphics[angle = 270, scale = 1]{figures/fig5.eps}
\caption{{\bf Timing of optimal control relative to disease and mosquito dynamics.}
The vertical black line indicates week 3, during which mosquito control produces the largest decline in predicted dengue case reports. Each colored line represents a different year: dark green = 2008, orange = 2009, purple = 2010, pink = 2011, light green = 2012. A: posterior median estimate of new dengue cases per week. B: posterior median estimate of number of mosquitoes per person each week. C: posterior median estimate of weekly mosquito death rate.
}
\label{timing}
\end{figure}

\section*{Discussion}

Targeted mosquito control can be an effective tool in reducing the disease burden within a city.
When faced with limited resources with which to deploy that control, these results suggest that focusing efforts early in the year, specifically in the second to fourth weeks, will lead to the largest reductions in disease.
Because adult control (increasing the mortality rate) and larval control (decreasing the birth rate) produce nearly identical reductions in disease, the effect of control is likely driven by reducing the standing population of adult mosquitoes, rather than by limiting the transmission potential of exposed or infectious mosquitoes.
The lag between optimal control and both mosquito population dynamics and human disease highlights the importance of early intervention and the need to account for the nonlinear processes governing transmission.
Targeting the mosquito population early in the year as dengue just begins to spread can disrupt the transmission cycle and prevent further amplification and feedback.

Discussion of the effectiveness of adult control often focuses on the fact that it produces only a brief decline in mosquito abundance \cite{Newton1992, Esu2010}.
When applied at the peak of an epidemic, this reduction in abundance is likely to have a relatively small effect, as the population rebounds while there are still a large number of infectious human hosts available \cite{Newton1992, Burattini2008}. 
However, adult control can also have an effect by shortening mosquito lifespan and preventing mosquitoes from living long enough to progress through the extrinsic incubation period and become infectious.
This forms the basis for much of the theory of adult control \cite{Burattini2008, Morrison2008, Smith2012}, but modeling this effect in a standard differential equation framework is difficult.
In particular the assumption of exponential wait times implicit in a compartmental model is perhaps ill-suited to modeling the race between mosquito lifespan and incubation period.
The long tails of the exponential distribution may overestimate the probability that a mosquito survives long enough to become infectious and underestimate the effect of decreasing mosquito lifespan.
More accurate models of the extrinsic incubation period could incorporate explicit delays (i.e. assuming a fixed incubation period \cite{Burattini2008}), or additional chained exposed compartments (i.e. assuming the incubation period is gamma-distributed, \cite{Lloyd2001}).
Including such mechanisms in our model would allow us to better differentiate the effects of larval and adult control and provide a more accurate picture of the effectiveness of adult control.
In fact, modeling studies that assume exponential wait times suggest that adult control is likely to be ineffective when applied during an epidemic \cite{Newton1992, Pinho2010}, while similar models that include explicit delays for the extrinsic incubation period show adult control being much more effective \cite{Burattini2008}.

Thus, our model may underestimate the effect of adult mosquito control applied later in the year when infectious mosquitoes are prevalent.
However, our model nonetheless shows that reducing mosquito abundance early in the year can be highly effective at preventing dengue cases later in the year.
This effectiveness is driven primarily by the timing and could be achieved by either adult or larval control strategies.
Though the mechanisms above might suggest further advantages for adult control, a mixed strategy of adult and larval control may be most effective in the long-term \cite{Burattini2008, Pinho2010}.
Fully exploring such combined strategies requires a more complete model of mosquito life history and the lags in population dynamics induced by the mosquito's aquatic stage.
These lags likely mean that larval control should be applied prior to week 3, in order to achieve the desired reduction in the adult population in week 3.
Models of mosquito development can be very complex, e.g. \cite{Magori2009}, but our model could be extended to account for these lags through a compartment for the aquatic phase (as in \cite{Burattini2008, Pinho2010}), though at the cost of introducing several new parameters.

As implemented, mosquito mortality is the only source of uncertainty/stochasticity in the transmission process.
As a result, the estimated mosquito mortality could be soaking up other sources of stochasticity or model misspecification.
Our implementation of the mortality process is conceptually similar to the empirical forcing functions developed by \cite{Hooker2015}.
In that framework, \cite{Hooker2015} estimate nonparametric functions in time that modify either the differential equations or specific model parameters to provide a good fit to the data.
These forcing functions serve as residuals on the time derivatives, and can be more readily interpreted as indicators of lack-of-fit than residuals on the state variables.
Though we do not employ the same explicit testing framework as \cite{Hooker2015}, we treat the estimated mosquito mortality time series in the same spirit.
In particular, the weak periodic structure in the estimated $\epsilon_{dt}$ suggests that there may be some unmodeled, higher frequency process influencing transmission, either through the death rate itself or through another process.
Despite this, our estimated mosquito mortalities generally fall within the reasonable range from the literature \cite{Maciel-de-Freitas2008, Brady2013}.
The fact that our estimated mortality rate varies seasonally and increases with temperature above 20 degrees also agrees with the empirical literature on mosquito survival \cite{Yang2009}, though our estimated relationship is perhaps more extreme than usually estimated in laboratory studies.
These observations suggest that our estimated mosquito mortality rates may actually be capturing the patterns of mosquito mortality in this system and are not obviously influenced by other sources of uncertainty or misspecification.

Though the estimated mosquito mortality process seems biologically reasonable, we are still employing it to link mosquito abundance and human disease at a relatively large spatial scale, assuming homogeneous mixing between mosquitoes and humans throughout the city.
However, structured human movement within the city is likely induce heterogeneous human-mosquito mixing \cite{Adams2009, Cosner2009a, Stoddard2009}.
In addition spatial variability in socioeconomic factors within the city may also modulate the extent to which mosquitoes in different parts of the city contribute to disease spread \cite{Mondini2008, Honorio2009, Hu2012, DeMattosAlmeida2007}.
This heterogeneity can have substantial consequences for disease dynamics.
In fact, \cite{Dye1986, Hasibeder1988} found that introducing heterogeneous mixing to models of vector-borne disease spread always increases $R_0$ relative to models with homogeneous mixing.

Though this suggests that our model may overestimate the efficacy of mosquito control, it also points a way towards more efficient and effective control methods.
Though our results suggest that applying control early in the year is critical to disrupting city-wide disease spread, targeting the city's entire mosquito population is likely infeasible and spatial prioritization of control is also necessary.
If disease risk is driven by high levels of contact with a relatively small proportion of the mosquito population \cite{Yoon2012}, control policies that preferentially target these subsets of the mosquito population are likely to be very effective.
However, these subsets can be difficult to identify in practice.
One possible strategy is to employ contact-tracing, which can identify and target specific premises that have been directly implicated in transmission \cite{Vazquez-Prokopec2017}.
Though effective, this strategy is extremely labor-intensive and would be difficult to implement in endemic areas.

Due to these difficulties, new methods are needed that can make use of currently available surveillance data to more effectively guide spatial prioritization of control.
Though aggregated for our model, MosquiTRAP surveillance data are available at the level of the individual trap, and case reports for the city of Vitoria are available for each of the city's 75 neighborhoods.
To interface with these data, spatial structure and human movement could be incorporated into models of dengue transmission, e.g. through the methods outlined in \cite{Cosner2015}, but at the cost of substantial statistical and computational burden.
For instance, embedding the model presented here into a metapopulation framework based on the city's neighborhoods leads to a very large, unwieldy mechanistic model.
Moreover, without external data on human movement (possible sources of which are discussed in \cite{Stoddard2009}), it would be very difficult to identify its role in disease dynamics.
These dimensionality and identifiability issues would likely exacerbate existing difficulties inherent in fitting ODE models to data \cite{Girolami2008, Calderhead2011}.
Developing such spatial models and the methods necessary to deal with them is a promising area of future research.
In the absence of such methods, spatial targeting of control based on more traditional methods (e.g. local entomological thresholds) could be combined with temporal targeting based on city-scale mechanistic modeling.
This would allow mosquito control efforts to exploit the critical timing identified by the mechanistic modeling, while also taking advantage of the spatial richness of the surveillance data to allocate limited resources within that window.

\subsection*{Conclusions}

Mosquito surveillance is a valuable tool in managing dengue fever, but is most powerful when integrated into a larger modeling framework.
We have developed a simple yet realistic mechanistic model of dengue fever spread that allows us to combine mosquito and clinical surveillance to estimate the latent mosquito demographic rates that connect mosquito abundance and human disease.
The mechanistic framework allows us to capture important lags and feedbacks in the transmission process and to identify critical intervention points that would not be apparent otherwise.
The fully hierarchical Bayesian framework in which the mechanistic model is embedded allows for a thorough accounting of uncertainty that is carried through to the evaluation of different control strategies.
This combination of model features helps to meet the need for more effective, biologically grounded, and data-driven dengue control policies and offers a building block on which these tools can be further developed in the future.

\section*{Supporting information}

% Include only the SI item label in the paragraph heading. Use the \nameref{label} command to cite SI items in the text.
\paragraph*{S1 Text.}
\label{S1}
{\bf Full description of model and priors, along with all supplemental plots.}

\section*{Acknowledgements}

Thanks!

\nolinenumbers

% Either type in your references using
% \begin{thebibliography}{}
% \bibitem{}
% Text
% \end{thebibliography}
%
% or
%
% Compile your BiBTeX database using our plos2015.bst
% style file and paste the contents of your .bbl file
% here. See http://journals.plos.org/plosone/s/latex for 
% step-by-step instructions.
% 

\bibliographystyle{plos2015}
\bibliography{dengue}


\end{document}

